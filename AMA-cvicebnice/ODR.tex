
\chapter{Obyčejné diferenciální rovnice}
obyčejná diferenciální rovnice -- vztah mezi neznámou funkcí jedné proměnné a jejími derivacemi;
řád ODR -- řád nejvyšší derivace v ODR.

ODR $n$-tého řádu: $F(x,y,y',\ldots,y^{(n)})=0$;
řešení (integrál) ODR -- funkce, která vyhovuje rovnici v daném oboru.

ODR 1. řádu: $F(x,y,y')=0$, resp. $y'=f(x,y)$ (rozřešená vzhledem k derivaci).

\subsection{Rovnice se separovanými proměnnými}
   Rovnici ve tvaru ve tvaru $f(x)+g(y)y'=0$, kde $f,g$ jsou funkce nazýváme
   rovnicí se separovanými proměnými.
   Je-li $f$ spojitá na $(a,b)$ a $g$ je spojitá na $(c,d)$,
   potom každé řešení na $I\subset(a,b)$ splňuje na $I$ rovnici
   \[
      \int f(x)\,dx + \int g(y)\,dy = C, \; C\in\mathbb{R}.
   \]
   Funkce určená na intervalu $J\subset(c,d)$ touto rovnicí, je řešením diferenciální rovnice na $J$.

   {\bf Příklad}
   Řešte rovnici $yy'=e^y$.

   Prvně si uvědomíme, že ve standardním tvaru je pravá strana rovnice $e^y/y$ což je lipschitcovská funkce
   proměnné $y$ všude kromě bodu $y=0$.
   Derivaci napíšeme jako symbolický zlomek diferencí:
   \[
        y\frac{\d y}{\d x}=e^y
   \]
   a úpravami převedeme na jednu stranu všechny členy  obsahující $y$ a na druhou čelny obsahující $x$:
   \[
        ye^{-y}\,\d y=1\,\d x
   \]
   nyní integrujeme (per partes) a dostáváme
   \[
      -(1+y)e^{-y}+C=x
   \]
   Což je předpis pro řešení zapsané jako funkce $x(y)$. V souladu s větou o existenci a jednoznačnosti
   zjistíme, že pro $y=0$ není rovnice splněna. Ovšem na intervalech $y\in(-\infty,0)$ a  $y\in(0,\infty)$
   existuje jednoznačné řešení. Na každém z nich můžeme volit jinou konstantu $C$. Pokud bude dána počáteční
   podmínka např. $x(1)=0$, bude řešení jednoznačné jen na intervalu $y\in(0,\infty)$. Nicméně, když si nakreslíme graf,
   zjistíme, že na intervalu $y\in(-\infty,0)$ existuje právě jedna volba $C$, tak aby na sebe obě části navazovaly
   v bodě nula. Tomu říkáme, že existuje jednoznačné spojité prodloužení.
\section{Maximalni reseni}
Kterymi body roviny prochazi prave jedno maximalni reseni rovnice
\[
  xy'-y=0 ?\quad [R^2\setminus \{x=0\}]
\]

Najdete vsechna max. res. rce $3y'=5y^\frac{5}{2}$ splnujici $y(-2)=-1$ a
\[
  a)\ y(0)=\frac12\quad 
  b)\ y(0)=2\quad
  c)\ y(0)=0.
\]
Reseni ma tri vetve $y=-(-x+C_1)^{\frac53}$ pro $x\le C_1$;
$y=0$ na $(C_1,C_2)$;
$y=(x+C_2)^{\frac53}$, pro $x\ge C_2$;
a) existuje b) neexistuje c) horni cast zacina v libovolnem kladnem bode

\section{Linearni rovnice, variace konstant}
uloha
\[
  y'+\frac{y}{x}=x
\]
reseni homegeni rovnice:
\[
  y=C x^{-1}
\]
dosadime
\[
  C' x^{-1} + C [- x^{-2} + x^{-2}]=x  
\]
tedy
\[
  C=\frac{x^3}{3}+c; y=\frac{x^2}{3}+\frac{c}{x}.
\]
Reseni ma 2 nezavisle vetve na $(-\infty,0)$ a na $(0,\infty)$.

\section{substituce}
\subsection{Bernoulliho}
\[
  y'+f(x)y=g(x)y^\alpha,\ \alpha\ne 1
\]
substituce $z=y^{1-\alpha}$

uloha:
\[
  xy'-y=x^2y^{-1}
\]
reseni
\[
  \frac12 x (y^2)' - y^2 = x^2, 
\]
reseni homogeni rce: $z=Cx^2$
cele reseni: $z=(2\ln x+C)x^2$, $y=x\sqrt{2\ln x +C}$.

\subsection{Homogenni rovnice}
\[
  y'=f(\frac{y}{x})
\]
substituce $z=\frac{y}{x}$ vede na $xz'=f(z)-z$

stejna uloha, reseni:
\[
  y'=\frac{x}{y}+\frac{y}{x}\ ->\ xz'=z^{-1}
\]
tedy $z=\sqrt{2\ln x+C}$ a $y=x\sqrt{2\ln x +C}$.

Dalsi ulohy na substituci:\\
---
\[
  x+y-2+y'(x-y+4)=0  
\]
substituce $X=x+1$, $Y=y-3$ vede na
\[
  Y'=-\frac{X+Y}{X-Y}=-\frac{1+z}{1-z},
\]
kde $z=\frac{Y}{X}$  je homogenni substituce, ktera da
\[
  \frac{z^2-2z-1}{1-z}=Xz',
\]
substituci za jmenovatele pocitame
\[
  \int \frac{1-z}{z^2-2z-1} dz=-\frac12\ln(z^2-2z-1)
\]
tedy
\[
  z^2-2z-1=CX^{-2}, Y^2-2XY-X^2=C.
\]

---
\[
  (2e^y-x)y'=1
\]
substituce $z=2e^y-x$, reseni 
\[
y=\ln\big(\frac12(\sqrt{x^2+C}+x\big)
\]


\subsubsection{Bernoulliho rovnice} 
   Někdy nelze separaci proměnných provést přímo, ale je třeba nejprve provést vhodnou transformaci rovnice.
   Příkladem je tzv. {\it Bernoulliho rovnice} ve tvaru
   \[y'+f(x)y=g(x)y^(\alpha)\]
   kde $f$ a $g$ jsou funkce spojité na $(a,b)$, a $\alpha$ je reálný parametr z množiny $\Real\setminus\{0,1\}$.
   Pro $\alpha>0$ je jedním z řešení funkce $y=0$. Ostatní řešení získáme následovně.
   Násobeníme-li rovnici $y^{-\alpha}$, dostaneme
   \[
      y^{-\alpha}y' + f(x)y^{1-\alpha} = g(x).
   \]
   Položíme-li $z=y^{1-\alpha}$, potom $z'=(1-\alpha)y^{-\alpha}y'$.
   Dosazením do původní rovnice tedy máme
   \[
      \frac{1}{1-\alpha}z' + f(x)z = g(x),
   \]
   což je lineární diferenciální rovnice 1. řádu.
   %
  
\subsection{Lineární diferenciální rovnice 1. řádu}
   Jsou rovnice ve tvaru $y'+f(x)y=g(x)$, kde $f,g$ jsou funkce spojité na $(a,b)$.
   Je-li $g=0$, nazývá se rovnice homogenní, v opačném případě je nehomogenní.
   
   {\bf Homogenní rovnici} řešíme separací proměnných; máme
   \[
      y'+f(x)y = 0.
   \]
   Odtud integrací dostaneme
   \[
      y(x) = Ce^{-\int f(x)\,dx}, \; C\in\mathbb{R}.
   \]

   {\bf Příklad}
   Řešte rovnici $xy'=2y+y'$.

   Rovnici převedeme do ekvivalentního tvaru $(x-1)y'=2y$ (jedná se o rovnici se separovatelnými proměnnými,
   tj. rovnici, kterou lze převést na rovnici se separovanými proměnnými).
   Nechť $y\neq 0$ (nulové řešení je řešením rovnice) a $x\in(1,\infty)$.
   Pak dostaneme rovnici $y'/y=2/(x-1)$. Funkce $1/y$ je spojitá v $(0,\infty)$ (rovněž v $(-\infty,0)$)
   a $1/(x-1)$ je spojitá na uvažovaném intervalu $(1,\infty)$.
   Integrací máme
   \[
      \ln|y| = \ln|x-1|+C, \; C\in\mathbb{R}.
   \]
   Odlogaritmováním dostaneme
   \[
      |y| = C(x-1)^2, \; C>0.
   \]
   Protože uvažujeme $y>0$ a pravá strana je kladná, máme řešení
   \[
      y(x) = C(x-1)^2, \; C>0.
   \] 
   Uvažujeme-li $y<0$, dostaneme řešení
   \[
      y(x) = -C(x-1)^2, \;C>0.
   \]
   Protože nulová funkce je rovněž řešením původní diferenciální rovnice,
   její řešení na intervalu $(1,\infty)$ mají tedy tvar
   \[
      y(x) = C(x-1)^2, \; C\in\mathbb{R}.
   \]
   (Podobně můžeme dostat řešení na intervalu $(-\infty,1)$.)

   Úplná množina řešení jsou tedy paraboly procházející bodem $(x=1,y=0)$, přičemž konstanta $C$ může být
   různá na levé a pravé straně od bodu $x=1$! Tedy při dané počáteční podmínce např. $y(2) = 0$ je řešení
   jednoznačné, $y(x)=0$, pouze pro $x>1$. To přesně odpovídá větě o existenci a jednoznačnosti, jelikož v bodě
   $x=1$ není pravá strana rovnice spojitá v proměnné $x$.

   {\bf Nehomogenní rovnici} řešíme variací konstanty. Uvažujeme ho ve tvaru
   \[
      y(x) = C(x)e^{-\int f(x)\,dx},
   \]
   kde $C$ je funkce parametru $x$.
   Dosazením do původní rovnice máme
   \[
      C'(x) = g(x)e^{\int f(x)\,dx}.
   \]
   Pak tedy
   \[
      y(x)=\left[\int g(x)e^{\int f(x)\,dx}\,+K\right]\,e^{-\int f(x)\,dx}, \;K\in\mathbb{R}
   \]
   je řešením nehomogenní rovnice na intervalu $(a,b)$.
   %

   {\bf Příklad}
   Řešte rovnici $y'-y\,\mathrm{cotg}\,x=e^x\,\sin\,x$.
   
   Jedná se o nehomogenní lineární rovnici prvního řádu.
   Nejprve nalezneme řešení homogenní rovnice $y'-y\,\mathrm{cotg}\,x=0$.
   Funkce $\mathrm{cotg}\,x$ je spojitá na každém intervalu ve tvaru
   $(k\pi,(k+1)\pi)$, kde $k\in\mathrm{Z}$.
   Pro jednoduchost uvažujme $x\in(0,\pi)$.
   Provedeme separaci:
   \[
        y^{-1}\,\d y=\cotg x\,\d x.
   \]
   a integrací (substituce za $\sin x$) dostaneme
   \[
      \ln |y| = \ln |\sin x|+C, \; C\in\mathrm{R}.
   \]
   Odlogaritmováním je
   \[
      |y| = C\sin x, \; C>0
   \]
   Protože pravá strana je kladná ( pro $x\in(0,\pi)$ ) a nulové řešení je rovněž řešením homogenní rovnice,
   máme řešení homogenní rovnice ve tvaru
   \[
      y(x) = C\sin x,\; C\in\mathrm{R}.
   \]
   Řešení nehomogenní rovnice nalezneme variací konstanty, čili položme $C\equiv C(x)$.
   Dosazením řešení ve tvaru $y(x)=C(x)\sin x$ do původní rovnice máme $C'(x) = e^x$,
   odkud $C(x) = e^x+K$, $K\in\Real$.
   Obecné řešení nehomogenní rovnice na intervalu $(0,\pi)$ je tedy
   \[
      y(x) = (e^x+K)\sin x, \; K\in\Real.
   \]

\subsection{Lineární diferenciální rovnice $n$-tého řádu s konstantními koeficienty}
Jedná se o rovnice ve tvaru
   \[
      a_n y^{(n)}+a_{n-1}y^{(n-1)}+\cdots+a_1y'+a_0y = f(x),
   \]
kde $y^{(n)}$ značí $n$-tou derivaci hledané funkce $y$, $f$ je spojitá na nějakém intervalu $(a,b)$ a $a_i$
jsou reálné konstanty ($a_n\neq 0$). Tuto rovnici lze převést na soustavu $n$ lineárních rovnic prvního řádu takto:
\begin{align}
       y_n'(t)&=-\frac{a_{n-1}}{a_n} y_{n}(t) -\frac{a_{n-2}}{a_n} y_{n-1}(t) - \cdots - \frac{a_1}{a_n} y_2(t) - \frac{a_0}{a_n} y_1(t) +f(x)\\
       y_{n-1}'(t)&=y_n'(t)\\
       \dots  \\
       y_1''(t)&=y_2(t) 
\end{align}
přičemž $y_n$ je $n-1$ derivace funkce $y$. Například rovnici
\begin{equation}\label{reseny_n-radu}
%   3x''−5x'−2x = 0
\end{equation}
můžeme převést na soustavu
\begin{align*}
        x_2'&=\frac53x_2+\frac23x_1)\\
        x_1'&=x_2.
\end{align*}
Z teorie pro soustavy lineárních rovnic plyne, že množina všech řešení soustavy $n$ rovnic je vektorový prostor dimenze $n$. 
Stejně tak pro rovnice $n$-tého řádu. Z toho plyne, že pro jednoznačnost řešení je třeba zadat $n$ počátečních (nebo jiných) podmínek.






%, x(0) = 2, x′(0) = −3.
%hx(t) = 3e−t/3
%− e2ti     

   Na tomto intervalu existuje řešení této rovnice;
%   toto je určeno jednoznačně počátečními podmínkami
%   $y(x_0)=y_0, \; y'(x_0)=y_0^{1},\; \ldots, \; y^{(n)}(x_0)=y_0^{n}$
%   pro nějaké $x_0\in(a,b)$. 
%   Rovnice
%   \[
%      a_ny^{(n)}+a_{n-1}y^{(n-1)}+\cdots+a_1y'+a_0y = 0
%   \]
%   se nazývá homogenní (je-li $f(x)\neq 0$ někde v $(a,b)$, nazývá se rovnice nehomogenní).
%   Řešení homogenní rovnice tvoří vektorový podprostor $C(a,b)$
%   a existuje $n$ lineárně nezávislých funkcí (s nenulovým Wronského determinantem), 
%   které jsou jejím řešením (fundamentální systém řešení).
%   Každé řešení lze potom jednoznačně zapsat jako lineární kombinaci funkcí fundamentálního systému.
%   Nechť $\lambda\in\mathbb{C}$ je kořen charakteristické rovnice
%   \[
%       a_n\lambda^n+a_{n-1}\lambda^{n-1}+\cdots+a_1\lambda+a_0 = 0
%   \]
%   s násobností $r$.
%   Je-li $\lambda\in\mathbb{R}$, jsou funkce
%   \[
%      S(\lambda)=\{x^ie^{\lambda x}\,| \; i=0,1,\ldots,r-1\}
%   \]
%   řešením homogenní rovnice.
%   Je-li $\lambda\in\mathbb{C}$ kořen charakteristické rovnice
%   (a tedy i $\bar{\lambda}$ je kořenem), kde $\lambda=\sigma+\mathrm{i}\omega$ s $\sigma,\omega\in\mathbb{R}$, 
%   jsou funkce
%   \[
%      S(\lambda)=\{x^ie^{\sigma x}\cos\omega x,\;x^ie^{\sigma x}\sin\omega x\,|\;i=0,1,\ldots,r-1\}
%   \]
%   rovněž řešením homogenní rovnice (přesněji reálná řešení).
%   Množina
%   \[
%      S = \bigcup\,\{S(\lambda)\,|\,\lambda\text{ je kořen charakteristické rovnice}\}
%   \]
%   tvoří fundamentální systém řešení homogenní rovnice.
%   Řešení nehomogenní rovnice (tj. rovnice s nenulovou pravou stranu) hledáme
%   např. jako v případě řešení lineární rovnice prvního řádu variací konstant. 
%   Stačí nalézt alespoň jedno řešení, neboť řešení nehomogenní rovnice lze psát
%   jako součet libovolného řešení nehomogenní rovnice (partikulární řešení)
%   a funkce, která je lineární kombinací fundamentálního systému řešení homogenní rovnice. 
%   Uvažujme pravou stranu ve speciálním tvaru $p(x)e^{\lambda x}$,
%   kde $p$ je polynom v $x$. Je-li $\lambda$ $r$-násobný kořen charakteristické rovnice ($r\geq 0$)
%   a $p$ je stupně $k$, pak hledáme řešení ve tvaru $t^rP(x)e^{\lambda x}$,
%   kde $P$ je polynom v $x$ rovněž stupně $k$.
%   Má-li pravá strana tvar $p(x)e^{\sigma x}\cos\omega x+q(x)e^{\sigma x}\sin\omega x$,
%   kde $p$ a $q$ jsou polynomy v $x$ stupně nejvýše $k$,
%   a je-li $\sigma+\mathrm{i}\omega$ (a tedy rovněž $\sigma-\mathrm{i}\omega$)
%   $r$-násobný kořen charakteristické rovnice ($r\geq 0$),
%   hledáme řešení ve tvaru $t^rP(x)e^{\sigma x}\cos\omega x+t^rQ(x)e^{\sigma x}\sin\omega x$,
%   kde $P$ a $Q$ jsou polynomy proměnné $x$ nejvýše $k$.
%\end{enumerate}
%begin{ex}
%   Řešte rovnici $y''-y=x^2e^x$ s počátečními podmínkami $y(0)=1$, $y'(0)=0$.
%\end{ex}
%\begin{sol}
%   Kořeny charakteristické rovnice $\lambda^2-1=0$ jsou $\lambda_1=1$, $\lambda_2=-1$,
%   takže fundamentální systém řešení příslušné homogenní rovnice je tvořen funkcemi $e^x$ a $e^{-x}$.
%   Řešení nehomogenní rovnice hledáme ve tvaru $y_p(x)=x(ax^2+bx+c)e^x$, kde $a,b,c\in\mathbb{R}$.
%   Derivováním dostaneme
%   \[
%      (y''_p-y_p)(x) = (6ax^2+(6a+4b)x+2b+2c)e^x
%   \]
%   a porovnáním s pravou stranou máme podmínky $6a=1$, $6a+4b=0$ a $2b+2c=0$, odkud
%   $a=1/6$, $b=-1/4$ a $c=1/4$, takže obecné řešení nehomogenní rovnice má tvar
%   \[
%      y(x) = \left(\frac{1}{6}x^2-\frac{1}{4}x+\frac{1}{4}\right)e^x+c_1e^x+c_2e^{-x}, \; c_1,c_2\in\mathbb{R}.
%   \]
%   Konstanty $c_1$ a $c_2$ určíme z počátečních podmínek.
%   Derivací je
%   \[
%      y'(x) = \left(\frac{1}{6}x^2+\frac{1}{12}x\right)e^x+c_1e^x-c_2e^{-x},
%   \]
%   takže má platit $c_1+c_2=-1/4$ a $c_1-c_2=0$, odkud $c_1=c_2=-1/8$.
%   Řešením tedy je funkce
%   \[
%      y(x) = \left(\frac{1}{6}x^2-\frac{1}{4}x+\frac{1}{8}\right)e^x-\frac{1}{8}e^{-x}.
%   \]
%\end{sol}
%
%\begin{ex}
%   Řešte rovnici $y''-2y'+2y=\cos x$.% s počátečními podmínkami $y(0)=1$, $y'(0)=0$.
%\end{ex}
%\begin{sol}
%   Kořeny charakteristické rovnice $\lambda^2-2\lambda+2=0$ jsou $\lambda_1=1+\mathrm{i}$, $\lambda_2=1-\mathrm{i}$,
%   takže fundamentální systém řešení příslušné homogenní rovnice je tvořen funkcemi 
%   $e^x\cos x$ a $e^x\sin x$.
%   Protože $0+\mathrm{i}$ není kořenem charakteristické rovnice (resp. je $0$-násobným kořenem),
%   hledáme řešení nehomogenní rovnice ve tvaru $y_p(x)=a\cos x+b\sin x$, kde $a,b\in\mathbb{R}$.
%   Derivováním dostaneme
%   \[
%      (y''_p-2y'_p+2y_p)(x) = (a-2b)\cos x+(2a+b)\sin x
%   \]
%   a porovnáním s pravou stranou máme podmínky $a-2b=1$ a $2a+b=0$, odkud
%   $a=1/5$ a $b=-2/5$, takže obecné řešení nehomogenní rovnice má tvar
%   \[
%      y(x) = \left(c_1e^x+\frac{1}{5}\right)\cos x+\left(c_2e^x-\frac{2}{5}\right)\sin x, \; c_1,c_2\in\mathbb{R}.
%   \]
%\end{sol}

\subsection*{Neřešené příklady}
\begin{enumerate}
  \item 
    $ x+yy'=0 $
  \item 
    $ (x^2 +1)(y^2-1)+ xyy'=0,\ y(1)=\sqrt{2},$
  \item
    $ (x+1)y'+xy=0, y(0)=1$
  \item
    $y'+\frac{y}{x}=x$
  \item
    $y'-\frac{2}{x+1}y=(x+1)^3$
  \item
    $xy'-y=x^2 y^{-1}$
\end{enumerate}
reseni:
\begin{enumerate}
  \item
    $y^2+x^2=C$
    pro vsechna $C>0$, je reseni pro $x\in(-\sqrt C,+\sqrt C)$.
  \item
    obecne
    $ 1-y^2=\frac{C}{x^2}e^{-x^2},\ y=+\sqrt{1-\frac{C}{x^2} e^{-x^2}} $
    partikularni reseni $C=-e$,
    $y=\sqrt{1+x^{-x}e^{1-x^2}}$
     pro $x\in(0,\infty)$ a libovolna volba $C$ pro $x\in(-\infty,0)$.
  \item  
    $y=(1+x)e^{-x}$
  \item
    $y=\frac{x^2}{3}+\frac{C}{x}$
    reseni ma dve vetve s nezavislou volbou $C$: $x\in(-\infty,0)$ a $x\in(0,\infty)$.
  \item
    $y=(\frac{x^2}{2}+x+C)(x+1)^2$
    pro $x=-1$ neni splnena rovnice, situace podobna jako v 1)
  \item
    $y=\pm x\sqrt{\ln x^2+C}$
    konstanta $C$ a znamenko lze volit nezavisle na dvou vetvich reseni
    $x\in (-\infty,0)$ a $x\in (0,\infty)$.
\end{enumerate}
