 \chapter{Normované prostory}

 

\section{Vektorové prostory funkcí}
   V definicích metrik $l^1$, $l^2$, $l^\infty$ se uvnitř vždy vyskytuje rozdíl $x-y$ (po složkách). To je ovšem možné jen pokud
   víme jak prvky odčítat. To je možné pokud množina metrického prostoru tvoří {\it vektorový prostor}. 
   \begin{definition}
      Množinu $X$ na níž je definováno sčítání prvků - {\it vektorů} a násobení (reálným) číslem - {\it skalárem} 
      nazveme (reálným) vektorovým prostorem, pokud operace sčítání a násobení splňují následující:
      \begin{description}
        \item $(V1)$ sčítání je komutativní, tj. nezáleží na pořadí : $\vc x+\vc y = \vc y+\vc x$
        \item $(V2)$ sčítání je asociativní, tj. nezáleží na ozávorkování : $(\vc x+\vc y)+\vc z = \vc x+(\vc y+\vc z)$
        \item $(V3)$ existence nulového prvku, jehož přičtení nemění žádný vektor : $\vc x + \vc 0 =\vc x$
        \item $(V4)$ existence inverzního prvku: $\forall \vc x \exists \vc y: \vc y+\vc x=\vc 0$
        \item $(V5)$ distributivita (roznásobování) vektorového sčítání: $a(\vc x + \vc y) = a\vc x +a\vc y$
        \item $(V6)$ distributivita skalárního sčítání: $(a+b)\vc x=a\vc x + b \vc x$
        \item $(V7)$ násobení skalární jedničkou nemění vektor: $1 \vc x = \vc x$
      \end{description}
   \end{definition}
   Jelikož sčítání vektorů a sčítání skalárů jsou princpiálně odlišné operace, je dobré ve značení odlišit vektory a skaláry. 
   Budeme tedy vektory značit tučně: $\vc a,\vc b, \vc \phi$ a skaláry normálně $a,b, \phi$. Tenzory nebo matice budeme značit dvojitým fontem:
   $\tn A, \tn B$.

   Podmnožina vektorového prostoru nemusí být vektorový prostor. Např. množina $M=\{ (1,0)\ ,\ (2,0)\} \subset \Real^2$ není vektorovým
   prostorem jelikož do ní nepatří bod $(3,0)=(1,0)+(2,0)$. Množina $M$ není uzavřená vzhledem k operacím sčítání a násobení. {\it Vektorovým 
   podprostorem} nazýváme podmnožinu $M$, která je uzavřená vzhledem k operacím sčítání a násobení tj. pokud $\vc x$ a $\vc y$ patří do $M$, patří
   tam i $a\vc x+ b\vc y$. Z každé množiny $M\subset X$ můžeme vektorový prostor udělat tím, že k ní přidáme všechny 
   konečné lineární kombinace jejích prvků, píšeme:
   \[
       Lin(M)=\big\{ z\in X\where z=\sum_{i=1}^n a_i \vc x_i,\text{ pro libovolnou $n$-tici $\{x_i\} \subset M$} \big\}.
   \]
   Množina $Lin(M)$ je vektorovým podprostorem prostoru $X$ a říkáme jí {\it lineární obal $M$}.

\subsection{Cvičení}
\begin{enumerate}
 \item Ověřte podle definice vektorového prostoru, že množina všech  funkcí z $\Real^N$ do $\Real^M$ tvoří vektorový prostor s operacemi
  \[
      (f + g)(x) = f(x)+g(x),\quad (af)(x) = af(x).
  \]
 \item Ověřte, že množina polynomů stupně nejvýše $n$, tj. $P^n$ je podprostorem prostoru všech funkcí.
 \item Rozhodněte zda následující množiny jsou vektorovými prostory. Své tvrzení dokažte.
 \begin{enumerate}
        \item $M$ je podmnožina $\Real^2$ vektorů ve tvaru $\vc x=(x,2x)$, tj. $M=\{(x,2x)\where x\in \Real\}$.
                \vysl{je vektorový prostor}
        \item $M$ je podmnožina $\Real^3$ vektorů ve tvaru $\vc x=(x,y,3)$, tj. $M=\{(x,y,3)\where x,\,y\in \Real\}$.
                \vysl{není vektorový prostor}
        \item $M = \{p(\cdot) \in P^2\where p'(x) = x\}$
                \vysl{je vektorový prostor}
        \item $M$ je množina spojitých funkcí na intervalu $(0,1)$, tj. $C(0,1)$.
                \vysl{je vektorový prostor}
        \item $M$ je množina všech funkcí se spojitou derivací na $[0,1]$
                \vysl{je normovaný prostor}
        \item $M$ je množina polynomů, takových, že  $\abs{p(x)}\le 1$ ne intervalu $[-1, 1]$
                \vysl{není vektorový prostor}     
        \item $M$ je množina funkcí, pro které $\int_{-\infty}^{\infty} f =1$. Zálaží na tom jak je definován integrál?
                \vysl{je vektorový prostor}
         
\end{enumerate}
\item Ověřte, že množina diferenciálních operátorů tvoří vektorový prostor. (\dots detaily)
\item Ověřte, že množina distribucí tvoří vektorový prostor. (\dots detaily)
\end{enumerate}


\section{Normované prostory}
   Na vektorových prostorech je možno definovat speciální případ metriky - {\it normu}.
   \begin{definition}
      Nechť $X$ je vektorový prostor a $\norm{\argdot}: X\to \Real$ reálná funkce na $X$ splňující:
      \begin{description}
        \item $(N1)$ norma je nulová jen pro nulový prvek: $\norm{\vc x}=0 \equiv \vc x=\vc 0$
        \item $(N2)$ skalár je možno vytknout v absolutní hodnotě: $\norm{a \vc x} = \abs{a}\norm{\vc x}$
        \item $(N3)$ trojúhelníková nerovnost: $\norm{x+y}\le \norm{\vc x} + \norm{\vc y}$
      \end{description}
   \end{definition}
   Pokud je na vektorovém prostoru $X$ definována norma $\norm{\argdot}$ mluvíme o {\it normovaném lineárním prostoru}. Na takovém prostoru
   nám norma dává metriku: $\rho(\vc x,\vc y) = \norm{\vc x-\vc y}$.
   \begin{exercise}
      Rozmyslete si jak z vlastností normy plynou vlastnosti příslušné metriky. Potřebujete k důkazu symetrie (M1) vlastnost (N1)?
   \end{exercise}
  
   Zde už ale neplatí, že by každá  podmnožina normovaného prostoru $X$ byla opět normovaným prostorem, jelikož podmnožina nemusí být lineárním
   prostorem. Ale platí, že lineární obal $Lin(M)$ každé podmnožiny $M\subset X$ je opět normovaným prostorem se stejnou normou jako na $X$. 
   
   \subsection{Příklady normovaných lineárních prostorů}
   \begin{itemize}
    \item Množina reálných čísel $\Real$ tvoří normovaný prostor, normou je standardně absolutní hodnota $\abs{\cdot}$.
    \item Množina $n$-tic reálných čísel $\Real^n$ tvoří vektorový prostor. Metriky prostorů $l^p_n$, $l^\infty_n$ jsou indukovány normami:
          \[
                \norm{(x_1,x_2,\dots,x_n)}_{l^p_n}=\big(\sum_{i=1}^n \abs{x_i}\big)^{1/p}
          \]
          \[
                \norm{(x_1,x_2,\dots,x_n)}_{l^\infty_n}=\max_{i=1,\dots,n} \abs{x_i}
          \]
    \item Množina $C([0,1])$ spojitých funkcí na uzavženém intervalu je vektorovým prostorem. Analogicky k normám $l^p_n$ můžeme na zde zavést
          normu tak, že sumu nahradíme integrálem:
          \[
                \norm{f}_{L^p([0,1])} = \Big( \int_0^1 \abs{f(x)}^p \d x \Big)^{1/p} 
          \]
          Analogicky k $l^\infty_n$ zavedeme normu
          \[
                \norm{f}_{L^\infty([0,1])} = \sup_{x\in[0,1]} \abs{f(x)}.
          \]
          Zde $\sup$ znamená supremum. To je jistá obdoba maxima. Rozdíl je v tom, že pro nekonečnou množinu např. 
        $\{ 1-1/n\}$ nemusí existovat maximum (= největší prvek množiny), ale supremum (= nejmenší číslo větší než všechny prvky množiny) existuje (je rovno $1$). Zde se hodí ještě jedna poznámka. Jedna ze základních vět
        analýzy říká, že spojitá funkce nabývá na uzavřeném intervalu svého maxima. Tedy v případě funkcí na 
        intervalu $[0,1]$ vždy existuje bod $x_m\in [0,1]$, kde je funkce $\abs{f(x)}$ maximální a tudíž supremum
         je totéž jako maximum. Definice normy pomocí suprema přijde vhod pro obecnější definiční obory, což
        je obsahem dalšího příkladu.
   \item Množina funkcí spojitých na množině $\Omega\subset \Real^m$ tvoří spolu se supremovou (maximovou) normou
          \[
                \norm{f}_{L^\infty(\Omega)} = \sup_{x\in \Omega]} \abs{f(x)}.
          \]
          standardní normovaný prostor $C(\Omega)$.
   \item Množina funkcí spojitě diferencovatelných na množině $\Omega\subset \Real^m$ tvoří s normou
        \[
                \norm{f}_{C^1(\Omega)}=\norm{f}_{C(\Omega)} 
                        + \sum_{i=1}^m \sup_{x\in \Omega} \bigabs{\frac{\prtl f}{\prtl x_i}}
        \]
        standardní prostor $C^1(\Omega)$. Podobně se definuje prostor $n$ krát spojitě diferencovatelných funkcí 
        $C^n(\Omega)$.
        
          - $L^2$, $L^\infty$
          - $H^k$ prostory

   \end{itemize}

\subsection{Cvičení}
 Ověřte, že množina $M$ s ``normou'' $\norm{\cdot}$ tvoří normovaný prostor. K tomu je třeba ověřit,
 že množina je uzavřená na sčítání a násobení skalárem, a ověřit vlastnosti normy. Pokud myslíte, že některá vlastnost 
 není splněna, je třeba uvést protipříklad. Tedy konkrétní prvky množiny, pro které vlastnost neplatí.

\begin{enumerate}
 \item Na vektorovém prostoru $V=\{ ax^2 + bx + b\}$ (rozmyslete se o jaký jde vektorový prostor) uvažujte funkce:
       \begin{align*}
          \norm{ p}_a =  \abs{p'(-1) + p'(1)} \\ 
          \norm{ p}_b = \abs{p'(-1) + p'(1)} + \abs{p''(0)} \\
          \norm{ p}_c = \abs{p'(-1) + p'(1)} + \abs{p''(0)}^{\frac12} \\
          \norm{ p}_d = \big(\abs{p'(-1) + p'(1)}^\frac12 + \abs{p''(0)}^\frac12\big)^2  
       \end{align*}
       Která z nich je normou? Dokažte její vlastnosti. Každá ze zbylých funkcí porušuje právě jednu z vlastností normy. Zjistěte o kterou vlastnost se jedná,
       najděte prvky $V$ pro které vlastnost neplatí a dokažte, že zbylé vlastnosti platí.
        
 \end{enumerate}

\section{Prostory se skalárním součinem}
\begin{itemize}
 \item definice skalárního součinu (R a C)
 \item Schwartzova nerovnost
 \item příklady: $R^N$, $l^2$, $L^2$, $H^k$
 \item ortogonální prvky a množiny
 \item existence nejbližšího prvku, promítání na uzavřené podprostory
 \item ortogonální doplněk, rozklad
 \item ortogonální báze, Gram-Schmidtův ortogonalizační proces
\end{itemize}

\subsection{Cvičení}
\begin{enumerate}
 \item Dokažte, že množina
  \[
     \mathcal E =\{\frac{1}{\sqrt{2\pi}}, \frac{1}{\sqrt{\pi}}\sin nx, \frac{1}{\sqrt{\pi}}\cos nx\where n\in N\}
  \]
  tvoří ortonormální soustavu. Plyne z toho už, že jde o ortonormální bázi?

  \item Nechť $M$ je množina polynomů druhého stupně chápaná jako podprostor $L^2([0,1])$. Najděte ortogonální průmět funkce $e^x$ na $M$

   \item Pokud je množina vektorů $\{e_i\}$ ortonormélní bází v Hilbertova prostoru $X$, pak platí zobecněná Pythagorova věta, tzv. Parsevalova rovnost:
  \begin{equation}\label{parseval}
      \sum_{i=1}^\infty \abs{(e_i, f)}^2 = \norm{f}^2
  \end{equation}
  a to pro libovolný prvek $f\in X$. Rozviňte funkci $f(x)=x$ do Fourierovy řady v prostoru $L^2(-\pi,\pi)$ a použijte 
  \eqref{parseval} pro výpočet součtu 
  \[
     \sum_{i=1}^\infty \frac{1}{n^2} = ?
  \]

  \item Najděte ortonormální bázi prostoru $\Lin\{1,x,x^2\}\subset H^1(0,1)$. Na prostoru $H^1(0,1)$ je definován standardní skalární součin
  \[
     (f,g)= \int_0^1 f'(x)g'(x)+f(x)g(x) \dx. 
  \]
  $\Lin\, M$ značí lineární obal množiny $M$, tedy podprostor všech lineárních kombinací vektorů z množiny $M$.

  \item Uvažujte prostor $H^1(0,1)$ s nestandardním skalárním součinem 
  \[
     (f,g)=\int_0^1 x^2 f'(x) g'(x) +f(x) g(x) \dx
  \]
  ověřte, že se skutečně jedná o skalární součin. A pro funkce $1$ a $e^x$ ověřte Schwartzovu nerovnost.


\end{enumerate}

\section{Operátory na prostorech funkcí}
Operátor $A:X \to Y$ je zobrazení, které zobrazuje prvky (funkce) z normovaného prostoru $X$ do normovaného prostoru $Y$
(tiše předpokládáme, že oba prostory jsou též úplné). Budeme uvažovat jen operátory lineární a spojité
Operátor je {\it lineární} pokud:
\[ 
        A(f+g)=A(f)+A(g), \text{ a } A(cf)=cA(f)
\]
pro libovolné funkce $f,g\in X$ a libovolnou konstantu $c\in \Real$. Operátor je {\it spojitý}, pokud
pro každou posloupnost $f_n$ funkcí konvergující v prostoru $X$ k funkci $f$ , tj.
\[ \norm{f_n - f}_X \to 0 \]
platí, že obrazy těchto funkcí konvergují k obrazu funkce $f$, tj.
\[ \norm{ A(f_n) - A(f) }_Y \to 0 .\]

Příklad: Derivace je operátorem z prostoru $H^1((0,1))$ do $L^2((0,1))$, tedy
\[
        A:H^1((0,1)) \to L^2((0,1)),\quad A(f)=\frac{\prtl f}{\prtl x}.
\]
Tento operátor je lineární, jelikož derivace je lineární, v každém bodě $x\in (0,1)$ platí:
\[
        A(f+g)[x]=\frac{\prtl(f+g)}{\prtl x}[x]=
        \frac{\prtl f}{\prtl x}[x] +\frac{\prtl g}{\prtl x}[x]=A(f)[x]+A(g)[x]  
\]
a
\[
  A(cf)[x]=\frac{\prtl (cf)}{\prtl x}[x]=c\frac{\prtl f}{\prtl x}[x]=cA(f).
\]
A je také spojitý, jelikož pokud 
\[
   \norm{f_n - f}_X^2 = \norm{f_n - f}^2_{H^1}=\int_0^1 (f_n-f)^2 +\int_0^1 (f'_n-f')^2 \to 0
\]
pak musí konvergovat k nule oba integrály jelikož jsou oba nezáporné a tudíž platí
\[
  \norm{ A(f_n) - A(f) }^2_X =\norm{f'_n-f'}^2_{L^2} =\int_0^1 (f'_n-f')^2 \to 0. 
\]

Zatím to vypadá, že operátory jsou zobecněním funkcí, ale jelikož jsou lineární podobají se spíše maticím.
Matice jsou vlastně zápisy lineárních zobrazení mezi konečně rozměrnými vektorovými prostory. Násobení maticí je
pak provedení příslušného operátoru.  Snadno si ověříte, že násobení maticí je vždy spojité zobrazení, tedy že
\[
        x_n,x\in \Real^n:\ x_n \to x\quad \impl \quad \tn M x_n \to \tn M x \text{ v } \Real^m
\]
pro libovolnou matici $\tn M\in \Real^{m,n}$. Matice jsou tedy další důležitý příklad operátorů.
Na nekonečně rozměrných prostorech, jako jsou prostory funkcí, nemusí být každý lineární operátor spojitý!

Pokud operátor $A:X\to Y$ operuje mezi prostory $X$ a $Y$, které jsou podrostory nějakého prostoru $Z$ se
skalárním součinem, definujeme podobně jako u matic

{\it symetrický operátor} pokud:
\[      (Af,g)_Z = (f,Ag)_Z  \quad \text{pro všechna } f,g\in X \]
 
 a {\it pozitivně definitní operátor} pokud existuje $c>0$ takové, že
\[ (Af,f)_Z\ge c (f,f)_Z=\norm{f}_Z^2\quad \text{pro všechna } f\in X. \]

Pro ``přehazování'' derivací funkcí více proměnných lze použít Greenovu větu:

\begin{theorem}
Nechť $f,g$ jsou diferencovatelné funkce na oblasti s hladkou hranicí $\Omega\subset \Real^N$, pak
\[
        \int_\Omega \frac{\prtl f}{x_i} g\d \vc x = \int_{\prtl\Omega} fg \vc n_i \d\sigma
          -\int_{\Omega} f\frac{\prtl g}{x_i}\d \vc x,
\]
kde $\prtl\Omega$ je hranice oblasti $\Omega$ a $\vc n_i$ je $i$-tá složka vektoru $\vc n$, což je tzv. {\it vnější normála} tedy jednotkový vektor kolmý na hranici a směřující ven z oblasti.
Normála je definována pro každý bod hranice, přes kterou integrujeme.
\end{theorem}


\section{Cvičení}
\subsection{Vlastnosti norem}
 

 {\bf Neřešené příklady: }
 \begin{enumerate}
        \item $M$ je podmnožina $\Real^3$ vektorů ve tvaru $\vc x=(x,y,3)$, tj. $M=\{(x,y,3)\where x,\,y\in \Real\}$.
                Za normu $\norm{\vc x}$ berte eukleidovskou normu.
                \vysl{není vektorový prostor}
        \item $M$ je podmnožina $\Real^2$ vektorů ve tvaru $\vc x=(x,2x)$, tj. $M=\{(x,2x)\where x\in \Real\}$.
                 $\norm{(x_1,x_2)}=\abs{x_1}$.
                \vysl{je normovaný prostor}
        \item $M$ je $\Real^2$, $\norm{(x_1,x_2)}=(\sqrt{x_1}+\sqrt{x_2})^2$.
                \vysl{$\norm{\argdot}$ nesplňuje vlastnost (N3)}
        \item $M$ je množina polynomů nejvýše 2. stupně, tj. $M=\{a_1x^2+a_2x +a_3\where a_i\in\Real\}$.
                $\norm{p}=\norm{a_2x^2+a_3x +a_4} = \abs{p(0)}+\abs{p(1)}+\abs{p(2)}$.
                \vysl{je normovaný prostor}
        \item $M$ jako v předchozím příkladu. $\norm{p}=\abs{p'(0)}+\abs{p'(1)}+\abs{p'(2)}$, 
                  kde $p'(0)$ je hodnota derivace polynomu $p$ v bodě $0$.
                \vysl{$\norm{\cdot}$ nesplňuje vlastnost (N1)}
        \item $M$ opět stejná. 
                \[\norm{p}=\big\vert \int_0^1 p(x) \d x \big\vert + \sqrt{p(1)^2+ p(0)^2}\]
                Je třeba použít nerovnost 
                \[ \sqrt{\sum_{i=1}^{n} (a_i+b_i)^2}\le \sqrt{\sum_{i=1}^{n} a_i^2}+\sqrt{\sum_{i=1}^n b_i^2} \]
                \vysl{je normovaný prostor}
        \item $M$ je množina spojitých funkcí se spojitou derivací na $[-1,1]$, tj. $M=C^1([0,1])$. 
                \[\norm{f}=\sqrt{\norm{f}_{L^2([-1,1])}^2+\norm{f'}_{L^2([-1,1])}^2}.\]
                což je norma prostoru $H^1([-1,1])$.
                Použijte trojúhelníkovou nerovnost prostoru $L^2([-1,2])$:
                \[ \sqrt{\int_{-1}^{1} (f(x)+g(x))^2\d x}\le 
                        \sqrt{\int_{-1}^{1} f(x)^2\d x} +\sqrt{\int_{-1}^{1} g(x)^2\d x}\]
                \vysl{je normovaný prostor}     
        \item $M$ je opět $C([0,1])$, ale $\norm{f}=\int_{0}^{1} \abs{f(x)}^2 \d x$
                \vysl{$\norm{\argdot}$ nesplňuje vlastnosti (N2) a (N3)}
\end{enumerate}

\subsection{Výpočet norem}
V následujících úlohách je cílem vypočítat normu/normy funkce. Důležitou kontrolou je, že norma je vždy nezáporná.
{\bf Řešený příklad: } Spočítejte $L^1$, $L^\infty$, a $H^1$ normu pro funkci
        \[
                f(x)=(x+1)(x-2)
        \]
        na intervalu $[-2,3]$.

 {\bf Řešení: }

{\bf Neřešené příklady: }
        \begin{enumerate}
         \item $\norm{f}_{L^1([0,2])}$ a $\norm{f}_{L^\infty([0,1])}$ pro $f(x)=-x(x-1)$.
          \vysl{$1$, $\frac14$ }
         \item $\norm{f}_{L^1([0,1])}$ a $\norm{f}_{L^2([0,1])}$ pro $f(x)=x^{-\frac12}$
           \vysl{$2$, $\infty$ }
         \item $\norm{f}_{L^2([0,2\pi])}$ pro $f(x) =\sin(kx)$ a $f(x)=\cos(kx)$ kde $k$ je libovolné celé číslo.
           \vysl{$\sqrt{\pi}$} 
         \item $\norm{f}_{L^1([-\infty,\infty]}$ a $\norm{f}_{L^\infty([-\infty,\infty]}$  pro $f(x)=\frac{1}{1+x^2}$
           \vysl{$\Pi$, $1$ }
         \item normu v prostoru $H^1([-1,1)$ pro funkce $\sqrt[3]{x}$ a $\sqrt[3]{x^2}$, pro které hodnoty parametru
                $p$ bude mít funkce $x^p$ konečnou normu?
            \vysl{$\infty$, $\sqrt{(\frac67)^2+(6)^2}$, $p>\frac12$}
        \end{enumerate}

\subsection{Konvergence v prostorech funkcí}
V následujících příkladech vypočtěte limitu posloupnosti funkcí v daném prostoru (dané normě):
\begin{enumerate}
        \item $f_n(x)= x^2+ \frac{x}{n}$ v normě $L^2([0,2])$
                \vysl{$x^2$}
    \item stejná posloupnost v normě $L^\infty([0,2])$
        \vysl{$x^2$}
    \item $f_n(x)= x^{\frac{1}{2n+1}}$ v normě $L^1([-1,1])$
        \vysl{$sgn(x)$}
    \item stejná posloupnost v normě $L^{\infty}([-1,1])$
        \vysl{nekonverguje}
    \item $f_n(x)= \cos(\frac{x}{n}) e^x$ v normě $L^1([-1,1])$
        \vysl{$e^x$}
    \item \[f_n(x)=
                \left\{\begin{aligned}
                        n^{1/2} \text{ na $[0,\frac1n]$}\\
                        0 \text{ na $(\frac1n,1]$}
                       \end{aligned}\right.
               \]
               v normách $L^1{(0,1)}$ a $L^2{(0,1)}$.
                \vysl{$0$, nekonverguje}
    \item Konverguje posloupnost $f_n(x)=\sin(nx)$ na intervalu $[-\pi,\pi]$ v nějaké $L^p$ - normě?
         Pokuste se vysvětlit proč.
         \vysl{v žádné normě; Existuje bodová limita?}
\end{enumerate}

\subsection{Vlastnosti množin v metrických prostorech}
V následujících úlohách máte uvést příklady množin splňující požadované vlastnosti. Na vašem řešení je
v tomto případě nejdůležitější argumentace, ža vámi uvedená množina skutečně splňuje dané vlastnosti.
Napište definice použitých pojmů.
\begin{enumerate}
 \item Najděte všechny hromadné a izolované body množin:
        \begin{enumerate}
                \item $A={-1}\cup(1,\infty)$ v prostoru $\Real$
                \vysl{$A$, \{-1\}}
                \item $B=\{x\in \Qnum\where x\in[0,1]\}\times\{y=1/n\where n\in\Nnum\}$ v prostoru $\Real^2$
                \vysl{$\{x\in \Real\}\times\{y=1/n \vee y=0\}$, $\emptyset$}
                \item $C=\{f_n(x)\where n\in \Nnum\}$,
                
                        \[ f_n(x)=\left\{\begin{aligned}
                                                \sin(nx), &&x\in(0,\frac x n)\\
                                                0,                &&x\in[\frac x n,1)                   
                                         \end{aligned}\right.\]
                         v normě $L^1$.
                \vysl{$f(x)=0$, $C$}     
                \item $C$ v normě $L^\infty$
                \vysl{$\emptyset$, $C$}
        \end{enumerate}
 \item Najděte množinu $M$ (v nějakém metrickém prostoru):
        \begin{enumerate}
                \item s prázdným vnitřkem a nekonečným množstvím hromadných bodů.
                \item s prázdným vnitřkem a nekonečným množstvím hromadných bodů, z nichž některé
                nepatří do $M$
                \item s prázdným vnitřkem a nekonečným množstvím hromadných bodů, z nichž nekonečno
                nepatří do $M$
                \item * s prázdným vnitřkem a nekonečným množstvím hromadných bodů, z nichž žádný
                nepatří do $M$
        \end{enumerate}
        \vysl{kružnice v $\Real$, kružnice s dírami, viz. předchozí příklad, za pivo}
        \item Existuje množina se dvěma izolovanými body a prázdnou hranicí? Pečlivě zdůvodněte.
\item Uveďte příklad spočetné množiny funkcí husté v prostoru $C((0,\pi))$.
                \vysl{ $ \{ \sum_{n=1}^\infty r_n \sin(nx)\where r_n\in \Qnum \}$ tedy něco jako lineární obal množiny
                $\{\sin(nx)\}$ ale jen s použitím racionálních čísel.}
 \item Ve které normě je prostor $C((0,1))$ úplný, a ve které je neúplný?
           \vysl{neuplný v $L^1$ - proč?, úplný v \dots to už je příliš lehké}
 \item Uveďte příklad metrického prostoru (množiny) $M$ funkcí na intervalu $(0,1)$, který je úplný v
       $L^\infty$ normě. Uveďte příklad prostoru $N$ funkcí na stejném intervalu, který je v
       $L^\infty$ normě naopak neúplný a je podmnožinou $M$.
       \vysl{ např. $N=\{f(x)=1/n\where n\in \Nnum\}$, $M=N\cup \{ f(x)=0 \} $}
  \item Najděte dva normované prostory $M$, $N$ splňující podmínky předchozího příkladu, ale s $L^1$
                normou. Tedy $N\subset M$, $M$ úplný a $N$ neúplný prostor funkcí v $L^1$-normě
                na intevalu $(0,1)$.
                \vysl{ např. $M=L^1((0,1))$, $N=C((0,1))$ }
\end{enumerate}

\subsection{Operátory}
V následujícím ověřte, že operátor $A:X\to Y$ je lineární, symetrický a pozitivně definitní na daném prostoru
se skalárním součinem $Z$ (předpokládáme $X, Y \subset Z$. Použijte integraci per partes, nebo Greenovy větu. V některých případech nelze pomocí Greenovy věty dokázat přímo pozitivní definitu, tedy
\[
  (Af,f)_X \ge c(f,f),
\]
ale pouze
\[
  (Af,f)_X \ge c(f',f')_X
\]
tehdy je nutno použít dodatečné nerovnosti
\begin{equation}\label{poincare}
  (f',f')_X \ge c(f,f)_X,
\end{equation}
která pro některé prostory funkcí platí. V takovém případě je to explicitně uvedeno v zadání.

\begin{enumerate}
\item 
\[
A:C^2((0,1))\to L^2((0,1)),\quad A(f)=\big(-\frac{\prtl^2 f}{\prtl x^2} +f \big),
\]
\[ Z=L^2((0,1)) \]
Použijte per partes.
\item 
\[
A:C^2(\Omega)\to L^2(\Omega),\quad A(f)=\big(-2\frac{\prtl^2 f}{\prtl x^2} -3 \frac{\prtl^2 f}{\prtl y^2} \big),
\]
\[ Z=L^2(\Omega),\quad \Omega=[0,1]\times[0,1] \]
Požijte per partes v obou osách a také nerovnost \eqref{poincare}. 
\item
\[
A:C^2((1,2))\to L^2((1,2)),\quad A(f)=-\frac{\prtl }{\prtl x}\big( x^2 \frac{\prtl f}{\prtl x} \big)+ f,
\]
\[ Z=L^2((1,2))\]
Použijte nerovnost \eqref{poincare}.
\item 
\[
A:C^2(\Omega)\to L^2(\Omega),\quad A(f)=\big(-\frac{\prtl^2 f}{\prtl x^2} -\frac{\prtl^2 f}{\prtl y^2} + f \big),
\]
\[ Z=L^2(\Omega),\quad \Omega= \text{  koule v }\Real^2\]
Požijte Greenovu větu.


\end{enumerate}



\begin{verbatim}
http://en.wikipedia.org/wiki/Metric_space
http://en.wikipedia.org/wiki/Vector_space
http://en.wikipedia.org/wiki/Norm_(mathematics)
http://en.wikipedia.org/wiki/Normed_vector_space
http://en.wikipedia.org/wiki/Closed_set
http://en.wikipedia.org/wiki/Open_set
http://en.wikipedia.org/wiki/Boundary_(topology)
http://en.wikipedia.org/wiki/Limit_point
\end{verbatim}


\begin{verbatim}  
Author: Atkinson, Kendall E.
Title: An introduction to numerical analysis
Edition: 2nd ed.
Publisher/year: [New York] : John Wiley & Sons, 1989
Physical descr. : xvi, 693 s. : il.
Signatures: A 58130
---------------------------------------------------------------
Author: Tyrtyshnikov, Jevgenij Jevgeněvič
Title: <>brief introduction to numerical analysis
Publisher/year: Boston : Birkhäuser, 1997
Physical descr.: xii, 202 s. : il.
Signatures: A 59461
---------------------------------------------------------------
Author: Novák, J.
Title: General topology and its relations to modern analysis and algebra
Publisher/year: Praha : Československá akademie věd, 1962
Physical descr. : 363 s.
Signatures: A 10836
---------------------------------------------------------------
Author: Stoer, Josef; Bulirsch, Roland; Bartels, R.; Gautschi, W.; Witzgall, C.; Marsden, Jerrold E.
Title: Introduction to numerical analysis. [series editor J. E. Marsden ... et al.]
Edition: 3rd ed.
Publisher/year: New York : Springer, 2002
Physical descr. : xv, 744 s. : il., graf.
Signatures: A 56811
\end{verbatim} 