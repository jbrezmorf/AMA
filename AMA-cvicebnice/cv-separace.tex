\documentclass[a4paper,10pt]{book}
%\usepackage[active]{srcltx}
\usepackage[czech]{babel}
\usepackage[utf8]{inputenc}

\usepackage{amsmath}
\usepackage{amsfonts}
\usepackage{amssymb}
\usepackage{amsthm}
%
\newtheorem{theorem}{Věta}[section]
\newtheorem{proposition}[theorem]{Tvrzení}
\newtheorem{definition}[theorem]{Definice}
\newtheorem{remark}[theorem]{Poznámka}
\newtheorem{lemma}[theorem]{Lemma}
\newtheorem{corollary}[theorem]{Důsledek}
\newtheorem{exercise}[theorem]{Cvičení}

%\numberwithin{equation}{document}
%
\def\div{{\rm div}}
\def\Lapl{\Delta}
\def\grad{\nabla}
\def\supp{{\rm supp}}
\def\dist{{\rm dist}}
%\def\chset{\mathbbm{1}}
\def\chset{1}
%
\def\Tr{{\rm Tr}}
\def\to{\rightarrow}
\def\weakto{\rightharpoonup}
\def\imbed{\hookrightarrow}
\def\cimbed{\subset\subset}
\def\range{{\mathcal R}}
\def\leprox{\lesssim}
\def\argdot{{\hspace{0.18em}\cdot\hspace{0.18em}}}
\def\Distr{{\mathcal D}}
\def\calK{{\mathcal K}}
\def\FromTo{|\rightarrow}
\def\convol{\star}
\def\impl{\Rightarrow}
\DeclareMathOperator*{\esslim}{esslim}
\DeclareMathOperator*{\esssup}{ess\,supp}
\DeclareMathOperator{\ess}{ess}
\DeclareMathOperator{\osc}{osc}
\DeclareMathOperator{\curl}{curl}
\DeclareMathOperator{\cotg}{cotg}

%
%\def\Ess{{\rm ess}}
%\def\Exp{{\rm exp}}
%\def\Implies{\Longrightarrow}
%\def\Equiv{\Longleftrightarrow}
% ****************************************** GENERAL MATH NOTATION
\def\Real{{\rm\bf R}}
\def\Rd{{{\rm\bf R}^{\rm 3}}}
\def\RN{{{\rm\bf R}^N}}
\def\D{{\mathbb D}}
\def\Nnum{{\rm\bf N}}
\def\Qnum{{\rm\bf Q}}
\def\Measures{{\mathcal M}}
\def\d{\,{\rm d}}               % differential
\def\sdodt{\genfrac{}{}{}{1}{\rm d}{{\rm d}t}}
\def\dodt{\genfrac{}{}{}{}{\rm d}{{\rm d}t}}
%
\def\vc#1{\mathbf{\boldsymbol{#1}}}     % vector
\def\tn#1{{\mathbb{#1}}}    % tensor
\def\abs#1{\lvert#1\rvert}
\def\Abs#1{\bigl\lvert#1\bigr\rvert}
\def\bigabs#1{\bigl\lvert#1\bigr\rvert}
\def\Bigabs#1{\Big\lvert#1\Big\rvert}
\def\ABS#1{\left\lvert#1\right\rvert}
\def\norm#1{\bigl\Vert#1\bigr\Vert} %norm
\def\close#1{\overline{#1}}
\def\inter#1{#1^\circ}
\def\eqdef{\mathrel{\mathop:}=}     % defining equivalence
\def\where{\,|\,}                    % "where" separator in set's defs
\def\timeD#1{\dot{\overline{{#1}}}}
%
% ******************************************* USEFULL MACROS
\def\RomanEnum{\renewcommand{\labelenumi}{\rm (\roman{enumi})}}   % enumerate by roman numbers
\def\rf#1{(\ref{#1})}                                             % ref. shortcut
\def\prtl{\partial}                                        % partial deriv.
\def\Names#1{{\scshape #1}}
\def\rem#1{{\parskip=0cm\par!! {\sl\small #1} !!}}
\def\vysl#1{\par$[$ #1 $]$}

%
%
% ******************************************* DOCUMENT NOTATIONS
% document specific
%***************************************************************************
%
\addtolength{\textwidth}{2cm}
\addtolength{\vsize}{2cm}
\addtolength{\topmargin}{-1cm}
\addtolength{\hoffset}{-1cm}
\begin{document}
\parskip=2ex
\parindent=0pt
\pagestyle{empty}

\chapter{Soustavy obyčejných diferenciálních rovnic}
\section{Obyčejné diferenciální rovnice}
obyčejná diferenciální rovnice -- vztah mezi neznámou funkcí jedné proměnné a jejími derivacemi;
řád ODR -- řád nejvyšší derivace v ODR.

ODR $n$-tého řádu: $F(x,y,y',\ldots,y^{(n)})=0$;
řešení (integrál) ODR -- funkce, která vyhovuje rovnici v daném oboru.

ODR 1. řádu: $F(x,y,y')=0$, resp. $y'=f(x,y)$ (rozřešená vzhledem k derivaci).

\subsection{Rovnice se separovanými proměnnými}
   Rovnici ve tvaru ve tvaru $f(x)+g(y)y'=0$, kde $f,g$ jsou funkce nazýváme
   rovnicí se separovanými proměnými.
   Je-li $f$ spojitá na $(a,b)$ a $g$ je spojitá na $(c,d)$,
   potom každé řešení na $I\subset(a,b)$ splňuje na $I$ rovnici
   \[
      \int f(x)\,dx + \int g(y)\,dy = C, \; C\in\mathbb{R}.
   \]
   Funkce určená na intervalu $J\subset(c,d)$ touto rovnicí, je řešením diferenciální rovnice na $J$.

   {\bf Příklad}
   Řešte rovnici $yy'=e^y$.

   Prvně si uvědomíme, že ve standardním tvaru je pravá strana rovnice $e^y/y$ což je lipschitcovská funkce
   proměnné $y$ všude kromě bodu $y=0$.
   Derivaci napíšeme jako symbolický zlomek diferencí:
   \[
    	y\frac{\d y}{\d x}=e^y
   \]
   a úpravami převedeme na jednu stranu všechny členy  obsahující $y$ a na druhou čelny obsahující $x$:
   \[
    	ye^{-y}\,\d y=1\,\d x
   \]
   nyní integrujeme (per partes) a dostáváme
   \[
      -(1+y)e^{-y}+C=x
   \]
   Což je předpis pro řešení zapsané jako funkce $x(y)$. V souladu s větou o existenci a jednoznačnosti
   zjistíme, že pro $y=0$ není rovnice splněna. Ovšem na intervalech $y\in(-\infty,0)$ a  $y\in(0,\infty)$
   existuje jednoznačné řešení. Na každém z nich můžeme volit jinou konstantu $C$. Pokud bude dána počáteční
   podmínka např. $x(1)=0$, bude řešení jednoznačné jen na intervalu $y\in(0,\infty)$. Nicméně, když si nakreslíme graf,
   zjistíme, že na intervalu $y\in(-\infty,0)$ existuje právě jedna volba $C$, tak aby na sebe obě části navazovaly
   v bodě nula. Tomu říkáme, že existuje jednoznačné spojité prodloužení.
  
\subsection{Lineární diferenciální rovnice 1. řádu}
   Jsou rovnice ve tvaru $y'+f(x)y=g(x)$, kde $f,g$ jsou funkce spojité na $(a,b)$.
   Je-li $g=0$, nazývá se rovnice homogenní, v opačném případě je nehomogenní.
   
   {\bf Homogenní rovnici} řešíme separací proměnných; máme
   \[
      y'+f(x)y = 0.
   \]
   Odtud integrací dostaneme
   \[
      y(x) = Ce^{-\int f(x)\,dx}, \; C\in\mathbb{R}.
   \]

   {\bf Příklad}
   Řešte rovnici $xy'=2y+y'$.

   Rovnici převedeme do ekvivalentního tvaru $(x-1)y'=2y$ (jedná se o rovnici se separovatelnými proměnnými,
   tj. rovnici, kterou lze převést na rovnici se separovanými proměnnými).
   Nechť $y\neq 0$ (nulové řešení je řešením rovnice) a $x\in(1,\infty)$.
   Pak dostaneme rovnici $y'/y=2/(x-1)$. Funkce $1/y$ je spojitá v $(0,\infty)$ (rovněž v $(-\infty,0)$)
   a $1/(x-1)$ je spojitá na uvažovaném intervalu $(1,\infty)$.
   Integrací máme
   \[
      \ln|y| = \ln|x-1|+C, \; C\in\mathbb{R}.
   \]
   Odlogaritmováním dostaneme
   \[
      |y| = C(x-1)^2, \; C>0.
   \]
   Protože uvažujeme $y>0$ a pravá strana je kladná, máme řešení
   \[
      y(x) = C(x-1)^2, \; C>0.
   \] 
   Uvažujeme-li $y<0$, dostaneme řešení
   \[
      y(x) = -C(x-1)^2, \;C>0.
   \]
   Protože nulová funkce je rovněž řešením původní diferenciální rovnice,
   její řešení na intervalu $(1,\infty)$ mají tedy tvar
   \[
      y(x) = C(x-1)^2, \; C\in\mathbb{R}.
   \]
   (Podobně můžeme dostat řešení na intervalu $(-\infty,1)$.)

   Úplná množina řešení jsou tedy paraboly procházející bodem $(x=1,y=0)$, přičemž konstanta $C$ může být
   různá na levé a pravé straně od bodu $x=1$! Tedy při dané počáteční podmínce např. $y(2) = 0$ je řešení
   jednoznačné, $y(x)=0$, pouze pro $x>1$. To přesně odpovídá větě o existenci a jednoznačnosti, jelikož v bodě
   $x=1$ není pravá strana rovnice spojitá v proměnné $x$.

   {\bf Nehomogenní rovnici} řešíme variací konstanty. Uvažujeme ho ve tvaru
   \[
      y(x) = C(x)e^{-\int f(x)\,dx},
   \]
   kde $C$ je funkce parametru $x$.
   Dosazením do původní rovnice máme
   \[
      C'(x) = g(x)e^{\int f(x)\,dx}.
   \]
   Pak tedy
   \[
      y(x)=\left[\int g(x)e^{\int f(x)\,dx}\,+K\right]\,e^{-\int f(x)\,dx}, \;K\in\mathbb{R}
   \]
   je řešením nehomogenní rovnice na intervalu $(a,b)$.
   %

   {\bf Příklad}
   Řešte rovnici $y'-y\,\mathrm{cotg}\,x=e^x\,\sin\,x$.
   
   Jedná se o nehomogenní lineární rovnici prvního řádu.
   Nejprve nalezneme řešení homogenní rovnice $y'-y\,\mathrm{cotg}\,x=0$.
   Funkce $\mathrm{cotg}\,x$ je spojitá na každém intervalu ve tvaru
   $(k\pi,(k+1)\pi)$, kde $k\in\mathrm{Z}$.
   Pro jednoduchost uvažujme $x\in(0,\pi)$.
   Provedeme separaci:
   \[
    	y^{-1}\,\d y=\cotg x\,\d x.
   \]
   a integrací (substituce za $\sin x$) dostaneme
   \[
      \ln |y| = \ln |\sin x|+C, \; C\in\mathrm{R}.
   \]
   Odlogaritmováním je
   \[
      |y| = C\sin x, \; C>0
   \]
   Protože pravá strana je kladná ( pro $x\in(0,\pi)$ ) a nulové řešení je rovněž řešením homogenní rovnice,
   máme řešení homogenní rovnice ve tvaru
   \[
      y(x) = C\sin x,\; C\in\mathrm{R}.
   \]
   Řešení nehomogenní rovnice nalezneme variací konstanty, čili položme $C\equiv C(x)$.
   Dosazením řešení ve tvaru $y(x)=C(x)\sin x$ do původní rovnice máme $C'(x) = e^x$,
   odkud $C(x) = e^x+K$, $K\in\Real$.
   Obecné řešení nehomogenní rovnice na intervalu $(0,\pi)$ je tedy
   \[
      y(x) = (e^x+K)\sin x, \; K\in\Real.
   \]


\subsection*{Neřešené příklady}
\begin{enumerate}
  \item 
    $ x+yy'=0 $
  \item 
    $ (x^2 +1)(y^2-1)+ xyy'=0,\ y(1)=\sqrt{2},$
  \item
    $ (x+1)y'+xy=0, y(0)=1$
  \item
    $y'+\frac{y}{x}=x$
  \item
    $y'-\frac{2}{x+1}y=(x+1)^3$
  \item
    $xy'-y=x^2 y^{-1}$
\end{enumerate}
reseni:
\begin{enumerate}
  \item
    $y^2+x^2=C$
    pro vsechna $C>0$, je reseni pro $x\in(-\sqrt C,+\sqrt C)$.
  \item
    obecne
    $ 1-y^2=\frac{C}{x^2}e^{-x^2},\ y=+\sqrt{1-\frac{C}{x^2} e^{-x^2}} $
    partikularni reseni $C=-e$,
    $y=\sqrt{1+x^{-x}e^{1-x^2}}$
     pro $x\in(0,\infty)$ a libovolna volba $C$ pro $x\in(-\infty,0)$.
  \item  
    $y=(1+x)e^{-x}$
  \item
    $y=\frac{x^2}{3}+\frac{C}{x}$
    reseni ma dve vetve s nezavislou volbou $C$: $x\in(-\infty,0)$ a $x\in(0,\infty)$.
  \item
    $y=(\frac{x^2}{2}+x+C)(x+1)^2$
    pro $x=-1$ neni splnena rovnice, situace podobna jako v 1)
  \item
    $y=\pm x\sqrt{\ln x^2+C}$
    konstanta $C$ a znamenko lze volit nezavisle na dvou vetvich reseni
    $x\in (-\infty,0)$ a $x\in (0,\infty)$.
\end{enumerate}

\end{document}
