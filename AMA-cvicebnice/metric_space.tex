\chapter{Metrický prostor}

\section{Metrický prostor}
  Každý z nás už asi zažil, že podle mapy měl být cíl vaší cesty 5km, ale v terénu to vydalo třeba 8km. Nebo vzdálenost 
  dvou měst vzdušnou čarou je vždy o dost nižšší než ujeté kilometry na tachmetru. Vidíme tedy, že kromě vzdálenosti vzdušnou čarou,
  která odpovídá Eukleidovské vzdálenosti bodů v prostoru, můžeme zavést i různé jiné vzdálenosti podle druhu dopravního prostředku a 
  dalších okolností. Abstrakcí vzdálenosti je tzv. metrika. Musíme však mít množinu objektů jejichž vzdálenost budeme měřit. 
  Metrický prostor je pak tato množina spolu se vzdáleností pro libovolné dva prvky. To je obsah následující definice.

  \begin{definition}
   Metrický prostor je dvojice  $(X, \rho)$, kde $X$ je množina a $\rho$ je reálná funkce na dvojicích prvků z $X$, tj.
   $ \rho: X\times X \to \Real$ splňující následující vlastnosti.
   \begin{description}
        \item $(M1)$ $\rho$ je symetrická: $\rho(x,y) = \rho(y,x)$
        \item $(M2)$ $\rho$ je nulová právě tehdy, když $x=y$: $\rho(x,y) = 0 \equiv x=y$.
        \item $(M3)$ $\rho$ splňuje {\it trojúhelníkovou nerovnost}: $\rho(x,y)\le\rho(x,z)+\rho(z,y)$.
   \end{description}
   Takovou funkci $\rho$ nazýváme {\it metrikou}, {\it vzdálenostní funkcí}, nebo jen {\it vzdáleností}.
  \end{definition}
  \begin{exercise}
  Dokažte, že z vlastností $(M1) - (M3)$ plyne, že $\rho$ je nezáporná funkce. Začněte 
  \[2\rho(x,y) = \rho(x,y)+\rho(x,y)= \dots\]
  a postupně použijte vlastnosti $(M1)$, $(M3)$, $(M2)$.
  \end{exercise}

  Často se metrický prostor označujeme stejným symbolem jako množinu, na které je definovaný. To je ovšem možné pouze pokud
  jsme se dohodli, jakou metriku na te množině budeme používat.

  \begin{remark} \label{metricky_podprostor}
        Jestliže $(X,\rho)$ je metrický prostor a $Y$ je podmnožina $X$. Pak dvojice $(Y,\rho)$ je také metrickým prostorem.
        Říkáme, že $Y$ je metrický {\it podprostor} prostoru $X$.
  \end{remark}
  
  \subsection{Příklady metrických prostorů:}
  \begin{enumerate}
    \item Prostor $\Real$. Množina reálných čísel spolu s metrikou danou absolutní hodnotou: $\rho(x,y)=\abs{x-y}$ je metrickým prostorem.
        Podle poznámky \ref{metricky_podprostor} je metrickým prostorem i libovolná podmnožina. Speciálně množina racionálních čísel 
        tvoří podprostor $(\tn Q,\abs{x-y})$.
    \item Prostory $\Real^n$. Množina $n$-tic reálných čísel $\Real^n$ s {\it Eukleidovskou metrikou}
  \[
     l_n^2(x,y) = \sqrt{\sum_{i=1}^{n} (x_i-y_i)^2} 
  \]
  tvoří standardní prostor $\Real^n$ nebo též $l_n^2$. V případě $\Real^2$ si můžeme představit mapu a 
  eukleidovská metrika odpovídá vzdélenosti vzdušnou čarou. 
  \item Prostory $l^p_n$. Opět uvažujeme množinu $\Real^n$, ale v metrice použijeme místo exponentu $2$ libovolný exponent $p$ z intervalu $[0,\infty)$.
  Dostaneme prostor $l_n^p$ s metrikou
  \[
        l_n^p(x,y) = \big(\sum_{i=1}^{n} \abs{x_i-y_i}^p\big)^{1/p}. 
  \]
  Speciálně pro $p=1$ máme "sčítací" metriku
  \[
     \rho(x,y) = \sum_{i=1}^{n} \abs{x_i-y_i} 
  \]
  a prostor $l_n^1$. Na množině $\Real^2$ tato metrika odpovídá "pouliční" vzdálenosti ve městě s pravoúhlou soustavou ulic.
  \item Prostor $l_n^p$. Pro "$p\to \infty$" dostáváme tzv. {\it maximovou metriku}
  \[
     l_n^\infty(x,y) = \max_{i=1..n} \abs{x_i-y_i}.
  \]
  a příslušný prostor $l^\infty_n$ na množině $\Real^n$. 
  \item Prostor $C([0,1])$, $C(I)$. Uvažujeme množinu spojitých funkcí na uzavřeném intervalu $[0,1]$ a analogii $l_n^\infty$ metriky:
        \[
          L^\infty(f,g)=\max_{x\in[0,1]} \abs{f(x)-g(x)}
        \]
        Zde ovšem musíme dát pozor. Děláme maximum přes nekonečnou množinu a toto maximum nemusí existovat (např. maximum množiny $(0,1)$ není 
        číslo $1$ jelikož do této množiny nepatří). Nicméně z analýzy víme, že spojitá funkce na uzavřeném intervalu nabývá maxima. Takže naše definice
        metriky je v pořádku. Zápis $C([0,1])$ označuje jak množinu spojitých funkcí na intervalu $[0,1]$ tak právě definovaný prostor s $L^\infty$
        metrikou. Podobně můžeme definovat prostor $C(I)$ pro libovolný uzavřený interval $I$.
  \item Prostory $( C(I), L^p )$. Podobně jako v předchozím příkladě můžeme zobecnit i metriky $l_n^p$. Na množině $C(I)$ zavedeme metriku
        \[
          L^p(f,g)=\big(\int_I \abs{f(x)-g(x)}^p\d x\big)^{1/p}. 
        \]
  \item Prostory s normou. Doposud uvedené příklady metrických prostorů jsou odvozeny od prostorů s normou, které zavedeme dále.
        To platí obecně: Každý prostor s normou je příkladem metrického prostoru.
  \item Diskrétní metrika. Na libovloné množině $X$ můžeme zavést tzv. {\it diskrétní metriku}
  \[
     \delta(x,y)=\left\{
     \begin{aligned}
        0&  \text{ pro }x= y\\
        1&  \text{ pro }x\ne y 
     \end{aligned}
     \right.
  \]
  tuto metriku (narozdíl od předchozích případů) lze použít na libovolné množině bez jakékoliv struktury. 
  \item Metrikcké prostory se neomezují jen na $n$-tice čísel, body v prostoru a funkce. V praxi se používá
   např. {\it Levenshteinova metrika} pro měření "vzdálenosti" znakových řetězců, {\it Hausdorfova metrika} pro 
   vzdálenost množin, speciální metriky pro porovnávání tvaru dvou- a troj- rozměrných objektů bez ohledu na jejich natočení a umístění v prostoru,
  a mnoho dalších aplikací.
\end{enumerate}

\subsection{Cvičení}

viz. zdroje exercise1.pdf

\section{Množiny v metrických prostorech}
Pojem metrického prostoru umožňuje zobecnit pojem otevřeného a uzavřeného intervalu a dále pojmy známé z analýzy
jako spojitost, konvergence, úplnost. Nejprve si zopakujme základní množinové operace.
Nechť $M,N\subset X$ jsou podmnožiny metrického prostoru $X$. Definujeme
\begin{itemize}
   \item sjednocení: $M\cup N\equiv\{x\in X\,|\,x\in M\;\vee\;x\in N\}$;
   \item průnik: $M\cap N\equiv\{x\in X\,|\,x\in M\;\&\;x\in N\}$;
   \item rozdíl: $M\setminus N\equiv\{x\in X\,|\,x\in M\;\&\;x\not\in N\}$;
   \item doplněk: $M^c\equiv\{x\in X\,|\,x\not\in M\}$;
\end{itemize}
zřejmě platí $M\setminus N=M\cap N^c$ a $M^c = X\setminus M$.

V metrickém prostoru $(X,\rho)$ definujeme kouli $B_r(x)$ v bodě $x\in X$ o poloměru $r\in\Real$ jako množinu bodů, které mají od
$x$ vzdálenost menší než $r$:
\[
        B_r(x)=\{ y\in X\where \rho(x,y)< r\}.
\]
Jak vypadají koule v prostorech $l^p_2$? Jak vypadají koule v prostoru s diskrétní metrikou, např. $(\tn Q, \delta)$?

Dále zavedeme pojem {\it okolí}. Okolí bodu $x$ ($\mathcal O(x)$)je libovolné množina, která obsahuje nějakou (malou) kouli se středem v bodě $x$. {\it Prstencové okolí}
bodu $x$ ($\mathcal P(x)$) je jeho okolí bez bodu $x$ samotného.

Nechť $M\subset X$ je podmnožina metrického prostoru $X$ a $x\in X$.
Říkáme, že $x$ je
\begin{itemize}
   \item {\it vnitřním bodem} množiny $M$, pokud exituje okolí $\mathcal O(x)$, které je podmnožinou $M$:
        \[
          \exists r>0\,:\ B_r(x)\subset M
        \]
   \item {\it vnějším bodem} množiny $M$, pokud exituje okolí $\mathcal O(x)$, které je má s $M$ prázdný průnik:
        \[
          \exists r>0\,:\ B_r(x)\cap M=\emptyset
        \]
   \item {\it hraničním bodem} množiny $M$, pokud každé okolí $\mathcal O(x)$ má neprázdný průnik jak s množinou $M$ tak s jejím doplňkem $M^c$:
        \[
          \forall r>0\,:\ B_r(x)\cap M\neq\emptyset \;\&\; B_r(x)\cap M^c\neq\emptyset
        \]
   \item {\it hromadným bodem} množiny $M$, pokud každé prstencové okolí $\mathcal P(x)$ má s množinou $M$ neprázdný průnik.
        \[         
          \forall r>0\,:\ (B_r(x)\setminus\{x\})\cap M\neq\emptyset
        \]
   \item {\it izolovaným bodem} množiny $M$, pokud existuje prstencové okolí $\mathcal P(x)$, které má s $M$ prázdný průnik.
        \[
         \exists r>0\,:\ B_r(x)\cap M=\{x\}
        \]
\end{itemize}

Pro množinu $M$ pak definujeme:
\begin{itemize}
 \item {\it vnitřek} $\inter{M}$ jako množinu jejích vnitřních bodů
 \item {\it hranici} $\partial M$ jako množinu jejích hraničních bodů
 \item {\it uzávěr} $\close{M}$ jako množinu vnitřních a hraničních bodů. 
\end{itemize}

Pro vlastní potřebu označíme $Hr(M)$ množinu hromadných bodů $M$ a $Iz(M)$ množinu izolovaných bodů $M$.

Některé vztahy mezi zavedenými pojmy (rozmyslete si proč platí):
\begin{itemize}
 \item $\close{M} = M\cup \prtl M$
 \item $\inter{M} = M\subset \prtl M$ 

 \item $\close{M}=\inter{M}\cup \prtl M$ přičemž vnitřek a hranice jsou disjunktní: $\inter{M} \cap \prtl M = \emptyset$
 \item $\close{M}=Hr(M) \cup Iz(M)$ přičemž hromadné a izolované body jsou disjunktní: $Hr(M) \cap Iz(M) = \emptyset$

 \item množina izolovaných bodů $Izol(M)$ je podmnožinou hranice tj. $Iz(M) \subset \prtl M$
 \item vnitřek je podmnožina hromadných bodů tj. $\inter{M} \subset Hr(M)$

 \item hromadný bod množiny $M$ nemusí patřit do $M$ tj. neplatí $Hr(M) \subset M$ (najděte příklad)
 \item izolovaný bod $M$ vždy patří do $M$ tj. $Iz(M) \subset M$

 \item v každém okolí hromadného bodu množiny $M$ leží nekonečně bodů z $M$
 \item pro izolovaný bod $M$ existuje okolí ve kterém není žádný jiný bod z $M$
\end{itemize}

Množina $M\subset X$ je
\begin{itemize}
   \item otevřená $\Leftrightarrow \; M=M^0$,
   \item uzavřená $\Leftrightarrow \; M=\overline{M}$.
\end{itemize}
Platí, že množina $M$ je otevřená, pravě když její doplněk $M^c$ je uzavřená
a naopak. (důsledek de Morganových zákonů: $M\cup N=(M^c\cap N^c)^c$, $M\cap N=(M^c\cup N^c)^c$ pro množiny $\inter{M}$, $\prtl M$)


\subsection{Poslopnosti, úplnost, ekvivalence metri}
O prvku $x\in X$ řekneme, že je limitou posloupnosti $\{x_n\}\subset X$,
pokud $\varrho(x_n,x)\rightarrow 0$ pro $n\rightarrow\infty$,
tj. kovergenci prvků metrického prostoru převedeme na konvergenci reálné posloupnosti.
Definici můžeme formalizovat následovně:
výrok $\varrho(x_n,x)\rightarrow 0$ pro $n\rightarrow\infty$
je ekvivalentní výroku
$(\forall \varepsilon>0)(\exists N\in\mathbb{N})(\forall n\in\mathbb{N})
(n>N\;\Rightarrow\;\varrho(x_n,x)<\varepsilon)$.

Z jednoznačnosti limity reálné posloupnosti a vlastnosti~\ref{item:m1}
metriky plyne jednoznačnost limity posloupnosti v metrickém prostoru.

Můžeme-li na množině $X$ definovat více metrik, např. $\varrho_1$ a $\varrho_2$,
nemusí konvergence v jedné metrice být vždy ekvivalentní konvergenci v druhé metrice.
Toto platí jen pro tzv. ekvivalentní metriky, tj. metriky, pro něž existují
kladné konstanty $\alpha$ a $\beta$ tak, že 
$\alpha\varrho_1(x,y)\leq\varrho_2(x,y)\leq\beta\varrho_1(x,y)$.
Metriky $\varrho_1$ a $\varrho_2$ na $X$ jsou ekvivalentní, právě když
$(\forall x\in X)(\forall r>0)(\exists r_1<r)(\exists r_2>r)
 (B_{r_1}^{\varrho_1}\subset B_r^{\varrho_2}\subset B_{r_2}^{\varrho_1})$.

Množina $M$ je uzavřená, pokud každá konvergentní posloupnost z $M$
má limitu v $M$.

{Úplné metrické prostory}

$\{x_n\}\subset X$ je cauchyovská $\Leftrightarrow$
$(\forall\varepsilon>0)(\exists N\in\mathbb{N})(\forall m,n\in\mathbb{N})
(m,n>N\;\Rightarrow\;\varrho(x_m,x_n)<\varepsilon)$.
Metrický prostor $(X,\varrho)$ nazveme úplný, pokud každá cauchyovská posloupnost
$\{x_n\}\in X$ konverguje k prvku $x\in X$ v metrice $\varrho$.

Každá konvergentní posloupnost je nutně cauchyovská.

Uzavřený metrický podprostor (uzavřená podmnožina metrického prostoru
s metrikou daného prostoru) úplného metrického prostoru je úplný.

Příklady úplných a neúplných metrických prostorů:
   \begin{enumerate}
   \item Příkladem neúplného metrického prostoru je $(\mathbb{Q},|\cdot|)$.
         Víme, že posloupnost $x_n=(1+1/n)^n$, $n\in\mathbb{N}$,
         konverguje k číslu $e=\exp(1)\not\in\mathbb{Q}$.
   \item Metrický prostor $(\mathbb{R},|\cdot|)$ je úplný.
   \item Metrické prostory 
         $(\mathbb{R}^n,\varrho_1)$, $(\mathbb{R}^n,\varrho_2)$, 
         $(\mathbb{R}^n,\varrho_{\infty})$ jsou úplné.
   \item Metrický prostor $(C(I),\varrho_{\infty})$ je úplný,
         prostory $(C(I),\varrho_{1})$ a $(C(I),\varrho_{2})$ nikoliv.
         Obecně úplnost metrického prostoru závisí na použité metrice
         (viz diskuze o ekvivalentních metrikách výše).
   \item Diskrétní metrický prostor $X$ je úplný
         (cauchyovské posloupnosti mají tvar $(x,x,\ldots,x,\ldots)$
         a konvergují k $x$, kde $x\in X$).
   \end{enumerate}

\subsection{Cvičení}
V následujících úlohách máte uvést příklady množin splňující požadované vlastnosti. Na vašem řešení je
v tomto případě nejdůležitější argumentace, ža vámi uvedená množina skutečně splňuje dané vlastnosti.
Napište definice použitých pojmů.
\begin{enumerate}
 \item Najděte všechny hromadné a izolované body množin:
        \begin{enumerate}
                \item $A={-1}\cup(1,\infty)$ v prostoru $\Real$
                \vysl{$A$, \{-1\}}
                \item $B=\{x\in \Qnum\where x\in[0,1]\}\times\{y=1/n\where n\in\Nnum\}$ v prostoru $\Real^2$
                \vysl{$\{x\in \Real\}\times\{y=1/n \vee y=0\}$, $\emptyset$}
                \item $C=\{f_n(x)\where n\in \Nnum\}$,
                
                        \[ f_n(x)=\left\{\begin{aligned}
                                                \sin(nx), &&x\in(0,\frac x n)\\
                                                0,                &&x\in[\frac x n,1)                   
                                         \end{aligned}\right.\]
                         v normě $L^1$.
                \vysl{$f(x)=0$, $C$}     
                \item $C$ v normě $L^\infty$
                \vysl{$\emptyset$, $C$}
        \end{enumerate}
 \item Najděte množinu $M$ (v nějakém metrickém prostoru):
        \begin{enumerate}
                \item s prázdným vnitřkem a nekonečným množstvím hromadných bodů.
                \item s prázdným vnitřkem a nekonečným množstvím hromadných bodů, z nichž některé
                nepatří do $M$
                \item s prázdným vnitřkem a nekonečným množstvím hromadných bodů, z nichž nekonečno
                nepatří do $M$
                \item * s prázdným vnitřkem a nekonečným množstvím hromadných bodů, z nichž žádný
                nepatří do $M$
        \end{enumerate}
        \vysl{kružnice v $\Real$, kružnice s dírami, viz. předchozí příklad, za pivo}
        \item Existuje množina se dvěma izolovanými body a prázdnou hranicí? Pečlivě zdůvodněte.
\item Uveďte příklad spočetné množiny funkcí husté v prostoru $C((0,\pi))$.
                \vysl{ $ \{ \sum_{n=1}^\infty r_n \sin(nx)\where r_n\in \Qnum \}$ tedy něco jako lineární obal množiny
                $\{\sin(nx)\}$ ale jen s použitím racionálních čísel.}
 \item Ve které normě je prostor $C((0,1))$ úplný, a ve které je neúplný?
           \vysl{neuplný v $L^1$ - proč?, úplný v \dots to už je příliš lehké}
 \item Uveďte příklad metrického prostoru (množiny) $M$ funkcí na intervalu $(0,1)$, který je úplný v
       $L^\infty$ normě. Uveďte příklad prostoru $N$ funkcí na stejném intervalu, který je v
       $L^\infty$ normě naopak neúplný a je podmnožinou $M$.
       \vysl{ např. $N=\{f(x)=1/n\where n\in \Nnum\}$, $M=N\cup \{ f(x)=0 \} $}
  \item Najděte dva normované prostory $M$, $N$ splňující podmínky předchozího příkladu, ale s $L^1$
                normou. Tedy $N\subset M$, $M$ úplný a $N$ neúplný prostor funkcí v $L^1$-normě
                na intevalu $(0,1)$.
                \vysl{ např. $M=L^1((0,1))$, $N=C((0,1))$ }
\end{enumerate}


\section{Věta o pevném bodě a její aplikace}

Nechť $(X,\varrho)$ je metrický prostor a $A:\,X\rightarrow X$ je operátor na $X$.
Prvku $x\in X$ říkáme pevný bod operátoru $A$, pokud $Ax=x$.
Operátor $A$ nazýváme kontrakce (kontraktivní operátor), pokud existuje
$0\leq \alpha <1$ tak, že $\varrho(Ax,Ay)\leq\varrho(x,y)$ pro každé $x,y\in X$.

Každá kontrakce je spojitá.

Banachova věta o pevném bodě:
Každá kontrakce definovaná na úplném metrickém prostoru má právě jeden pevný bod.

Důkaz: Nechť $x_0\in X$ a definujme rekurzivně 
$x_n=Ax_{n-1}=A^2x_{n-2}=\cdots=A^nx_0$ pro $n\in\mathbb{N}$.
Nechť $m,n\in\mathbb{N}$, $m\leq n$, pak
\[
   \begin{split}
      \varrho(x_m,x_n)&=\varrho(A^mx_0,A^nx_0)\leq\alpha^m\varrho(x_0,x_{n-m})\\
                      &\leq\alpha^m(\varrho(x_0,x_1)+\varrho(x_1,x_2)+\cdots
                       +\varrho(x_{n-m-1},x_{n-m})\\
                      &=\alpha^m(\varrho(x_0,x_1)+\varrho(Ax_0,Ax_1)+\cdots
                       +\varrho(A^{n-m-1}x_0,A^{n-m-1}x_1)\\
                      &\leq\alpha^m\varrho(x_0,x_1)(1+\alpha+\cdots
                       +\alpha^{n-m-1})\\
                      &\leq\frac{\alpha^m}{1-\alpha}\varrho(x_0,x_1)
   \end{split}
\]
Přitom jsme využili trojúhelníkové nerovnosti~\ref{item:m3}
a součtu nekonečné geometrické řady s kvocientem $\alpha<1$.
Pro zvolené $\varepsilon>0$ a dostatečně velká $m$ je $\varrho(x_m,x_n)<\varepsilon$,
takže posloupnost $\{x_n\}$ je cauchyovská a v důsledku úplnosti $(X,\varrho)$
má limitu $x=\lim_{n\rightarrow\infty}x_n\in X$.
Protože $x_n\rightarrow x$, plyne ze spojitosti $A$, že
$Ax=A\lim_{n\rightarrow\infty}x_n=\lim_{n\rightarrow\infty}Ax_n
=\lim_{n\rightarrow\infty}x_{n+1}=x$.
Našli jsme tedy pevný bod $x$ kontrakce $A$ včetně konstruktivního návodu,
jak takový bod nalézt včetně odhadu rychlosti konvergence.
Jsou-li dále $x,y\in X$ dva pevné body kontrakce $A$, máme
$\varrho(x,y)=\varrho(Ax,Ay)\leq\alpha\varrho(x,y)$.
Toto je ale možné jen případě, že $\varrho(x,y)$, tj. $x=y$.
Tím jsme dokázali jednoznačnost pevného bodu.

{Aplikace věty o pevném bodě při numerickém řešení nelineárních rovnic.}
Řešme rovnici 
\begin{equation}\label{eq:nelin}
   f(x)=0
\end{equation}
kde $f$ je reálná funkce reálné proměnné $x$.
Jednou z metod pro její řešení je metoda pevného bodu.
Přepišme rovnici~(\ref{eq:nelin}) do tvaru
\[
   x = g(x)
\]
($g$ se nazývá iterační funkce).
Zvolme $x_0\in I$ a definujme
\begin{equation}\label{eq:iter}
   x_n = g(x_{n-1}), \; n\in\mathbb{N}.
\end{equation}
Předpokládejme, že existuje uzavřený interval $I$ tak, že $g(I)\subset I$,
iterační funkce je diferencovatelná na $I$ a platí $|g'(x)|\leq 1$ pro $x\in I$
a nějakou konstantu $0\leq K<1$.
Pak má~(\ref{eq:nelin}) řešení $x^*\in I$ a posloupnost $\{x_n\}$ konverguje k $x^*$.

Důkaz: Dvojice $(I,|\cdot|)$ je uzavřený metrický podprostor $(\mathbb{R},|\cdot|)$,
je tedy je úplný. Nechť $x,y\in I$, $x<y$. Pak
\[
   \left|\frac{g(y) - g(x)}{y-x}\right| = |g'(c)| \leq K,
\]
kde $c\in(x,y)$ (z věty o střední hodnotě).
Odtud $|g(y) - g(x)| \leq K|y-x|$ pro každé $x,y\in I$,
a tedy $g$ je kontrakce na $I$.
Z věty o pevném bodě tedy plyne, že existuje $x^*\in I$ takové,
že $x^*=g(x^*)$ a z důkazu této věty plyne $x^n\rightarrow x^*$.

Zpravidla se uvádí slabší formulace, kde předpokládáme, že $g$
je spojitě diferencovatelná na nějakém otevřeném okolí pevného bodu $x^*$
a platí na něm $|g'(x)|\leq K<1$. Zvolíme-li $x_0$ v dostatečně malém okolí
bodu $x^*$, konverguje posloupnost definovaná v~(\ref{eq:iter}) k $x^*$.


   Hledejme kořen rovnice $f(x)=x^2-x-2=0$.
   Zvolme $g(x)=\sqrt{x+2}$. 
   Možných voleb funkce $g$ je několik,
   ne všechny vedou ke konvergenci iterací~(\ref{eq:iter}).
   Volbou kladného znaménka odmocniny jsme redukovali množinu pevných bodů
   ze dvou na jeden.
   Zvolme $I=[0,7]$. Pak $g(0)\sqrt{2}\geq 0$ a $g(7)=3\leq 7$,
   je tedy $g(I)\subset I$.
   Derivací $g$ podle $x$ máme $g'(x)=1/(2\sqrt{x+2})$.
   Platí $g'(x)<1$ pro každé $x>-7/4$, tedy i pro $x\in I$.
   Odtud posloupnost iterací~(\ref{eq:iter}) konverguje ke kořenu
   rovnice $f(x)=0$ na $I$, tj. k $x^*=2$.
   Volíme-li např. $x_0=7$, máme
   $x_1=7.0000$, $x_2=3.0000$, $x_3=2.2361$, $x_4=2.0582$,
   $x_5=2.0145$, $x_6=2.0036$, $x_7=2.0009$ atd.


{Aplikace věty o pevném bodě v teorii diferenciálních rovnic.}
Nechť $G$ je otevřená souvislá množina (oblast) v $\mathbb{R}^2$,
$(x_0,y_0)\in G$ a nechť $f$ je spojitá na $G$ 
a splňuje Lipschitzovu podmínku v argumentu $y$
($f$ je lipschitzovská, lipschitzovsky spojitá)
$|f(x,y_1)-f(x,y_2)|\leq M|y_1-y_2|$ pro každé $(x,y_1)\in G$, $(x,y_2)\in G$,
kde $M>0$ je pevná konstanta (nezávislá na $x$, $y_1$, $y_2$,
tzv. Lipschitzova konstanta).
Pak existuje $\delta > 0$ tak, že rovnice $y'(x) = f(x,y)$ má právě jedno řešení
$y(x)=\varphi(x)$ na intervalu $[x_0-\delta,x_0+\delta]$ splňující
počáteční podmínku $\varphi(x_0)=y_0$.

Důkaz: Řešení rovnice $y'(x) = f(x,y)$ s počáteční podmínkou $\varphi(x_0)=y_0$
můžeme zapsat ve tvaru
\begin{equation}\label{eq:int_eq}
   \varphi(x) = y_0 + \int\limits_{x_0}^x f(x,\varphi(t))\,dt.
\end{equation}
Funkce $f$ je na $G$ spojitá (poznamenejme, že na funkci $f$
se můžeme dívat jako na operátor $f\,\mathbb{R}^2\rightarrow\mathbb{R}$),
tedy je omezená, tj. existuje $K>0$ tak, že $|f(x,y)|\leq K$ na nějaké
podoblasti $G'\subset G$ takové, že $(x_0,y_0)\in G'$.
Vyberme $\delta>0$ tak, že oblast $G'$ obsahuje obdélník
$R\equiv [x_0-\delta,x_0+\delta]\times [y_0-K\delta,y_0+K\delta]$
a zároveň $\delta\leq 1/M$.
Označme $C^*\equiv\{\psi\in C[x_0-\delta,x_0+\delta]\,|\,
(\forall x\in[x_0-\delta,x_0+\delta])(|\varphi(x)-y_0|\leq K\delta)\}$
a použijme metriku $\varrho_{\infty}$.
Metrický prostor $(C^*,\varrho_{\infty})$ je uzavřený podprostor prostoru
$(C[x_0-\delta,x_0+\delta],\varrho_{\infty})$, který je úplný, a tedy
$(C^*,\varrho_{\infty})$ je rovněž úplný.
Definujme operátor $\psi=A\phi$, $\phi\in C^*$ předpisem
\[
   \psi(x)=(A\varphi)(x)\equiv y_0+\int\limits_{x_0}^x f(t,\varphi(t))\,dt.
\]
Ukážeme, že je $A:\,C^*\rightarrow C^*$ a že $A$ je kontrakce.
Abychom ověřili, že $A$ je operátor do $C^*$, musíme ověřit podmínku
$(\forall x\in[x_0-\delta,x_0+\delta])(|(A\varphi)(x)-y_0|\leq K\delta)$.
Platí
\[
   |(A\varphi)(x)-y_0|=\left|\int\limits_{x_0}^xf(t,\varphi(t))\,dt\right|
                      \leq\int\limits_{x_0}^x|f(t,\varphi(t))|\,dt
                      \leq K|x-x_0|\leq K\delta.
\]
Abychom ověřili kontraktivnost operátoru, nechť $\varphi,\psi\in C^*$, 
$x\in[x_0-\delta,x_0+\delta]$, pak
\[
  \begin{split}
   |(A\varphi)(x)-(A\psi)(x)|&=
    \left|\int\limits_{x_0}^xf(t,\varphi(t))-f(t,\psi(t)\,dt\right|\\
    &\leq\int\limits_{x_0}^x|f(t,\varphi(t))-f(t,\psi(t)|\,dt\\
    &\leq M\int\limits_{x_0}^x|\varphi(t)-\psi(t)|\,dt
    \leq M\delta\,\varrho_{\infty}(\varphi,\psi),
  \end{split}
\]
odkud $\varrho_{\infty}(A\varphi,A\psi)\leq M\delta\varrho(\varphi,\psi)
=\alpha\varrho(\varphi,\psi)$, kde $\alpha\equiv M\delta<1$.
Z věty o pevném bodě existuje tedy $\varphi\in C^*$ tak, že
$A\varphi=\varphi$, tj. existuje právě jedno řešení integrální 
rovnice~(\ref{eq:int_eq}) v $C^*$.

\subsection{Cvičení}


\begin{enumerate}
   \item Dokažte pomocí BVPP, že posloupnost $x_n = cos(x_n)$ má pevný bod na intervalu $[0,1]$. Odhadněte počet iterací pořebných k získání 4 platných cifer a porovnejte to s 
         výpočtem na kalkulačce.
   \item Řešte pomocí BVPP rovnici $x^2-1 = 0$.
   \item Řešme rovnici $x^2-2x+2 = 0$. Použijeme iterace $x_{n+1} = (x_n^2+2)/2$. Ukažte, že Lipschicovská konstanta je menší než $1$ na intervalu $[-1/2, 1/2]$. Podle BVPP by měl existovat pevný bod, ale
     ůvodní rovnice nemá žádný reálný kořen (ověřte). Kde je chyba?
   \item Použijte větu o pevném bodě pro řešení rovnice:
  \[ \cos x = e^x\]
   Rovnici upravte do tvaru $\arctan( \argdot) = x$. 
   Kolik má uvedená rovnice řešení? Zvolte vhodné počáteční přiblížení a proveďte 4 iterace pro získání největšího nenulového řešení. Lze pomocí stejného tvaru hledat všechna řešení uvedené rovnice? Zdůvodněte.
  \item 
   Rovnice $\sqrt{x}=e^x-1$ má řešení na intervalu $I=(0,1)$, pro ověření načrtněte grafy. Proveďte transformaci na tvar
   $f(x)=\ln(\sqrt(x)+1)=x$ a ověřte předpoklady věty o pevném bodě, tj. že funkce $f$ zobrazuje interval $I$ do sebe,
   a že má derivaci v absolutní hodnotě menší než 1.
   Iteracemi najděte řešení s přesností na 2 desetinná místa.
   \item
   Najděte maximum funkce $f(x)=x\sin x$ na intervalu $[0,2\pi]$ s přesností na 2 desetinná místa. Pro řešení nelineární rovnice použijte větu o pevném bodě. Diskutujte její předpoklady.
   \item

\end{enumerate}





