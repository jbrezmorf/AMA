\documentclass[fleqn,12pt]{report}
\usepackage{a4}
%\documentstyle[fleqn,czech,a4,12pt]{report}
\newcommand{\e}{{\rm e}}
%\newcommand{\klasif}{\vfill {\bf Klasifikace:}\begin{enumerate}\item 20--17 bod�\item 16--13 bod�\item 12--10 bod�\item 9--0 bod�\end{enumerate}}
\newcommand{\nadpis}[1]{\newpage\noindent{\Huge Applied Mathematics}

\begin{flushright}
\_ \_ \_ \_ \_ \_ \_ \_ \_ \_ \_ \_ \_ \_ \_ \_ \_ \_\\
name and surname~~~~~~~~
\end{flushright}

\noindent{\Large #1}

~

}
\pagestyle{empty}
\headheight 0pt
\headsep 0pt
\begin{document}
\nadpis{Excercise Nr. 1}
Find the solution of linear ordinary differential equations with the right-hand side
\begin{displaymath}
\begin{array}{rcrcrcr}
\dot x_1&=&~    &~&3 x_2&+& 2\e^{2t}\\
\dot x_2&=&- x_1&+&4 x_2&+& \e^{2t}
\end{array}
\end{displaymath}
with the initial condition $x_1(0)=-1, x_2(0)=1$.

\nadpis{Excercise Nr. 2}
Construct the equations of the Finite Difference Method for the solution of the 
elliptic partial differential equation
\begin{displaymath}
2\frac{\partial^2}{\partial x^2} u(x,y) + 3\frac{\partial^2}{\partial y^2} u(x,y)+ u(x,y)=(x-y)^2
\end{displaymath}
in the rectangular region $\Omega=\langle 0;1\rangle\times\langle 0;0.8\rangle$ 
with the Dirichlet boundary conditions $u(0,y)= -y$, $u(1,y)= 2y$, $u(x,0)= 0$, $u(x, 0.8)= -0.8+2.4 x$,
with the mesh step $h=\frac14$ and the initial mesh point $[0;0]$.

~

Draw the mesh, denote its nodes, classify them to internal and boundary ones, and
construct the equation system for approximation of the values of the function $u$ 
in all nodes. Write down the values of the function $u$ in the nodes lying at 
the region boundary. Construct the equations for boundary nodes lying out of the 
boundary using the linear interpolation method.

\nadpis{Theoretical part}

{\Large Question 1:}
Verify whether the set $M = \{ p(x) = a x^3 + b x^2 + c x + d ;  \frac{d p}{d x} (1) = 3b \}$ 
is a vector subspace of the space of polynomials up to the third order. Discuss it.

\vspace{6cm}
{\Large Question 2:}
Formulate the Riesz Theorem on Linear Functional Representation in a Hilbert
Space. Write down the main ideas used in its proof.

\vspace{5cm}
{\Large Question 3:}
What is the General One Step Method? Write down which types of problems can be 
solved using it and write down its general formula. Write down an example of one 
specific one step method.

\vfill
~\hfill{\tiny turn over}\newpage

{\Large Question 4:}
Write down a general boundary problem governed by a {\em system of ordinary 
differential equations}. Which special types of boundary conditions for a system 
of ordinary differential equations of the first order do you know?

\vspace{6cm}
{\Large Question 5:}
Verify which of the differential operators
$A_1 (u) = ( -4 \frac{\partial ^2 u}{\partial x^2} -  \frac{\partial ^2 u}{\partial y^2}  +  u  )$, 
$A_2 (u) = ( -4x \frac{\partial ^2 u}{\partial x^2} - y \frac{\partial ^2 u}{\partial y^2} )$ 
is symmetrical and which one is positive deffinite
in the space of twice continuously differentiable functions with zero trace on the boundary
$\partial\Omega$ where $\Omega$ is the Cartesian product of intervals $(0;2)\times (0;2)$.
Give reasons for your answer -- discuss it.

{\bf Hint:} 
symmetrical operator:  $(A u , v) = (u, A v)$ for all $u,v \in C^2(\Omega)$  and  
positive deffinite operator $(A u, u) \geq c (u,u)$ where $c>0$.

\vspace{6cm}
{\Large Question 6:}
Let is given the general problem:
$A u = f$ in $\Omega$
with the boundary condition $u = 0$ on $\partial \Omega$.
Formulate the Theorem on the Minimum of a Quadratic Functional.
Deduce the Ritz Method for finding the approximate solution of this problem for basis 
vectors $\varphi_1,\dots,\varphi_n$. Write down the conditions sufficient for 
convergence of the sequence of approximate solutions to the solution $u_0$ of 
the equation $A u = f$.

\end{document}
