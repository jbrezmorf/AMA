\documentclass[fleqn,12pt]{report}
\usepackage{a4}
%\documentstyle[fleqn,czech,a4,12pt]{report}
\newcommand{\e}{{\rm e}}
%\newcommand{\klasif}{\vfill {\bf Klasifikace:}\begin{enumerate}\item 20--17 bod�\item 16--13 bod�\item 12--10 bod�\item 9--0 bod�\end{enumerate}}
\newcommand{\nadpis}[1]{\newpage\noindent{\Huge Applied Mathematics}

\begin{flushright}
\_ \_ \_ \_ \_ \_ \_ \_ \_ \_ \_ \_ \_ \_ \_ \_ \_ \_\\
name and surname~~~~~~~~
\end{flushright}

\noindent{\Large #1}

~

}
\pagestyle{empty}
\headheight 0pt
\headsep 0pt
\begin{document}
\nadpis{Excercise Nr. 1}
Find the solution of linear ordinary differential equations with the right-hand side
\begin{displaymath}
\begin{array}{rcrcrcr}
\dot x_1&=&3 x_1&-&4 x_2&+& \e^{2t}\\
\dot x_2&=&2 x_1&-&3 x_2&-&2\e^{2t}
\end{array}
\end{displaymath}
with the initial condition $x_1(0)=2, x_2(0)=3$.

\nadpis{Excercise Nr. 2}
Construct the equations of the Finite Difference Method for the solution of the 
elliptic partial differential equation
\begin{displaymath}
3\frac{\partial^2}{\partial x^2} u(x,y) + \frac{\partial^2}{\partial y^2} u(x,y)-2 u(x,y)=2xy+1
\end{displaymath}
in the rectangular region $\Omega=\langle 0;\frac 23\rangle\times\langle 0;0{,}8\rangle$ 
with Dirichlet boundary conditions $u(0,y)= -y$, $u(\frac 23,y)= 2y$, $u(x,0)= 0$, $u(x, 0{,}8)= -0{,}8+3{,}6 x$,
with the mesh step $h=\frac15$ and the initial mesh point $[0;0]$.

~

Draw the mesh, denote its nodes, classify them to internal and boundary ones, and
construct the equation system for approximation of the values of the function $u$ 
in all nodes. Write down the values of the function $u$ in the nodes lying at 
the region boundary. Construct the equations for boundary nodes lying out of the 
boundary using the linear interpolation method.

\nadpis{Theoretical part}

{\Large Question 1:}
Verify whether the set $M = \{ p(x) = a x^3 + a x^2 + b x + c ;  \frac{d^2 p}{d x ^2} (1) = b \}$
is a vector subspace of the space of polynomials up to the second degree. Discuss it.

\vspace{6cm}
{\Large Question 2:}
Explain what does it mean that the set $M$ is dense in the metric space $X$. Is the space of 
continuous functions $C((-1; 1))$ dense in the space $L^2((-1; 1))$? Explain what does this question 
mean for functions from $L^2$.

\vspace{5cm}
{\Large Question 3:}
Write down the properties of fundamental matrix and standard fundamental matrix 
of solution of a system of linear ordinary differential equations. How the standard 
fundamental matrix differs from a general fundamental matrix?

\vfill
~\hfill{\tiny turn over}\newpage

{\Large Question 4:}
What is it a general one-step method? Write down the class of the problems that can be solved using it.
Write down its formula and an example of any specific one-step method.

\vspace{6cm}
{\Large Question 5:}
Using the integration per partes or Green Theorem verify that the differential operator
$A(u)=(-x\frac{\partial ^2 u}{\partial x^2} - y\frac{\partial ^2 u}{\partial y^2}  +  u  )$ 
is symmetric positive deffinite so that
$(A u , v) = (u, A v)$ for all $u,v \in L^2(\Omega)$  and
$(A u, u) \geq c (u,u)$ where $c>0$,
in the space of twice continuously differentiable functions with zero trace on 
the boundary $\partial \Omega$ where $\Omega$ is the Cartesian product of intervals $(-1;0)\times (0;1)$.

\vspace{6cm}
{\Large Question 6:}
Let is given the general problem:
$A u = f$ in $\Omega$
with the boundary condition $u = 0$ on $\partial \Omega$.
Formulate the Theorem on the Minimum of a Quadratic Functional.
Deduce the Least Square Method (Courant Method) for finding the approximate solution of this problem 
for basis vectors $\varphi_1,\dots,\varphi_n$. Write down the conditions sufficient for 
convergence of the sequence of approximate solutions to the solution $u_0$ of 
the equation $A u = f$.

\end{document}
