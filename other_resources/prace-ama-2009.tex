\documentclass[a4paper,10pt]{article}
%\usepackage[active]{srcltx}
\usepackage[czech]{babel}
\usepackage[utf8]{inputenc}

\usepackage{amsmath}
\usepackage{amsfonts}
\usepackage{amssymb}
\usepackage{amsthm}
%
\newtheorem{theorem}{Věta}[section]
\newtheorem{proposition}[theorem]{Tvrzení}
\newtheorem{definition}[theorem]{Definice}
\newtheorem{remark}[theorem]{Poznámka}
\newtheorem{lemma}[theorem]{Lemma}
\newtheorem{corollary}[theorem]{Důsledek}
\newtheorem{exercise}[theorem]{Cvičení}

%\numberwithin{equation}{document}
%
\def\div{{\rm div}}
\def\Lapl{\Delta}
\def\grad{\nabla}
\def\supp{{\rm supp}}
\def\dist{{\rm dist}}
%\def\chset{\mathbbm{1}}
\def\chset{1}
%
\def\Tr{{\rm Tr}}
\def\to{\rightarrow}
\def\weakto{\rightharpoonup}
\def\imbed{\hookrightarrow}
\def\cimbed{\subset\subset}
\def\range{{\mathcal R}}
\def\leprox{\lesssim}
\def\argdot{{\hspace{0.18em}\cdot\hspace{0.18em}}}
\def\Distr{{\mathcal D}}
\def\calK{{\mathcal K}}
\def\FromTo{|\rightarrow}
\def\convol{\star}
\def\impl{\Rightarrow}
\DeclareMathOperator*{\esslim}{esslim}
\DeclareMathOperator*{\esssup}{ess\,supp}
\DeclareMathOperator{\ess}{ess}
\DeclareMathOperator{\osc}{osc}
\DeclareMathOperator{\curl}{curl}
%
%\def\Ess{{\rm ess}}
%\def\Exp{{\rm exp}}
%\def\Implies{\Longrightarrow}
%\def\Equiv{\Longleftrightarrow}
% ****************************************** GENERAL MATH NOTATION
\def\Real{{\rm\bf R}}
\def\Rd{{{\rm\bf R}^{\rm 3}}}
\def\RN{{{\rm\bf R}^N}}
\def\D{{\mathbb D}}
\def\Nnum{{\mathbb N}}
\def\Measures{{\mathcal M}}
\def\d{\,{\rm d}}               % differential
\def\sdodt{\genfrac{}{}{}{1}{\rm d}{{\rm d}t}}
\def\dodt{\genfrac{}{}{}{}{\rm d}{{\rm d}t}}
\def\dx{\d x}
\def\Lin{\mathit{Lin}}
%
\def\vc#1{\mathbf{\boldsymbol{#1}}}     % vector
\def\tn#1{{\mathbb{#1}}}    % tensor
\def\abs#1{\lvert#1\rvert}
\def\Abs#1{\bigl\lvert#1\bigr\rvert}
\def\bigabs#1{\bigl\lvert#1\bigr\rvert}
\def\Bigabs#1{\Big\lvert#1\Big\rvert}
\def\ABS#1{\left\lvert#1\right\rvert}
\def\norm#1{\bigl\Vert#1\bigr\Vert} %norm
\def\close#1{\overline{#1}}
\def\inter#1{#1^\circ}
\def\eqdef{\mathrel{\mathop:}=}     % defining equivalence
\def\where{\,|\,}                    % "where" separator in set's defs
\def\timeD#1{\dot{\overline{{#1}}}}
%
% ******************************************* USEFULL MACROS
\def\RomanEnum{\renewcommand{\labelenumi}{\rm (\roman{enumi})}}   % enumerate by roman numbers
\def\rf#1{(\ref{#1})}                                             % ref. shortcut
\def\prtl{\partial}                                        % partial deriv.
\def\Names#1{{\scshape #1}}
\def\rem#1{{\parskip=0cm\par!! {\sl\small #1} !!}}
\def\reseni#1{\par{\bf Řešení:}#1}
%
%
% ******************************************* DOCUMENT NOTATIONS
% document specific
%***************************************************************************
%
\addtolength{\textwidth}{2cm}
\addtolength{\vsize}{2cm}
\addtolength{\topmargin}{-1cm}
\addtolength{\hoffset}{-1cm}
\begin{document}
\parskip=2ex
\parindent=0pt
\pagestyle{empty}
 \section{Zadání příkladů pro semestrální práci - AMA 2009}
 Z každé sekce vypracujte jednu úlohu podle kombinace, kterou jste si vylosovali. Vypracovanou práci odevzdejte ve vlastním zájmu nejpozději první týden po vánocích. Za každou vyřešenou úlohu vám klesne limit pro zápočet u závěrečné písemky o 10\%.
 Při maximální aktivitě na cvičeních v rámci vaší skupiny máte limit nižší o 20\%.
 \subsection{Normovaný prostor}
 \begin{enumerate}
  \item Uvažujme vektorový prostor $P^4=\{ax^3+bx^2+cx+d\}$ polynomů třetího stupně. Rozhodněte, které z následujících
        funkcí na tomto vektorovém prostoru jsou normy, a které nejsou normy a proč.
   \begin{enumerate}
       \item $\norm{p}=a+b+c+d$ , kde $a$, $b$, $c$, $d$ jsou koeficienty polynomu $p$.
       \item $\norm{p}=\abs{p''(1)}+\abs{p''(-1)}+\abs{p'(1)}+\abs{p'(-1)}$
       \item $\norm{p}=\abs{p'''(1)}+\abs{p''(-1)}+\abs{p'(1)}+\abs{p(-1)}$
       \item $\norm{p}=\int_0^1\abs{p'(x)}^2\dx+\abs{p(0)}$
   \end{enumerate}
  \item Pro funkce 
       \[
        \norm{p}_a=\sqrt{\int_0^1 \abs{p''}^2 \dx},\quad \norm{p}_b=\sup_{(0,1)} \abs{2p'-p}
       \] 
       najděte neprázdné podprostory $P_a$, $P_b$ prostoru $P^4$ polynomů třetího stupně tak, aby funkce byla na příslušném podprostoru normou. 
  \item Dokažte, že 
        \[
                \norm{f}_{1*}=\int_0^1 x^2 \abs{f(x)} \dx 
        \]
        je normou na $L^1(0,1)$. Spočtěte $\norm{e^x}_1$ a $\norm{e^x}_{1*}$.   
 \end{enumerate}
 \subsection{Výpočet norem. Konvergence. Množiny v metrických prostorech.}
 \begin{enumerate}
  \item Vypočtěte velikost funkce
  \[
      f(x,y)=(x^2+y^2)^{-p/2}
  \]
  v prostorech $H^1(B)$ a $L^3(B)$, kde $B$ je jednotková koule v prostoru $\Real^2$. Pro které hodnoty parametru $p$ patří funkce $f$ do prostoru $H^1(B)$?
  \item V každém metrickém prostoru jsou právě dvě množiny, které jsou uzavřené i otevřené. Které to jsou? Nalezněte podmnožinu $M$ prostoru $\Real$, která má neprázdný vnitřek a dva hromadné body, které nepatří do $M$. Kolik izolovaných bodů má množina
  $\{\sin(nx)\}_{i=1}^\infty$ v prostoru $L^2(0,2\pi)$?
  \item Určete limitu posloupnosti funkcí
     \[
        f_n(x)=\sin^n(x)
     \]
     pro $n\to \infty$ v prostorech $L^1(0,2\pi)$ a $L^\infty(0,2\pi)$.
 \end{enumerate}

 \subsection{Věta o pevném bodě}
 \begin{enumerate}
  \item Použijte větu o pevném bodě pro řešení rovnice:
  \[ \cos x = e^x\]
   Rovnici upravte do tvaru $\arctan( \argdot) = x$. 
   Kolik má uvedená rovnice řešení? Zvolte vhodné počáteční přiblížení a proveďte 4 iterace pro získání největšího nenulového řešení. Lze pomocí stejného tvaru hledat všechna řešení uvedené rovnice? Zdůvodněte.
  \item 
   Rovnice $\sqrt{x}=e^x-1$ má řešení na intervalu $I=(0,1)$, pro ověření načrtněte grafy. Proveďte transformaci na tvar
   $f(x)=\ln(\sqrt(x)+1)=x$ a ověřte předpoklady věty o pevném bodě, tj. že funkce $f$ zobrazuje interval $I$ do sebe,
   a že má derivaci v absolutní hodnotě menší než 1.
   Iteracemi najděte řešení s přesností na 2 desetinná místa.
   \item
   Najděte maximum funkce $f(x)=x\sin x$ na intervalu $[0,2\pi]$ s přesností na 2 desetinná místa. Pro řešení nelineární rovnice použijte větu o pevném bodě. Diskutujte její předpoklady.
 \end{enumerate}


 \subsection{Hilbertovy prostory}
 \begin{enumerate}
  \item Pokud je množina vektorů $\{e_i\}$ ortonormélní bází v Hilbertova prostoru $X$, pak platí zobecněná Pythagorova věta, tzv. Parsevalova rovnost:
  \begin{equation}\label{parseval}
      \sum_{i=1}^\infty \abs{(e_i, f)}^2 = \norm{f}^2
  \end{equation}
  a to pro libovolný prvek $f\in X$. Rozviňte funkci $f(x)=x$ do Fourierovy řady v prostoru $L^2(-\pi,\pi)$ a použijte 
  \eqref{parseval} pro výpočet součtu 
  \[
     \sum_{i=1}^\infty \frac{1}{n^2} = ?
  \]
  \item Najděte ortonormální bázi prostoru $\Lin\{1,x,x^2\}\subset H^1(0,1)$. Na prostoru $H^1(0,1)$ je definován standardní skalární součin
  \[
     (f,g)= \int_0^1 f'(x)g'(x)+f(x)g(x) \dx. 
  \]
  $\Lin\, M$ značí lineární obal množiny $M$, tedy podprostor všech lineárních kombinací vektorů z množiny $M$.
  \item Uvažujte prostor $H^1(0,1)$ s nestandardním skalárním součinem 
  \[
     (f,g)=\int_0^1 x^2 f'(x) g'(x) +f(x) g(x) \dx
  \]
  ověřte, že se skutečně jedná o skalární součin. A pro funkce $1$ a $e^x$ ověřte Schwartzovu nerovnost.
 \end{enumerate}
\end{document}
