\documentclass[a4,11pt]{article}
%\usepackage[active]{srcltx}
\usepackage[czech]{babel}
\usepackage[utf8]{inputenc}

\usepackage{amsmath}
\usepackage{amsfonts}
\usepackage{amssymb}
\usepackage{amsthm}
%
\newtheorem{theorem}{Věta}[section]
\newtheorem{proposition}[theorem]{Tvrzení}
\newtheorem{definition}[theorem]{Definice}
\newtheorem{remark}[theorem]{Poznámka}
\newtheorem{lemma}[theorem]{Lemma}
\newtheorem{corollary}[theorem]{Důsledek}
\newtheorem{exercise}[theorem]{Cvičení}

%\numberwithin{equation}{document}
%
\def\div{{\rm div}}
\def\Lapl{\Delta}
\def\grad{\nabla}
\def\supp{{\rm supp}}
\def\dist{{\rm dist}}
%\def\chset{\mathbbm{1}}
\def\chset{1}
%
\def\Tr{{\rm Tr}}
\def\to{\rightarrow}
\def\weakto{\rightharpoonup}
\def\imbed{\hookrightarrow}
\def\cimbed{\subset\subset}
\def\range{{\mathcal R}}
\def\leprox{\lesssim}
\def\argdot{{\hspace{0.18em}\cdot\hspace{0.18em}}}
\def\Distr{{\mathcal D}}
\def\calK{{\mathcal K}}
\def\FromTo{|\rightarrow}
\def\convol{\star}
\def\impl{\Rightarrow}
\DeclareMathOperator*{\esslim}{esslim}
\DeclareMathOperator*{\esssup}{ess\,supp}
\DeclareMathOperator{\ess}{ess}
\DeclareMathOperator{\osc}{osc}
\DeclareMathOperator{\curl}{curl}
%
%\def\Ess{{\rm ess}}
%\def\Exp{{\rm exp}}
%\def\Implies{\Longrightarrow}
%\def\Equiv{\Longleftrightarrow}
% ****************************************** GENERAL MATH NOTATION
\def\Real{{\rm\bf R}}
\def\Rd{{{\rm\bf R}^{\rm 3}}}
\def\RN{{{\rm\bf R}^N}}
\def\D{{\mathbb D}}
\def\Nnum{{\mathbb N}}
\def\Measures{{\mathcal M}}
\def\d{\,{\rm d}}               % differential
\def\sdodt{\genfrac{}{}{}{1}{\rm d}{{\rm d}t}}
\def\dodt{\genfrac{}{}{}{}{\rm d}{{\rm d}t}}
%
\def\vc#1{\mathbf{\boldsymbol{#1}}}     % vector
\def\tn#1{{\mathbb{#1}}}    % tensor
\def\abs#1{\lvert#1\rvert}
\def\Abs#1{\bigl\lvert#1\bigr\rvert}
\def\bigabs#1{\bigl\lvert#1\bigr\rvert}
\def\Bigabs#1{\Big\lvert#1\Big\rvert}
\def\ABS#1{\left\lvert#1\right\rvert}
\def\norm#1{\bigl\Vert#1\bigr\Vert} %norm
\def\close#1{\overline{#1}}
\def\inter#1{#1^\circ}
\def\eqdef{\mathrel{\mathop:}=}     % defining equivalence
\def\where{\,|\,}                    % "where" separator in set's defs
\def\timeD#1{\dot{\overline{{#1}}}}
%
% ******************************************* USEFULL MACROS
\def\RomanEnum{\renewcommand{\labelenumi}{\rm (\roman{enumi})}}   % enumerate by roman numbers
\def\rf#1{(\ref{#1})}                                             % ref. shortcut
\def\prtl{\partial}                                        % partial deriv.
\def\Names#1{{\scshape #1}}
\def\rem#1{{\parskip=0cm\par!! {\sl\small #1} !!}}
\def\reseni#1{\par{\bf Řešení:}#1}
%
%
% ******************************************* DOCUMENT NOTATIONS
% document specific
%***************************************************************************
%
\addtolength{\textwidth}{3cm}
\addtolength{\textheight}{3cm}
\addtolength{\topmargin}{-1.5cm}
\addtolength{\hoffset}{-1.5cm}
\begin{document}
\parskip=2ex
\parindent=0pt
\pagestyle{empty}
\begin{enumerate}
 \item Metrický prostor, příklady. Množiny v metrických prostorech:\\ 
       uzavřená, otevřená, hranice
 \item Úplnost, Banachova věta, aplikace.

\hrule
 \item Vektorové prostory polynomů a funkcí.
 \item Normované prostory, příklady, vlastnosti normy, ověřování? 
 \item Výpočty norem v různých normovaných prostorech.
 \item Konvergence v normovaných prostorech. Výpočetně nenáročné, ale zajímavé příklady.
 
\hrule

 \item prostory se skal. součinem, příklady skla. souč. a prostorů, ověřování
 \item Schwartzova nerovnost. Indukovaná norma.
 \item orto. báze, Fourierovy řady, orto. proces, orto polynomy, kráté rekurence
 \item Slabá derivace, Greenova věta, Galerkinova metoda
 \item Lax-Milgramova věta, aplikace Rieszovy věty, příklady pro konkrétní rovnice.

\hrule
 %základní teorie, částečně skripta od H. Šembery, řešený příklady a neřešený př. zadání + výsledky (ne u příkladu ale na konci kapitoly)
 %Putzerova metoda, navíc
 %Předělat vronského determinanty
 \item opakování ODR 
 \item Soustavy ODR, prostor řešení, wronskián, převod rovnic vyššího řádu
 \item Výpočet SFM pomocí vlastních vektorů.
 \item Výpočet SFM metodou rozvoje v řadu, reálný a komplexní případ
 \item Řešení nehomogenních rovnic.

{\bf skupiny PI a NANO}

 \item typy PDR
 \item Aproximace derivací. Metoda sítí  pro okrajové úlohy.
 \item Metoda sítí pro eliptické rovnice.
 \item Metoda sítí pro parabolické rovnice.
 \item písemka

\hrule
{\bf skupiny AŘ a ME}
 \item ODR, okrajové úlohy
 \item Numerické metody pro ODR (RK, explicitní/implicitní), stabilita
 \item Minimalizační úlohy, vícekrokové metody
 \item Singulární rozklad. Číslo podmíněnosti.
 \item písemka
\end{enumerate}

\end{document}
