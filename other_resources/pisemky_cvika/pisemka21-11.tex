\documentclass[a4paper,10pt]{article}
%\usepackage[active]{srcltx}
\usepackage[czech]{babel}
\usepackage[utf8]{inputenc}

\usepackage{amsmath}
\usepackage{amsfonts}
\usepackage{amssymb}
\usepackage{amsthm}
%
\newtheorem{theorem}{Věta}[section]
\newtheorem{proposition}[theorem]{Tvrzení}
\newtheorem{definition}[theorem]{Definice}
\newtheorem{remark}[theorem]{Poznámka}
\newtheorem{lemma}[theorem]{Lemma}
\newtheorem{corollary}[theorem]{Důsledek}
\newtheorem{exercise}[theorem]{Cvičení}

%\numberwithin{equation}{document}
%
\def\div{{\rm div}}
\def\Lapl{\Delta}
\def\grad{\nabla}
\def\supp{{\rm supp}}
\def\dist{{\rm dist}}
%\def\chset{\mathbbm{1}}
\def\chset{1}
%
\def\Tr{{\rm Tr}}
\def\to{\rightarrow}
\def\weakto{\rightharpoonup}
\def\imbed{\hookrightarrow}
\def\cimbed{\subset\subset}
\def\range{{\mathcal R}}
\def\leprox{\lesssim}
\def\argdot{{\hspace{0.18em}\cdot\hspace{0.18em}}}
\def\Distr{{\mathcal D}}
\def\calK{{\mathcal K}}
\def\FromTo{|\rightarrow}
\def\convol{\star}
\def\impl{\Rightarrow}
\DeclareMathOperator*{\esslim}{esslim}
\DeclareMathOperator*{\esssup}{ess\,supp}
\DeclareMathOperator{\ess}{ess}
\DeclareMathOperator{\osc}{osc}
\DeclareMathOperator{\curl}{curl}
%
%\def\Ess{{\rm ess}}
%\def\Exp{{\rm exp}}
%\def\Implies{\Longrightarrow}
%\def\Equiv{\Longleftrightarrow}
% ****************************************** GENERAL MATH NOTATION
\def\Real{{\rm\bf R}}
\def\Rd{{{\rm\bf R}^{\rm 3}}}
\def\RN{{{\rm\bf R}^N}}
\def\D{{\mathbb D}}
\def\Nnum{{\mathbb N}}
\def\Measures{{\mathcal M}}
\def\d{\,{\rm d}}               % differential
\def\sdodt{\genfrac{}{}{}{1}{\rm d}{{\rm d}t}}
\def\dodt{\genfrac{}{}{}{}{\rm d}{{\rm d}t}}
%
\def\vc#1{\mathbf{\boldsymbol{#1}}}     % vector
\def\tn#1{{\mathbb{#1}}}    % tensor
\def\abs#1{\lvert#1\rvert}
\def\Abs#1{\bigl\lvert#1\bigr\rvert}
\def\bigabs#1{\bigl\lvert#1\bigr\rvert}
\def\Bigabs#1{\Big\lvert#1\Big\rvert}
\def\ABS#1{\left\lvert#1\right\rvert}
\def\norm#1{\bigl\Vert#1\bigr\Vert} %norm
\def\close#1{\overline{#1}}
\def\inter#1{#1^\circ}
\def\eqdef{\mathrel{\mathop:}=}     % defining equivalence
\def\where{\,|\,}                    % "where" separator in set's defs
\def\timeD#1{\dot{\overline{{#1}}}}
%
% ******************************************* USEFULL MACROS
\def\RomanEnum{\renewcommand{\labelenumi}{\rm (\roman{enumi})}}   % enumerate by roman numbers
\def\rf#1{(\ref{#1})}                                             % ref. shortcut
\def\prtl{\partial}                                        % partial deriv.
\def\Names#1{{\scshape #1}}
\def\rem#1{{\parskip=0cm\par!! {\sl\small #1} !!}}
\def\reseni#1{\par{\bf Řešení:}#1}
%
%
% ******************************************* DOCUMENT NOTATIONS
% document specific
%***************************************************************************
%
\addtolength{\textwidth}{2cm}
\addtolength{\vsize}{2cm}
\addtolength{\topmargin}{-1cm}
\addtolength{\hoffset}{-1cm}
\begin{document}
\parskip=2ex
\parindent=0pt
\pagestyle{empty}
\section{Písemka 21.11. verze A}
\subsection{Počítací příklady}
\begin{enumerate}
 \item Ověřte zda množina polynomů $M=\{p(x)=ax^2+bx+c\where p'(1)=c\}\subset \mathcal P^2$ je vektorovým podprostorem polynomů max. druhého stupně.
 Ověřte zda $\norm{p}=\abs{p(1)}+\abs{p'(1)}$ je normou na $M$.
 \item Vypočtěte $\norm{f}_{L^1([-1,3])}$ a $\norm{f}_{L^\infty([0,3])}$ pro $f(x)=x(x^2-6x+9)$
 \item Určete bodovou limitu posloupnosti
 		\[f_n(x)=
                \left\{\begin{aligned}
                        n^{1/2} \text{ na $[0,\frac1n]$}\\
                        0 \text{ na $(\frac1n,1]$}
                       \end{aligned}\right. .
               \]
        Rozhodněte zda posloupnost konverguje  v normách $L^1([0,1])$ a $L^2([0,1])$

\subsection{Teorie}
\item Napište definici vektorového prostoru (7 vlastností). Uveďte dva příklady vektorových prostorů, z nichž alespoň jeden je prostor,
jehož prvky jsou funkce.

\item  Pomocí integrace per partes resp. Greenovy věty ověřte, že diferenciální operátor 
\[
A[f]=\big(-2\frac{\prtl^2 f}{\prtl x^2} -3\frac{\prtl^2 f}{\prtl y^2} +  f\big)
\]
je symetrický a pozitivně definitní, tj. 
\[
    (Af,g)_{L^2(\Omega)}=(f,Ag)_{L^2(\Omega)},\text{ a } (Af,f) \ge c(f,f),\ c>0
\]
na prostoru dvakrát spojitě diferencovatelných funkcí nulových na hranici:
\[
C^2_0(\Omega)=\{ f\in C^2(\close{\Omega}),\ f=0 \text{ na }\prtl\Omega\},\quad \Omega=(-1,1)\times(1,1).
\]
Na tomto prostoru použijte skalární součin prostoru $L^2(\Omega)$. Je tento operátor lineární? Proč?

\item Zformulujte Rieszovu větu o reprezentaci lineárních funkcionálů v Hilbertově prostoru. Napište její důkaz, nebo alespoň jeho hlavní myšlenky.

\item Napište vlastnosti normy a vlastnosti metriky. Uveďte příklad metrického prostoru, jehož metrika není normou.

\item Vysvětlete co znamená, že množina $M$ je hustá v metrickém prostoru $X$. Je prostor spojitých funkcí $C((0,1))$ hustý v prostoru $L^2((0,1))$?
      Vysvětlete co tato otázka vlastně znamená pro funkce z $L^2$.

\item Ve které normě je prostor $C((-2,1))$ úplný, a ve které neúplný? Vysvětlete co to znamená? V případě neúplnosti uveďte příklad cauchyovské nekonvergentní posloupnosti.

\end{enumerate}

\pagebreak
\section{Písemka 21.11. verze B}
\subsection{Počítací příklady}
\begin{enumerate}
\item Ověřte zda množina polynomů $M=\{p(x)=ax^2+bx+c\where p'(-1)=c\}\subset \mathcal P^2$ je vektorovým podprostorem polynomů max. druhého stupně.
 Ověřte zda $\norm{p}=\abs{p(-1)}+\abs{p'(-1)}$ je normou na $M$.
 \item Vypočtěte $\norm{f}_{L^1([-2,2])}$ a $\norm{f}_{L^\infty([-1,2])}$ pro $f(x)=(x+1)(x^2-4x+4)$
 \item Určete bodovou limitu posloupnosti
 		\[f_n(x)=
                \left\{\begin{aligned}
                        n^{1/2} \text{ na $[0,\frac1n]$}\\
                        0 \text{ na $(\frac1n,1]$}
                       \end{aligned}\right. .
               \]
        Rozhodněte zda posloupnost konverguje  v normách $L^1{(0,1)}$ a $L^2{(0,1)}$

\subsection{Teorie}
\item Napište definici vektorového prostoru (7 vlastností). Uveďte dva příklady vektorových prostorů, z nichž alespoň jeden je prostor,
jehož prvky jsou funkce.

\item  Pomocí integrace per partes resp. Greenovy věty ověřte, že diferenciální operátor 
\[
A[f]=\big(-2\frac{\prtl^2 f}{\prtl x^2} -\frac{\prtl^2 f}{\prtl y^2} +  2f\big)
\]
je symetrický a pozitivně definitní, tj. 
\[
    (Af,g)_{L^2(\Omega)}=(f,Ag)_{L^2(\Omega)},\text{ a } (Af,f) \ge c(f,f),\ c>0
\]
na prostoru dvakrát spojitě diferencovatelných funkcí nulových na hranici:
\[
C^2_0(\Omega)=\{ f\in C^2(\close{\Omega}),\ f=0 \text{ na }\prtl\Omega\},\quad \Omega=(-1,1)\times(1,1).
\]
Na tomto prostoru použijte skalární součin prostoru $L^2(\Omega)$. Je tento operátor lineární? Proč?

\item Zformulujte Rieszovu větu o reprezentaci lineárních funkcionálů v Hilbertově prostoru. Napište její důkaz, nebo alespoň jeho hlavní myšlenky.

\item Napište vlastnosti normy a vlastnosti metriky. Uveďte příklad metrického prostoru, jehož metrika není normou.

\item Vysvětlete co znamená, že množina $M$ je hustá v metrickém prostoru $X$. Je prostor spojitých funkcí $C((0,1))$ hustý v prostoru $L^2((0,1))$? Vysvětlete co tato otázka vlastně znamená pro funkce z $L^2$.

\item Ve které normě je prostor $C((-2,1))$ úplný, a ve které neúplný? Vysvětlete co to znamená? V případě neúplnosti uveďte příklad cauchyovské nekonvergentní posloupnosti.
\end{enumerate}

\end{document}


