\documentclass[a4paper,10pt]{article}
%\usepackage[active]{srcltx}
\usepackage[czech]{babel}
\usepackage[utf8]{inputenc}

\usepackage{amsmath}
\usepackage{amsfonts}
\usepackage{amssymb}
\usepackage{amsthm}
%
\newtheorem{theorem}{Věta}[section]
\newtheorem{proposition}[theorem]{Tvrzení}
\newtheorem{definition}[theorem]{Definice}
\newtheorem{remark}[theorem]{Poznámka}
\newtheorem{lemma}[theorem]{Lemma}
\newtheorem{corollary}[theorem]{Důsledek}
\newtheorem{exercise}[theorem]{Cvičení}

%\numberwithin{equation}{document}
%
\def\div{{\rm div}}
\def\Lapl{\Delta}
\def\grad{\nabla}
\def\supp{{\rm supp}}
\def\dist{{\rm dist}}
%\def\chset{\mathbbm{1}}
\def\chset{1}
%
\def\Tr{{\rm Tr}}
\def\to{\rightarrow}
\def\weakto{\rightharpoonup}
\def\imbed{\hookrightarrow}
\def\cimbed{\subset\subset}
\def\range{{\mathcal R}}
\def\leprox{\lesssim}
\def\argdot{{\hspace{0.18em}\cdot\hspace{0.18em}}}
\def\Distr{{\mathcal D}}
\def\calK{{\mathcal K}}
\def\FromTo{|\rightarrow}
\def\convol{\star}
\def\impl{\Rightarrow}
\DeclareMathOperator*{\esslim}{esslim}
\DeclareMathOperator*{\esssup}{ess\,supp}
\DeclareMathOperator{\ess}{ess}
\DeclareMathOperator{\osc}{osc}
\DeclareMathOperator{\curl}{curl}
%
%\def\Ess{{\rm ess}}
%\def\Exp{{\rm exp}}
%\def\Implies{\Longrightarrow}
%\def\Equiv{\Longleftrightarrow}
% ****************************************** GENERAL MATH NOTATION
\def\Real{{\rm\bf R}}
\def\Rd{{{\rm\bf R}^{\rm 3}}}
\def\RN{{{\rm\bf R}^N}}
\def\D{{\mathbb D}}
\def\Nnum{{\mathbb N}}
\def\Measures{{\mathcal M}}
\def\d{\,{\rm d}}               % differential
\def\sdodt{\genfrac{}{}{}{1}{\rm d}{{\rm d}t}}
\def\dodt{\genfrac{}{}{}{}{\rm d}{{\rm d}t}}
%
\def\vc#1{\mathbf{\boldsymbol{#1}}}     % vector
\def\tn#1{{\mathbb{#1}}}    % tensor
\def\abs#1{\lvert#1\rvert}
\def\Abs#1{\bigl\lvert#1\bigr\rvert}
\def\bigabs#1{\bigl\lvert#1\bigr\rvert}
\def\Bigabs#1{\Big\lvert#1\Big\rvert}
\def\ABS#1{\left\lvert#1\right\rvert}
\def\norm#1{\bigl\Vert#1\bigr\Vert} %norm
\def\close#1{\overline{#1}}
\def\inter#1{#1^\circ}
\def\eqdef{\mathrel{\mathop:}=}     % defining equivalence
\def\where{\,|\,}                    % "where" separator in set's defs
\def\timeD#1{\dot{\overline{{#1}}}}
%
% ******************************************* USEFULL MACROS
\def\RomanEnum{\renewcommand{\labelenumi}{\rm (\roman{enumi})}}   % enumerate by roman numbers
\def\rf#1{(\ref{#1})}                                             % ref. shortcut
\def\prtl{\partial}                                        % partial deriv.
\def\Names#1{{\scshape #1}}
\def\rem#1{{\parskip=0cm\par!! {\sl\small #1} !!}}
\def\reseni#1{\par{\bf Řešení:}#1}
%
%
% ******************************************* DOCUMENT NOTATIONS
% document specific
%***************************************************************************
%
\addtolength{\textwidth}{2cm}
\addtolength{\vsize}{2cm}
\addtolength{\topmargin}{-1cm}
\addtolength{\hoffset}{-1cm}
\begin{document}
\parskip=2ex
\parindent=0pt
\pagestyle{empty}
\section*{Zápočtová písemka - verze A}
\begin{enumerate}
 \item 
   Nalezněte řešení počáteční úlohy:
   \[
    \vc y'=\begin{pmatrix}
          1 & 1 \\ -1 & 1
       \end{pmatrix} \vc y
       +
       \begin{pmatrix}
         \cos t \\ -\sin t
       \end{pmatrix}
    \qquad
    \vc y(0)=\begin{pmatrix}
           -1 \\ 1
         \end{pmatrix}
   \]
 \item 
    Nalezněte libovolnou iterační metodou (doporučuji Newtonovu) všechna řešení rovnice
    \[
       \cos x=-\frac14 x
    \]
    Načrtněte grafy funkcí na obou stranách rovnice. Určete kolik má rovnice celkem řešení.
    Napište vzorce pro výpočet iterací, odůvodněte volbu zvoleného počátečního přiblížení
    a nalezněte řešení s přesností na dvě platné cifry.
 \item
    Nalezněte libovolný fundamentální systém pro počáteční úlohu
    \[
    \vc y'=\begin{pmatrix}
          0 & -1 & 3 \\ 
          -4 & 0 & 10 \\
          -2 & -1 & 6
       \end{pmatrix} \vc y   
    \]
\end{enumerate}

\pagebreak

\section*{Zápočtová písemka - verze B}
\begin{enumerate}
 \item 
   Nalezněte řešení počáteční úlohy:
   \[
    \vc y'=\begin{pmatrix}
          0 & 1 \\ 1 & 0
       \end{pmatrix} \vc y
       +
       \begin{pmatrix}
         -2 e^t \\  t^2
       \end{pmatrix}
    \qquad
    \vc y(0)=\begin{pmatrix}
           -2 \\ 1
         \end{pmatrix}
   \]
 \item 
    Nalezněte libovolnou iterační metodou kladné řešení rovnice
    \[
       e^x=(x+1)^2.
    \]
    Načrtněte grafy funkcí na obou stranách rovnice. Určete kolik má rovnice celkem řešení.    Odůvodněte volbu zvoleného počátečního přiblížení. Napište vzorce pro výpočet iterací
    a nalezněte řešení s přesností na dvě platné cifry.
 \item
    Nalezněte libovolný fundamentální systém pro počáteční úlohu
    \[
    \vc y'=\begin{pmatrix}
          1 & -1 & 2 \\ 
          1 & 1 & -1 \\
          2 & 1 & 1
       \end{pmatrix} \vc y   
    \]
\end{enumerate}
\pagebreak
\section*{Zápočtová písemka - verze C}
\begin{enumerate}
 \item 
 \item 
   Nalezněte řešení počáteční úlohy:
   \[
    \vc y'=\begin{pmatrix}
          1 & 1 \\ -1 & 1
       \end{pmatrix} \vc y
       +
       \begin{pmatrix}
         \cos t \\ -\sin t
       \end{pmatrix}
    \qquad
    \vc y(0)=\begin{pmatrix}
           -1 \\ 1
         \end{pmatrix}
   \]
\item 
    Nalezněte libovolnou iterační metodou záporné řešení rovnice
    \[
       x=2\ln\abs{x+1}.
    \]
    Načrtněte grafy funkcí na obou stranách rovnice. Určete kolik má rovnice celkem řešení.
    Odůvodněte volbu zvoleného počátečního přiblížení. Napište vzorce pro výpočet iterací
    a nalezněte řešení s přesností na dvě platné cifry.
 \item
    Nalezněte libovolný fundamentální systém pro počáteční úlohu
    \[
    \vc y'=\begin{pmatrix}
          2 & 1 & 0 \\ 
          1 & 3 & -1 \\
          -1 & 2 & 3
       \end{pmatrix} \vc y   
    \]
   (jedno vlastní číslo je 2)
\end{enumerate}

\pagebreak
\section*{Zápočtová písemka - verze D}
\begin{enumerate}
 \item 
   Nalezněte řešení počáteční úlohy:
   \[
    \vc y'=\begin{pmatrix}
          0 & 1 \\ 1 & 0
       \end{pmatrix} \vc y
       +
       \begin{pmatrix}
         -2 e^t \\  t^2
       \end{pmatrix}
    \qquad
    \vc y(0)=\begin{pmatrix}
           -2 \\ 1
         \end{pmatrix}
   \]
 \item 
   Oblast $\Omega$ tvoří trojúhelník s vrcholy v bodech $[0,0]$, $[1.5, 0]$, $[0,3]$. Proveďte diskretizaci 
   oblasti metodou sítí s krokem $1/2$ (nakreslete obrázek). Na této oblasti řešte metodou sítí rovnici:
   \[
      -\frac{\prtl}{\prtl x}\big( (1+y^2)\frac{\prtl u}{\prtl x}\big) - 2 \frac{\prtl^2 u}{\prtl y^2} + yu= x/y
   \]
   s konstantní Dirichletovou okrajovou podmínkou $u = 1$.
   Označte regulární a neregulární uzly. V regulárních uzlech sestavte diskrétní formu rovnice. V neregulárních 
   uzlech napište rovnici lineární interpolace okrajové podmínky.
      
 \item
    Nalezněte libovolný fundamentální systém pro počáteční úlohu
    \[
    \vc y'=\begin{pmatrix}
          0 & -1 & 3 \\ 
          -4 & 0 & 10 \\
          -2 & -1 & 6
       \end{pmatrix} \vc y   
    \]
\end{enumerate}

\pagebreak
\section*{Zápočtová písemka - verze E}
\begin{enumerate}
 \item 
   Nalezněte řešení počáteční úlohy:
   \[
    \vc y'=\begin{pmatrix}
          1 & 1 \\ -1 & 1
       \end{pmatrix} \vc y
       +
       \begin{pmatrix}
         \cos t \\ -\sin t
       \end{pmatrix}
    \qquad
    \vc y(0)=\begin{pmatrix}
           -1 \\ 1
         \end{pmatrix}
   \]
 \item 
   Oblast $\Omega$ tvoří trojúhelník s vrcholy v bodech $[1,1]$, $[2.5, 1]$, $[1,4]$. Proveďte diskretizaci 
   oblasti metodou sítí s krokem $1/2$ (nakreslete obrázek). Na této oblasti řešte metodou sítí rovnici:
   \[
      -\frac{\prtl}{\prtl x}\big( (x^2+y)\frac{\prtl u}{\prtl x}\big) - 3 \frac{\prtl^2 u}{\prtl y^2} + yu= xy
   \]
   s konstantní Dirichletovou okrajovou podmínkou $u = 1$.
   Označte regulární a neregulární uzly. V regulárních uzlech sestavte diskrétní formu rovnice. V neregulárních 
   uzlech napište rovnici lineární interpolace okrajové podmínky.
      
 \item
    Nalezněte libovolný fundamentální systém pro počáteční úlohu
    \[
    \vc y'=\begin{pmatrix}
          1 & -1 & 2 \\ 
          1 & 1 & -1 \\
          2 & 1 & 1
       \end{pmatrix} \vc y   
    \]
\end{enumerate}
\pagebreak
\section*{Zápočtová písemka - verze F}
\begin{enumerate}
 \item 
   Nalezněte řešení počáteční úlohy:
   \[
    \vc y'=\begin{pmatrix}
          2 & 1 \\ -1 & 4
       \end{pmatrix} \vc y
       +
       \begin{pmatrix}
         2 e^t \\  1
       \end{pmatrix}
    \qquad
    \vc y(0)=\begin{pmatrix}
           -1 \\ -1
         \end{pmatrix}
   \]
 \item 
    Nalezněte libovolnou iterační metodou (doporučuji Newtonovu) všechna řešení rovnice
    \[
       \cos x=-\frac14 x
    \]
    Načrtněte grafy funkcí na obou stranách rovnice. Určete kolik má rovnice celkem řešení.
    Napište vzorce pro výpočet iterací, odůvodněte volbu zvoleného počátečního přiblížení
    a nalezněte řešení s přesností na dvě platné cifry.
 \item
    Nalezněte libovolný fundamentální systém pro počáteční úlohu
    \[
    \vc y'=\begin{pmatrix}
          2 & 1 & 0 \\ 
          1 & 3 & -1 \\
          -1 & 2 & 3
       \end{pmatrix} \vc y   
    \]
   (jedno vlastní číslo je 2)
\end{enumerate}

\end{document}

