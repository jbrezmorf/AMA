\chapter{Soustavy obyčejných diferenciálních rovnic}
%%%%%%%%%%%%%%%%%%%%%%%%%%%%%%%%%%%%%%%%%%%%%%%%%%%%%%%%%%%%%%%%%%%%%%%%%%%%%%%%%%%%%%%%%%%%%%%%%%%%
%%%%%%%%%%%%%%%%%%%%%%%%%%%%%%%%%%%%%%%%%%%%%%%%%%%%%%%%%%%%%%%%%%%%%%%%%%%%%%%%%%%%%%%%%%%%%%%%%%%%

Soustavou obyčejných diferenciálních rovnic prvního řádu rozumíme (vektorovou) rovnici ve tvaru
\[
   \vec{y}' = \vec{f}(x,\vec{y}),
\]
kde $\vec{f}:\mathbb{R}\times\mathbb{R}^n\rightarrow\mathbb{R}^n$, $\vec{y}=(y_1,\ldots,y_n)^T$, $n\in\mathbb{N}$.

Definujeme derivaci vektorové funkce vztahem $\vec{y}'=(y'_1,\ldots,y'_n)^T$.

\paragraph{Soustavy lineárních diferenciálních rovnic prvního řádu s konstantní maticí}
%%%%%%%%%%%%%%%%%%%%%%%%%%%%%%%%%%%%%%%%%%%%%%%%%%%%%%%%%%%%%%%%%%%%%%%%%%%%%%%%%%%%%%%
V této kapitole se budeme zabývat řešením soustavy lineárních diferenciálních rovnic prvního řádu
s konstantní maticí, která má tvar
\begin{equation}\label{eq:slodr}
   \vec{y}' = \vec{A}\vec{y} + \vec{b}(x), \; \vec{A}\in\mathbb{R}^{n,n}, \; \vec{b}:\,\mathbb{R}\rightarrow\mathbb{R}^n,
\end{equation}
kde $\vec{b}$ je spojitá vektorová fukce na $(a,b)$.
Tato soustava má na $(a,b)$ řešení,
které je jednoznačně určeno počáteční podmínkou $\vec{y}(x_0)=\vec{y}_0$, kde $x_0\in(a,b)$.

\paragraph{Homogenní soustavy s konstantními koeficienty}
%%%%%%%%%%%%%%%%%%%%%%%%%%%%%%%%%%%%%%%%%%%%%%%%%%%%%%%%%
Homogenní soustava příslušná soustavě~(\ref{eq:slodr}) má tvar
\begin{equation}\label{eq:hslodr}
   \vec{y}'=\vec{A}\vec{y}.
\end{equation}
Řešení~(\ref{eq:hslodr}) tvoří vektorový prostor.

Předpokládejme, že matice $\vec{A}$ je diagonalizovatelná, tj. má úplný systém vlastních vektorů.
Je-li $\lambda\in\mathbb{R}$ reálné vlastní číslo matice $\vec{A}$ a $\vec{u}\in\mathbb{R}^n$ 
je jemu odpovídající vlastní vektor, pak funkce
\[
   \vec{v}(x) = \vec{u}e^{\lambda x}
\]
je řešením rovnice~(\ref{eq:hslodr}).
Je-li $\lambda\in\mathbb{C}\setminus\mathbb{R}$ komplexní vlastní číslo matice $\vec{A}$
(a tedy rovněž $\bar{\lambda}$ je vlastní číslo $\vec{A}$)
a $\vec{u}\in\mathbb{C}^n$ je jemu odpovídající vlastní vektor (vlastnímu číslu $\bar{\lambda}$ odpovídá
vlastní vektor $\bar{\vec{u}}$), pak funkce
\[
   \vec{v}_1(x) = \vec{u}_1 e^{\sigma x}\cos\omega x - \vec{u}_2 e^{\sigma x}\sin\omega x,\;\;
   \vec{v}_2(x) = \vec{u}_1 e^{\sigma x}\sin\omega x + \vec{u}_2 e^{\sigma x}\cos\omega x
\]
jsou reálná řešení rovnice~(\ref{eq:hslodr}),
kde $\sigma=\Re\lambda$, $\omega=\Im\lambda$, $\vec{u}_1 = \Re\vec{u}$, $\vec{u}_2=\Im\vec{u}$.

Nechť $\vec{V}(x)$, $x\in\mathbb{R}$ je matice obsahující ve sloupcích funkce definované v předchozím odstavci
pro každé vlastní číslo matice $\vec{A}$ (příp. dvojici komplexně sdružených vlastních čísel)
-- tyto funkce tvoří fundamentální systém řešení soustavy~(\ref{eq:hslodr}).
Platí tedy
\[
   \vec{V}'(x) = \vec{A}\vec{V}(x), \; x\in\mathbb{R}.
\]
Je-li $\vec{T}\in\mathbb{R}^{n,n}$ libovolná regulární matice stupně $n$, pak 
(definujeme-li $\tilde{\vec{V}}(x)\equiv\vec{V}(x)\vec{T}$, $x\in\mathbb{R}$) rovněž platí
\[
   \tilde{\vec{V}}'(x) = \vec{A}\tilde{\vec{V}}(x), \; x\in\mathbb{R}.
\]
Matice $\vec{V}(x)$ je regulární matice (má lineárně nezávislé sloupce), a tedy
rovněž matice $\tilde{\vec{V}}(x)$.
Všechny takové matice nazýváme fundamentálními maticemi soustavy~(\ref{eq:hslodr})
a každé řešení soustavy~(\ref{eq:hslodr}) můžeme napsat jako lineární kombinaci sloupců
$\vec{V}(x)$, tj. 
\[
   \vec{y}(x) = \vec{V}(x)\vec{c}, \; x\in\mathbb{R},
\]
pro nějaké $\vec{c}\in\mathbb{R}^{n}$.

\begin{ex}\label{priklad:1}
   Najděte fundamentální matici soustavy
   \[
      \vec{y}'\equiv
      \begin{pmatrix}
         y_1\\y_2
      \end{pmatrix}'
      =
      \begin{pmatrix}
         0 & 1 \\ -2 & -3
      \end{pmatrix}
      \begin{pmatrix}
         y_1\\y_2
      \end{pmatrix}
      \equiv \vec{A}\vec{y}
   \]
\end{ex}
\begin{sol}
   Vlastní čísla matice $\vec{A}$ jsou $\lambda_1=-1$ a $\lambda_2=-2$
   s odpovídajícími vlastními vektory $\vec{u}_1=(-1,1)^T$ a $\vec{u}_2=(-1,2)^T$.
   Fundamentální systém soustavy tedy tvoří funkce
   \[
      \vec{v}_1(x) = \begin{pmatrix}
         -1 \\ 1
      \end{pmatrix}e^{-x}, \;\;
      \vec{v}_1(x) = \begin{pmatrix}
         -1 \\ 2
      \end{pmatrix}e^{-2x}.
   \]
   Fundamentální matice soustavy (resp. jedna z fundamentálních matic) je
   \[
      \vec{V}(x) = \begin{pmatrix}
         -e^{-x} &  -e^{-2x}\\
          e^{-x} &  2e^{-2x}
      \end{pmatrix}.
   \]
\end{sol}

\paragraph{Homogenní soustavy s konstantními koeficienty -- Cauchyho úloha}
%%%%%%%%%%%%%%%%%%%%%%%%%%%%%%%%%%%%%%%%%%%%%%%%%%%%%%%%%%%%%%%%%%%%%%%%%%%
Řešme Cauchyho úlohu pro soustavu~(\ref{eq:hslodr}), tj. soustavu s podmínkou $\vec{y}(x_0)=\vec{y}_0$.
Každé řešení soustavy~(\ref{eq:hslodr}) lze psát ve tvaru $\vec{y}(x)=\vec{V}(x)\vec{c}$,
kde $\vec{c}\in\mathbb{R}^n$.
Užitím počáteční podmínky máme $\vec{y}_0=\vec{V}(x_0)\vec{c}$, odkud $\vec{c} = \vec{V}^{-1}(x_0)\vec{y}_0$,
takže
\[
   \vec{y}(x) = \vec{U}(x)\vec{y}_0
\]
je řešením počáteční úlohy.
Matice $\vec{U}(x) = \vec{V}(x)\vec{V}^{-1}(x_0)$ se nazývá standardní fundamentální matice
příslušná Cauchyho úloze~(\ref{eq:hslodr}) s počáteční podmínkou $\vec{y}(x_0)=\vec{y}_0$.

\begin{ex}\label{priklad:2}
   Najděte standardní fundamentální matici soustavy
   \[
      \vec{y}'\equiv
      \begin{pmatrix}
         y_1\\y_2
      \end{pmatrix}'
      =
      \begin{pmatrix}
         0 & 1 \\ -2 & -3
      \end{pmatrix}
      \begin{pmatrix}
         y_1\\y_2
      \end{pmatrix}
      \equiv \vec{A}\vec{y}
   \]
   a najděte řešení soustavy s počáteční podmínkou $\vec{y}_0 = (1,1)^T$.
\end{ex}
\begin{sol}
   Jedna z fundamentálních matic soustavy je podle příkladu~(\ref{priklad:1})
   \[
      \vec{V}(x) = \begin{pmatrix}
         -e^{-x} &  -e^{-2x}\\
          e^{-x} &  2e^{-2x}
      \end{pmatrix}.
   \]
   Dále je
   \[
      \vec{V}(0) = \begin{pmatrix}
         -1 & -1\\
          1 &  2
      \end{pmatrix},
   \]
   odkud
   \[
      \vec{V}^{-1}(0) = \begin{pmatrix}
         -2 & -1\\
          1 &  1
      \end{pmatrix},
   \]
   takže
   \[
      \begin{split}
      \vec{U}(x) = \vec{V}(x)\vec{V}^{-1}(0) &= 
      \begin{pmatrix}
         -e^{-x} &  -e^{-2x}\\
          e^{-x} &  2e^{-2x}
      \end{pmatrix}
      \begin{pmatrix}
         -2 & -1\\
          1 &  1
      \end{pmatrix}\\
      &=
      \begin{pmatrix}
         2e^{-x}-e^{-2x} & e^{-x}-e^{-2x}\\
         -2e^{-x}+2e^{-2x} & -e^{-x}+2e^{-2x}
      \end{pmatrix}.
      \end{split}
   \]
   Řešením Cauchyho úlohy je tedy funkce
   \[
      \vec{y}(x) = \vec{U}(x)\vec{y}_0
      =
      \begin{pmatrix}
         3e^{-x}-2e^{-2x}\\
         -3e^{-x}+4e^{-2x}
      \end{pmatrix}.
   \]
\end{sol}

\begin{ex}
   Řešte soustavu
   \[
      \vec{y}'\equiv
      \begin{pmatrix}
         y_1\\y_2\\y_3
      \end{pmatrix}'
      =
      \begin{pmatrix}
         0 & 1 & 0\\ 0 & 0 & 1\\ 0 & -1 & 0
      \end{pmatrix}
      \begin{pmatrix}
         y_1\\y_2\\y_3
      \end{pmatrix}
      \equiv \vec{A}\vec{y}
   \]
   s počáteční podmínkou $\vec{y}_0=(1,1,1)^T$.
\end{ex}
\begin{sol}
   Vlastní čísla matice $\vec{A}$ jsou $\lambda_1=0$, $\lambda_2=\mathrm{i}$ a $\lambda_3=-\mathrm{i}$.
   Odpovídající vlastní vektory jsou
   \[
      \vec{u}_1 = \begin{pmatrix}1\\0\\0\end{pmatrix}, \;\;
      \vec{u}_2 = \begin{pmatrix}-1\\-i\\1\end{pmatrix}, \;\;
      \vec{u}_3 = \begin{pmatrix}-1\\i\\1\end{pmatrix}.
   \]
   Funkce fundamentálního systému jsou
   \[
      \vec{v}_1(x) = \begin{pmatrix}
         1 \\ 0 \\ 0
      \end{pmatrix},\;
      \vec{v}_2(x) = \begin{pmatrix}
         -\cos x \\ \sin x \\ \cos x
      \end{pmatrix},\;
      \vec{v}_3(x) = \begin{pmatrix}
         -\sin x \\ -\cos x \\ \sin x
      \end{pmatrix}.
   \]
   Fundamentální matice je
   \[
      \vec{V}(x) = \begin{pmatrix}
         1 & -\cos x & -\sin x\\
         0 & \sin x  & -\cos x\\
         0 & \cos x  & \sin x
      \end{pmatrix}.
   \]
   Odtud
   \[
      \vec{V}(0) = \begin{pmatrix}
         1 & -1 & 0\\
         0 & 0  &-1\\
         0 & 1  & 0
      \end{pmatrix},
   \]
   takže
   \[
      \vec{V}^{-1}(0) = \begin{pmatrix}
         1 & 0 & 1\\
         0 & 0  &1\\
         0 & -1  & 0
      \end{pmatrix},
   \]
   odkud dostaneme standardní fundamentální matici ve tvaru
   \[
      \begin{split}
      \vec{U}(x) = \vec{V}(x)\vec{V}^{-1}(0) &=
      \begin{pmatrix}
         1 & -\cos x & -\sin x\\
         0 & \sin x  & -\cos x\\
         0 & \cos x  & \sin x
      \end{pmatrix}
      \begin{pmatrix}
         1 & 0 & 1\\
         0 & 0  &1\\
         0 & -1  & 0
      \end{pmatrix}\\
      &=
      \begin{pmatrix}
         1 & \sin x  & 1-\cos x\\
         0 & \cos x  & \sin x\\
         0 & -\sin x & \cos x
      \end{pmatrix}.
      \end{split}
   \]
   Řešení Cauchyho úlohy je tedy
   \[
      \vec{y}(x) = \vec{U}(x)\vec{y}_0
      =
      \begin{pmatrix}
         2 + \sin x  -\cos x\\
         \cos x  + \sin x\\
         \cos x - \sin x
      \end{pmatrix}.
   \]
\end{sol}

\paragraph{Nehomogenní soustavy s konstantní maticí}
%%%%%%%%%%%%%%%%%%%%%%%%%%%%%%%%%%%%%%%%%%%%%%%%%%%%
Mějme (nehomogenní) soustavu lineárních diferenciálních rovnic prvního řádu s konstantní maticí ve tvaru
\begin{equation}\label{eq:nhslodr}
   \vec{y}' = \vec{A}\vec{y} + \vec{b}(x), \; \vec{A}\in\mathbb{R}^{n,n}, \; \vec{b}:\,\mathbb{R}\rightarrow\mathbb{R}^n,
\end{equation}
kde $\vec{b}$ je spojitá vektorová fukce na $(a,b)$.
Řešení nehomogenní rovnice pak můžeme vyjádřit jako součet partikulárního řešení 
(tj. libovolného řešení rovnice~(\ref{eq:nhslodr})) a řešení homogenní rovnice, tedy platí
\[
   \vec{y}(x) = \vec{y}_p(x) + \vec{U}(x)\vec{c}, \; x\in(a,b),
\]
kde $\vec{U}(x)$ je standardní fundamentální matice příslušné homogenní soustavy, $\vec{c}\in\mathbb{R}^n$ a
\[
   \vec{y}_p(x) = \int\limits_{x_0}^x\vec{U}(x-\xi)\vec{b}(\xi)\,d\xi.
\]

\begin{ex}
   Najděte řešení soustavy
   \[
      \vec{y}'\equiv
      \begin{pmatrix}
         y_1\\y_2
      \end{pmatrix}'
      =
      \begin{pmatrix}
         0 & 1 \\ -2 & -3
      \end{pmatrix}
      \begin{pmatrix}
         y_1\\y_2
      \end{pmatrix}
      +
      \begin{pmatrix}
         1 \\ 2
      \end{pmatrix}
      e^{x}
      \equiv \vec{A}\vec{y}
   \]
   a najděte řešení soustavy s počáteční podmínkou $\vec{y}_0 = (1,1)^T$.
\end{ex}
\begin{sol}
   Z příkladu~(\ref{priklad:2}) máme standardní fundamentální matici homogenní soustavy ve tvaru
   \[
      \vec{U}(x) = 
      \begin{pmatrix}
         -1 &  -1\\
         1 &  2
      \end{pmatrix}
      \begin{pmatrix}
         e^{-x} &  0\\
         0 &  e^{-2x}
      \end{pmatrix}
      \begin{pmatrix}
         -2 & -1\\
          1 &  1
      \end{pmatrix}.
   \]
   Odtud
   \[
      \begin{split}
      \vec{U}^{-1}(x) &= 
      \begin{pmatrix}
         -2 & -1\\
          1 &  1
      \end{pmatrix}^{-1}
      \begin{pmatrix}
         e^{-x} &  0\\
         0 &  e^{-2x}
      \end{pmatrix}^{-1}
      \begin{pmatrix}
         -1 &  -1\\
         1 &  2
      \end{pmatrix}^{-1}\\
      &=
      \begin{pmatrix}
         -1 &  -1\\
         1 &  2
      \end{pmatrix}
      \begin{pmatrix}
         e^{x} &  0\\
         0 &  e^{2x}
      \end{pmatrix}
      \begin{pmatrix}
         -2 & -1\\
          1 &  1
      \end{pmatrix}
      \end{split}
   \]
   takže
   \[
      \begin{split}
      \int\limits_{0}^x\vec{U}^{-1}(\xi)\vec{b}(\xi)\,d\xi
                   &=\int\limits_{0}^x
                      \begin{pmatrix}
                        2e^{\xi}-e^{2\xi} & e^{\xi}-e^{2\xi} \\ -2e^{\xi}+2e^{2\xi} & -e^{\xi}+2e^{2\xi}
                      \end{pmatrix}
                      \begin{pmatrix}
                         1 \\ 2
                      \end{pmatrix}
                      e^{\xi}
                      \,d\xi\\
                   &=\int\limits_{0}^x
                      \begin{pmatrix}
                         4e^{2\xi}-3e^{3\xi}\\
                         -4e^{2\xi}+6e^{3\xi}
                      \end{pmatrix}
                      \,d\xi
                   =\begin{pmatrix}
                      2e^{2x}-e^{3x}-1\\
                      -2e^{2x}+2e^{3x}
                   \end{pmatrix},
      \end{split}
   \]
   odkud
   \[
      \begin{split}
      \vec{y}_p(x) &= 
      \begin{pmatrix}
         -1 &  -1\\
         1 &  2
      \end{pmatrix}
      \begin{pmatrix}
         e^{-x} &  0\\
         0 &  e^{-2x}
      \end{pmatrix}
      \begin{pmatrix}
         -2 & -1\\
          1 &  1
      \end{pmatrix}
      \begin{pmatrix}
            2e^{2x}-e^{3x}-1\\
            -2e^{2x}+2e^{3x}
      \end{pmatrix}\\
      &=
      \begin{pmatrix}
         e^{x}-2e^{-x}+e^{-2x}\\
         2e^{-x}-2e^{-2x}
      \end{pmatrix}
      \end{split}
   \]
   Řešení Cauchyho úlohy je tedy
   \[
      \vec{y}(x) = \vec{y}_p(x) + \vec{U}(x)\vec{y}_0
       =\begin{pmatrix}
         e^{x}+e^{-x}-e^{-2x}\\
         -e^{-x}+2e^{-2x}
       \end{pmatrix}
   \]
\end{sol}



\paragraph{Putzerova metoda výpočtu fundamentální matice}
%%%%%%%%%%%%%%%%%%%%%%%%%%%%%%%%%%%%%%%%%%%%%%%%%%%%%%%%% 
Tato metoda je použitelná i v případě nediagonalizovatelné matice $\vec{A}$.
Nechť $\lambda_1,\ldots,\lambda_n\in\mathbb{C}$ jsou vlastní čísla matice $\vec{A}$ včetně násobnosti.
Definujeme matice
\[
   \begin{split}
   \vec{P}_0 = \vec{I},\;
   \vec{P}_1 = (\vec{A}-\lambda_1\vec{I})\vec{P}_0,\;
   \vec{P}_2 = (\vec{A}-\lambda_2\vec{I})\vec{P}_1,\;
   \ldots,\;
   \vec{P}_{n-1} = (A-\lambda_{n-1}I)\vec{P}_{n-2}.
   \end{split}
\]
Nalezněme funkce $q_i$, $i=1,\ldots,n$ jako řešení Cauchyho úloh
\begin{align*}
      q_1' &= \lambda_1q_1,  & q_1(0) &= 1,\\
      q_2' &= \lambda_2q_2+q_1,  & q_2(0) &= 0,\\
      &\cdots\\
      q_n' &= \lambda_nq_n+q_{n-1},  & q_n(0) &= 0.
\end{align*}
Matice
\[
  \vec{U}(x) = q_1(x)\vec{P}_0 + q_2(x)\vec{P}_1 + \cdots + q_n(x)\vec{P}_{n-1}
\]
je standardní fundamentální maticí soustavy~(\ref{eq:hslodr}).

\begin{ex}
   Najděte řešení soustavy
   \[
      \vec{y}'\equiv
      \begin{pmatrix}
         y_1\\y_2
      \end{pmatrix}'
      =
      \begin{pmatrix}
         7 & -18 \\ 3 & -8
      \end{pmatrix}
      \begin{pmatrix}
         y_1\\y_2
      \end{pmatrix}
      +
      \begin{pmatrix}
         12 \\ 5
      \end{pmatrix}
      e^{-x}
      \equiv \vec{A}\vec{y}+\vec{b}(x)
   \]
   s podmínkou $\vec{y}_0=(2,1)^T$.
\end{ex}
\begin{sol}
   Vlastní čísla matice $\vec{A}$ jsou $\lambda_1 = 1$ a $\lambda_2 = -2$.
   Položme
   \[
      \vec{P}_0 = \begin{pmatrix}
         1 & 0 \\ 0 & 1
      \end{pmatrix}, \;\;
      \vec{P}_1 = \begin{pmatrix}
         6 & -18 \\ 3 & -9
      \end{pmatrix}.
   \]
   Dále řešme rovnice $q'_1=q'_1$ s podmínkou $q_1(0)=1$; odtud $q_1(x)=e^{x}$;
   $q'_2=-2q_2+e^x$ s podmínkou $q_2(0)=0$; odtud $q_2(x)=\frac{1}{3}(e^x-e^{-2x})$.
   Odtud
   \[
      \vec{U}(x)=q_1(x)\vec{P}_0+q_2(x)\vec{P}_1 = 
      \begin{pmatrix}
         3 & -6 \\ 1 & -2
      \end{pmatrix}
      e^x
      +
      \begin{pmatrix}
         -2 & 6 \\ -1 & 3
      \end{pmatrix}
      e^{-2x},
   \]
   takže
   \[
     \begin{split}
        \vec{y}(x) &= \vec{y}_p + \vec{U}(x)\vec{y}_0\\
                   &= \int\limits_0^x\vec{U}(x-\xi)\vec{b}(\xi)+\vec{U}(x)\vec{y}_0\\
                   &= \int\limits_0^x\left(
                         \begin{pmatrix}3 & -6 \\ 1 & -2\end{pmatrix}e^{x-\xi}+\begin{pmatrix}-2 & 6 \\ -1 & 3\end{pmatrix}e^{-2x+2\xi}\right)
                         \begin{pmatrix}12 \\ 5\end{pmatrix}e^{-\xi}\,d\xi\\
                    &+\left(
                           \begin{pmatrix}
                              3 & -6 \\ 1 & -2
                           \end{pmatrix}
                           e^x
                           +
                           \begin{pmatrix}
                              -2 & 6 \\ -1 & 3
                           \end{pmatrix}
                           e^{-2x}
                        \right)
                        \begin{pmatrix}
                        2 \\ 1
                        \end{pmatrix}\\
                   &= \int\limits_0^x\left(
                         \begin{pmatrix}6\\2\end{pmatrix}e^{x-2\xi}+\begin{pmatrix}6\\3\end{pmatrix}e^{-2x+\xi}\right)
                         \,d\xi
                    +
                        \begin{pmatrix}
                        2 \\ 1
                        \end{pmatrix}
                        e^{-2x}\\
                   &= 
                         \begin{pmatrix}-3\\-1\end{pmatrix}e^{x}(e^{-2x}-1)+\begin{pmatrix}6\\3\end{pmatrix}e^{-2x}(e^x-1)
                    +
                        \begin{pmatrix}
                        2 \\ 1
                        \end{pmatrix}
                        e^{-2x}\\
                   &= 
                         \begin{pmatrix}-3\\-1\end{pmatrix}(e^{-x}-e^x)+\begin{pmatrix}6\\3\end{pmatrix}(e^{-x}-e^{-2x})
                    +
                        \begin{pmatrix}
                        2 \\ 1
                        \end{pmatrix}
                        e^{-2x}\\
                   &= 
                        \begin{pmatrix}
                        -4 \\ -2
                        \end{pmatrix}
                        e^{-2x}
                        +
                        \begin{pmatrix}
                        3 \\ 2
                        \end{pmatrix}
                        e^{-x}
                        +
                        \begin{pmatrix}
                        3 \\ 1
                        \end{pmatrix}
                        e^{x}.
     \end{split}
   \]
\end{sol}
      
      


\paragraph{Mocninná metoda výčtu fundamentální matice}
%%%%%%%%%%%%%%%%%%%%%%%%%%%%%%%%%%%%%%%%%%%%%%%%%%%%%%
Definujeme-li exponent matice $\vec{A}$ jako
\[
   e^{\vec{A}x} \equiv \sum\limits_{k=1}^{\infty}\frac{1}{k!}(\vec{A}x)^k,
\]
lze ukázat, že $e^{\vec{A}x}$ je standardní fundamentální matice~(\ref{eq:hslodr}).
Protože $p({A})=0$, kde $p$ je charakteristický polynom matice $\vec{A}$,
lze každou mocninu matice $\vec{A}^i$, $i\geq n$ vyjádřit jako lineární kombinaci
matic $\vec{I}, \; \vec{A},\;\ldots,\;\vec{A}^{n-1}$.
Matici $e^{\vec{A}x}$ lze tedy hledat ve tvaru
\[
   e^{\vec{A}x} = b_0(x)\vec{I}+b_1(x)\vec{A}+\cdots+b_{n-1}(x)\vec{A}^{n-1},
\]
kde $b_i$ ($i=0,\ldots,n-1$) jsou funkce $x$.
Je-li $\lambda_j$ $r$-násobné vlastní číslo matice $\vec{A}$,
nalezneme tyto funkce jako řešení soustav (přes všechna vlastní čísla matice $\vec{A}$)
\[
   \begin{split}
   b_0(x)+b_1(x)\lambda_j+\cdots+b_{n-1}(x)\lambda_j^{n-1}&=e^{\lambda_j x},\\
   b_1(x)+\cdots+(n-1)b_{n-1}(x)\lambda_j^{n-2}&=xe^{\lambda_j x},\\
     &\cdots\\   
   (r-1)!b_{r-1}+\cdots+\frac{(n-1)!}{(n-r-1)!}b_{n-1}(x)\lambda_j^{n-r-1}&=x^{r-1}e^{\lambda_jx}
   \end{split}
\]

\begin{ex}
   Najděte řešení soustavy
   \[
      \vec{y}'\equiv
      \begin{pmatrix}
         y_1\\y_2
      \end{pmatrix}'
      =
      \begin{pmatrix}
         1 & 1 \\ 0 & 1
      \end{pmatrix}
      \begin{pmatrix}
         y_1\\y_2
      \end{pmatrix}
      +
      \begin{pmatrix}
         1 \\ 2
      \end{pmatrix}
      e^{x}
      \equiv \vec{A}\vec{y}+\vec{b}(x)
   \]
   s podmínkou $\vec{y}_0=(1,1)^T$.
\end{ex}
\begin{sol}
   Matice $\vec{A}$ má dvojnásobné vlastní číslo $\lambda=1$.
   Standardní fundamentální matici hledáme ve tvaru
   \[
      \vec{U}(x) = b_0(x)\vec{I} + b_1(x)\vec{A},
   \]
   kde $b_0$ a $b_1$ jsou řešením soustavy
   \[
      \begin{split}
         b_0(x) + b_1(x) &= e^{x},\\
                  b_1(x) &= xe^{x}.
      \end{split}
   \]
   Odtud
   \[
      b_0(x) = (1-x)e^x, \; b_1(x) = xe^x
   \]
   a tedy
   \[
      \vec{U}(x) = \begin{pmatrix}1&x\\0&1\end{pmatrix}e^x.
   \]
   Nakonec
   \[
     \begin{split}
        \vec{y}(x) &= \vec{y}_p + \vec{U}(x)\vec{y}_0\\
                   &= \int\limits_0^x\vec{U}(x-\xi)\vec{b}(\xi)+\vec{U}(x)\vec{y}_0\\
                   &= \int\limits_0^x
                         \begin{pmatrix}1&x-\xi\\0&1\end{pmatrix}e^{x-\xi}
                         \begin{pmatrix}1 \\ 2\end{pmatrix}e^{\xi}\,d\xi
                    +\begin{pmatrix}1&x\\0&1\end{pmatrix}e^x
                        \begin{pmatrix}
                        1\\1
                        \end{pmatrix}\\
                   &= \int\limits_0^x
                         \begin{pmatrix}1+2x-2\xi\\2\end{pmatrix}e^{x}
                         \,d\xi
                    +\begin{pmatrix}1+x\\1\end{pmatrix}e^x\\
                   &= 
                     \begin{pmatrix}x^2+x\\2x\end{pmatrix}e^{x}
                    +\begin{pmatrix}1+x\\1\end{pmatrix}e^x
                   = 
                     \begin{pmatrix}x^2+2x+1\\2x+1\end{pmatrix}e^{x}.
     \end{split}
   \]
\end{sol}



\section{Příklady}

\begin{ex}
   Hledejte standardní fundamentální matice soustav $\vec{y}'=\vec{A}\vec{y}$, kde
   \begin{enumerate}
      \item
      \[
         \vec{A} = \begin{pmatrix}
            0 & 1\\ -1 & 0
         \end{pmatrix},
      \]
      \item
      \[
         \vec{A} = \begin{pmatrix}
            1 & 1 \\ 0 & 1
         \end{pmatrix},
      \]
      \item
      \[
         \vec{A} = \begin{pmatrix}0 & 1 \\ -2 & -3\end{pmatrix},
      \]
      \item
      \[
         \vec{A} = \begin{pmatrix}2 & 1 \\ 0 & 2\end{pmatrix},
      \]
      \item
      \[
         \vec{A} = \begin{pmatrix}1 & 2 \\ 2 & 1\end{pmatrix}.
      \]
   \end{enumerate}
\end{ex}
\begin{sol}
   \begin{enumerate}
      \item
      \item
      \item
      \[
         \vec{Y}_S(x) = \begin{pmatrix}2e^{-x}-e^{-2x}&e^{-x}-e^{-2x}\\-2e^{-x}+2e^{-2x}&-e^{-x}+2e^{-2x}\end{pmatrix}.
      \]
      \item
      \[
         \vec{Y}_S(x) = \begin{pmatrix}e^{2x}&xe^{2x}\\0&e^{2x}\end{pmatrix},
      \]
      \item
      \[
         \vec{Y}_S(x) = \frac{1}{2}\begin{pmatrix}e^{-x}+e^{3x}&-e^{-x}+e^{3x}\\-e^{-x}+e^{3x}&e^{-x}+e^{3x}\end{pmatrix}.
      \]
   \end{enumerate}
\end{sol}

\paragraph{Převod lineární diferenciální rovnice $n$-tého řádu s konstantními koeficienty
           na soustavu diferenciálních rovnic prvního řádu}
%%%%%%%%%%%%%%%%%%%%%%%%%%%%%%%%%%%%%%%%%%%%%%%%%%%%%%%%%%%%%%%%%%%%%%%%%%%%%%%%%%%%%%%%%

Uvažujme rovnici
\[
   y^{(n)} + a_{n-1}y^{(n-1)} + \cdots + a_1 y' + a_0 y = g(x), \; x \in (a,b).
\]
Položme 
\[
   y_1=y, \; y_2=y', \; y_3=y'', \ldots, y_{n-1} = y^{(n-2)}, \; y_{n} = y^{(n-1)}.
\]
Pak
\[
   \begin{split}
      y'_1 &= y_2,\\
      y'_2 &= y_3,\\
           &\cdots\\
      y'_{n-1} &= y'_n,\\
      y'_n &= g(x) - a_{n-1}y_n - \cdots - a_1 y_2 - a_0 y_1.
   \end{split}
\]
Původní rovnici jsme převedli na ekvivalentní soustavu
\[
   \begin{pmatrix}
      y_1 \\ y_2 \\\vdots \\ y_{n-1} \\ y_n
   \end{pmatrix}'
   =
   \begin{pmatrix}
         0 & 1 & & \\
         0 & & \ddots & & \\
         \vdots & & & \ddots & \\
         0 & \cdots & \cdots & \cdots &        1\\
      -a_0 & -a_1   & \cdots & \cdots & -a_{n-1}
   \end{pmatrix}
   \begin{pmatrix}
      y_1 \\ y_2 \\ \vdots \\ y_{n-1} \\ y_n
   \end{pmatrix}
   +
   \begin{pmatrix}
      0 \\ 0 \\ \vdots \\ 0 \\ g(x)
   \end{pmatrix}.
\]

\begin{ex}
   Rovnici $y''-y=x^2e^x$ převeďte na soustavu rovnic prvního řádu.
\end{ex}
\begin{sol}
   Položíme
   \[
      y_1 = y, \; y_2 = y'.
   \]
   Potom je
   \[
      \begin{split}
         y_1' &= y_2\\
         y_2' &= x^2e^x+y_1.
      \end{split}
   \]
\end{sol}
