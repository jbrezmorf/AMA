\chapter{Metrické prostory}
%%%%%%%%%%%%%%%%%%%%%%%%%%%%%%%%%%%%%%%%%%%%%%%%%%%%%%%%%%%%%%%%%%%%%%%%%%%%%%%%

Metrické prostory jsou speciální typy množin, na nichž lze definovat tzv.
metriku, resp. vzdálenost, pro každou dvojici prvků z této množiny.
Formálně: nechť $X\neq\emptyset$, $\varrho:\,X\times X\rightarrow\mathbb{R}$,
zobrazení­ $\varrho$ nechť má následující­ vlastnosti:
\begin{enumerate}
\renewcommand{\theenumi}{M\arabic{enumi}}
   \item $(\forall x,y\in X)(\varrho(x,y)\geq 0\;\&\;\varrho(x,y)=0\; 
                 \Leftrightarrow\;x=y)$,\label{item:m1}
   \item $(\forall x,y\in X)(\varrho(x,y)=\varrho(y,x))$ (symetrie),
   \item $(\forall x,y,z\in X))(\varrho(x,z)\leq\varrho(x,y)+\varrho(y,z))$  
          (trojúhelní­ková nerovnost).\label{item:m3}
\end{enumerate}
Zobrazení $\varrho$ potom říkáme metrika (na $X$), dvojici $(X,\varrho)$ 
nazýváme metrickým prostorem.
Často se metrický prostor zastupuje pouze symbol $X$, je-li zřejmé, jakou
metriku na něm máme definovanou.
Je-li $Y\subset X$, můžeme na $Y$ použít metriku definovanou na $X$.
Dvojici $(Y,\varrho)$ říkáme metrický podprostor prostoru $(X,\varrho)$.

Pojem metrického prostoru umožňuje zobecnit pojmy známé z analýzy
jako spojitost, konvergence, jako zobecnění otevřených a uzavřených intervalů
můžeme definovat otevřené a uzavřené množiny.

Symbolem $B_r(x)$ budeme označovat otevřenou kouli se středem v bodě $x$
a poloměrem $r$, tj. $B_r(x)\equiv\{y\in X\,|\,\varrho(x,y)<r\}$
($r$-okolí bodu $x$).
Záleží-li na použité metrice, označujeme ji $B_r^{\varrho}(x)$.

\begin{ex}
   Příklady metrických prostorů:
   \begin{enumerate}
      \item Číselné množiny ($\mathbb{R}$ reálných, $\mathbb{C}$ racionálních,
            $\mathbb{Q}$ racionálních čísel) je metrický prostor s metrikou 
            definovanou pomocí absolutní hodnoty $|\cdot|$.
            V dalším textu budeme uvažovat zpravidla metrický prostor reálných čísel.
      \item Na množině reálných $n$-tic, $n\in\mathbb{N}$
            (aritmetických vektorů) lze definovat
            např. metriky:
            \[
               \begin{split}
                  \varrho_1(x,y)&\equiv
                           \sum\limits_{i=1}^n|\beta_i-\alpha_i|,\\
                  \varrho_2(x,y)&\equiv
                           \sqrt{\sum\limits_{i=1}^n(\beta_i-\alpha_i)^2},\\
                  \varrho_{\infty}(x,y)&\equiv
                           \max\limits_{i=1,\ldots,n}|\beta_i-\alpha_i|,
               \end{split}
            \]
            kde $x=(\alpha_1,\ldots,\alpha_n),
            y=(\beta_1,\ldots,\beta_n)\in\mathbb{R}^n$.
       \item Analogicky jako na množině aritmetických vektorů
             lze definovat metriky na množině $C(I)$ spojitých funkcí 
             na uzavřeném intervalu $I\subset\mathbb{R}$
            \[
               \begin{split}
                  \varrho_1(f,g)&\equiv
                           \int\limits_I|g(x)-f(x)|\,dx,\\
                  \varrho_2(f,g)&\equiv
                           \sqrt{\int\limits_I(g(x)-f(x))^2\,dx},\\
                  \varrho_{\infty}(f,g)&\equiv
                           \max\limits_{x\in I}|g(x)-f(x)|
               \end{split}
            \]
            pro $f,g\in C(I)$.
        \item Nechť $X$ je neprázdná množina, definujme
              $\varrho(x,y)=0$ pro $x=y$ a $\varrho(x,y)=1$ pro $x\neq y$.
              Prostor $(X,\varrho)$ je (diskrétní) metrický prostor.
   \end{enumerate}
\end{ex}

\section{Množiny v metrických prostorech}

Nechť $M,N\subset X$ jsou podmnožiny metrického prostoru $X$.
Definujeme
\begin{itemize}
   \item sjednocení: $M\cup N\equiv\{x\in X\,|\,x\in M\;\vee\;x\in N\}$;
   \item průnik: $M\cap N\equiv\{x\in X\,|\,x\in M\;\&\;x\in N\}$;
   \item rozdíl: $M\setminus N\equiv\{x\in X\,|\,x\in M\;\&\;x\not\in N\}$;
   \item doplněk: $M^c\equiv\{x\in X\,|\,x\not\in M\}$;
\end{itemize}
zřejmě platí $M\setminus N=M\cap N^c$, $M^c = X\setminus M$.

Nechť $M\subset X$ je podmnožina metrického prostoru $X$ a $x\in X$.
Říkáme, že $x$ je
\begin{itemize}
   \item vnitřním bodem množiny $M$ 
         $\Leftrightarrow\;(\exists r>0)(B_r(x)\subset M)$;
   \item hraničním bodem množiny $M$
         $\Leftrightarrow\;(\forall r>0)(B_r(x)\cap M\neq\emptyset
           \;\&\;B_r(x)\cap M^c\neq\emptyset)$;
   \item hromadným bodem množiny $M$
         $\Leftrightarrow\;(\forall r>0)((B_r(x)\setminus\{x\})\cap M\neq\emptyset)$;
   \item izolovaným bodem množiny $M$
         $\Leftrightarrow\;(\exists r>0)(B_r(x)\cap M=\{x\})$;
\end{itemize}
Definujme: vnitřek množiny $M$, $M^0$, jako množinu všech vnitřních bodů;
           hranici množiny $M$, $M'$, jako množinu všech hraničních bodů;
           uzávěr množiny $M$, $\overline{M}\equiv M\cup\partial M$.

Zřejmě: každý vnitřní bod množiny leží v této množině  
        a každý bod množiny leží v jejím uzávěru ($M^0\subset M\subset\overline{M}$);
        hraniční body a hromadné body nemusí nutně ležet v dané množině;
        každý hraniční bod je hromadným bodem množiny;
        izolovaný bod je hraniční bod množiny a zároveň v ní leží;
        v každém okolí hromadného bodu množiny leží nekonečně mnoho bodů této množiny,
        v případě izolovaného bodu platí opak;

Množina $M\subset X$ je
\begin{itemize}
   \item otevřená $\Leftrightarrow \; M=M^0$,
   \item uzavřená $\Leftrightarrow \; M=\overline{M}$.
\end{itemize}
Platí: $M$ je otevřená $\Leftrightarrow$ $M^c$ je uzavřená
       a naopak
       (důsledky de Morganových zákonů:
        $M\cup N=(M^c\cap N^c)^c$, $M\cap N=(M^c\cup N^c)^c$).

\section{Konvergence v metrickém prostoru}

O prvku $x\in X$ řekneme, že je limitou posloupnosti $\{x_n\}\subset X$,
pokud $\varrho(x_n,x)\rightarrow 0$ pro $n\rightarrow\infty$,
tj. kovergenci prvků metrického prostoru převedeme na konvergenci reálné posloupnosti.
Definici můžeme formalizovat následovně:
výrok $\varrho(x_n,x)\rightarrow 0$ pro $n\rightarrow\infty$
je ekvivalentní výroku
$(\forall \varepsilon>0)(\exists N\in\mathbb{N})(\forall n\in\mathbb{N})
(n>N\;\Rightarrow\;\varrho(x_n,x)<\varepsilon)$.

Z jednoznačnosti limity reálné posloupnosti a vlastnosti~\ref{item:m1}
metriky plyne jednoznačnost limity posloupnosti v metrickém prostoru.

Můžeme-li na množině $X$ definovat více metrik, např. $\varrho_1$ a $\varrho_2$,
nemusí konvergence v jedné metrice být vždy ekvivalentní konvergenci v druhé metrice.
Toto platí jen pro tzv. ekvivalentní metriky, tj. metriky, pro něž existují
kladné konstanty $\alpha$ a $\beta$ tak, že 
$\alpha\varrho_1(x,y)\leq\varrho_2(x,y)\leq\beta\varrho_1(x,y)$.
Metriky $\varrho_1$ a $\varrho_2$ na $X$ jsou ekvivalentní, právě když
$(\forall x\in X)(\forall r>0)(\exists r_1<r)(\exists r_2>r)
 (B_{r_1}^{\varrho_1}\subset B_r^{\varrho_2}\subset B_{r_2}^{\varrho_1})$.

Množina $M$ je uzavřená, pokud každá konvergentní posloupnost z $M$
má limitu v $M$.

\section{Úplné metrické prostory}

$\{x_n\}\subset X$ je cauchyovská $\Leftrightarrow$
$(\forall\varepsilon>0)(\exists N\in\mathbb{N})(\forall m,n\in\mathbb{N})
(m,n>N\;\Rightarrow\;\varrho(x_m,x_n)<\varepsilon)$.
Metrický prostor $(X,\varrho)$ nazveme úplný, pokud každá cauchyovská posloupnost
$\{x_n\}\in X$ konverguje k prvku $x\in X$ v metrice $\varrho$.

Každá konvergentní posloupnost je nutně cauchyovská.

Uzavřený metrický podprostor (uzavřená podmnožina metrického prostoru
s metrikou daného prostoru) úplného metrického prostoru je úplný.

\begin{ex} Příklady úplných a neúplných metrických prostorů:
   \begin{enumerate}
   \item Příkladem neúplného metrického prostoru je $(\mathbb{Q},|\cdot|)$.
         Víme, že posloupnost $x_n=(1+1/n)^n$, $n\in\mathbb{N}$,
         konverguje k číslu $e=\exp(1)\not\in\mathbb{Q}$.
   \item Metrický prostor $(\mathbb{R},|\cdot|)$ je úplný.
   \item Metrické prostory 
         $(\mathbb{R}^n,\varrho_1)$, $(\mathbb{R}^n,\varrho_2)$, 
         $(\mathbb{R}^n,\varrho_{\infty})$ jsou úplné.
   \item Metrický prostor $(C(I),\varrho_{\infty})$ je úplný,
         prostory $(C(I),\varrho_{1})$ a $(C(I),\varrho_{2})$ nikoliv.
         Obecně úplnost metrického prostoru závisí na použité metrice
         (viz diskuze o ekvivalentních metrikách výše).
   \item Diskrétní metrický prostor $X$ je úplný
         (cauchyovské posloupnosti mají tvar $(x,x,\ldots,x,\ldots)$
         a konvergují k $x$, kde $x\in X$).
   \end{enumerate}
\end{ex}

\section{Operátory na metrických prostorech}

Nechť $X$, $Y$ jsou neprázdné množiny.
Je-li pro každé $x\in X$ přiřazeno $y\in Y$, zapisujeme $y=Ax$ ($x\in X$, $y\in Y$),
říkáme, že je definován operátor z $X$ do $Y$ (značíme $A:\,X\rightarrow Y$).
Někdy se uvažuje operátor definovaný jen pro jistou podmnožinu množiny $X$
(které říkáme definiční obor operátoru $A$). Tomu se zde vyhneme.
Je-li $y=Ax$ ($x\in X$, $y\in Y$), říkáme prvku $x$ vzor prvku $y$,
prvku $y$ říkáme obraz prvku $x$.
Množina $R(A)\equiv\{y\in Y\,|\,(\exists x\in X)(y=Ax)\}\subset Y$
se nazývá obor hodnot operátoru $A$;
operátor $A:\,X\rightarrow Y$ je spojitý, pokud pro každou posloupnost
$\{x_n\}\subset X$ platí implikace $x_n\rightarrow x\;\Rightarrow\;Ax_n\rightarrow Ax$
(tj. pokud převádí konvergentní posloupnosti v konvergentní).

Každý spojitý operátor je omezený, tj. převádí omezené množiny na omezené.
Množinu $M$ nazveme omezenou, je-li 
$\mathrm{diam}(M)=\sup_{x,y\in M}\varrho(x,y)<\infty$.
Ekvivalentně: je-li operátor neomezený, není spojitý.

\begin{ex}
   Příklad nespojitého (lineárního) operátoru:
   Uvažujme prostor $C[0,1]$ spojitých funkcí na intervalu $[0,1]$
   a jeho podprostor $C^1[0,1]\equiv\{f\in C[0,1]\,|\,f'\in C[0,1]\}$,
   oba s metrikou $\varrho_{\infty}$.
   Definujme $A\,:f\mapsto f'\,:C^1[0,1]\rightarrow C[0,1]$, $f\in C^1[0,1]$.
   Uvažujme $f_n:\,x\mapsto x^n$ pro $n\in\mathbb{N}$.
   Zvolme $n\in\mathbb{N}$,
   pak $\varrho_{\infty}(f_n,f_1)=\max_{x\in[0,1]}|x-x^n|=
   \max_{x\in[0,1]}|x(1-x^{n-1})|\leq 1$.
   Množina $\{f_n\}$ je tedy omezená.
   Na druhou stranu $\varrho_{\infty}(Af_n,Af_1)=\varrho(nx^{n-1},0)=
   \max_{x\in[0,1]}|nx^n-1|=n$ je neomezená.
   Operátor $A$ je neomezený, a tedy není spojitý.
\end{ex}

\begin{ex}
   Příklad spojitého operátoru:
   Uvažujme opět prostor $C[0,1]$
   a definujme operátor $A:\,f\mapsto\int_0^1f(x)\,dx:
   \,C[0,1]\rightarrow\mathbb{R}$, $f\in C[0,1]$.
   Použijeme-li např. metriku $\varrho_{\infty}$,
   snadno se ukáže, že operátor $A$ převádí konvergentní posloupnosti
   na konvergentní.
\end{ex}

\section{Věta o pevném bodě a její aplikace}

Nechť $(X,\varrho)$ je metrický prostor a $A:\,X\rightarrow X$ je operátor na $X$.
Prvku $x\in X$ říkáme pevný bod operátoru $A$, pokud $Ax=x$.
Operátor $A$ nazýváme kontrakce (kontraktivní operátor), pokud existuje
$0\leq \alpha <1$ tak, že $\varrho(Ax,Ay)\leq\varrho(x,y)$ pro každé $x,y\in X$.

Každá kontrakce je spojitá.

Banachova věta o pevném bodě:
Každá kontrakce definovaná na úplném metrickém prostoru má právě jeden pevný bod.

Důkaz: Nechť $x_0\in X$ a definujme rekurzivně 
$x_n=Ax_{n-1}=A^2x_{n-2}=\cdots=A^nx_0$ pro $n\in\mathbb{N}$.
Nechť $m,n\in\mathbb{N}$, $m\leq n$, pak
\[
   \begin{split}
      \varrho(x_m,x_n)&=\varrho(A^mx_0,A^nx_0)\leq\alpha^m\varrho(x_0,x_{n-m})\\
                      &\leq\alpha^m(\varrho(x_0,x_1)+\varrho(x_1,x_2)+\cdots
                       +\varrho(x_{n-m-1},x_{n-m})\\
                      &=\alpha^m(\varrho(x_0,x_1)+\varrho(Ax_0,Ax_1)+\cdots
                       +\varrho(A^{n-m-1}x_0,A^{n-m-1}x_1)\\
                      &\leq\alpha^m\varrho(x_0,x_1)(1+\alpha+\cdots
                       +\alpha^{n-m-1})\\
                      &\leq\frac{\alpha^m}{1-\alpha}\varrho(x_0,x_1)
   \end{split}
\]
Přitom jsme využili trojúhelníkové nerovnosti~\ref{item:m3}
a součtu nekonečné geometrické řady s kvocientem $\alpha<1$.
Pro zvolené $\varepsilon>0$ a dostatečně velká $m$ je $\varrho(x_m,x_n)<\varepsilon$,
takže posloupnost $\{x_n\}$ je cauchyovská a v důsledku úplnosti $(X,\varrho)$
má limitu $x=\lim_{n\rightarrow\infty}x_n\in X$.
Protože $x_n\rightarrow x$, plyne ze spojitosti $A$, že
$Ax=A\lim_{n\rightarrow\infty}x_n=\lim_{n\rightarrow\infty}Ax_n
=\lim_{n\rightarrow\infty}x_{n+1}=x$.
Našli jsme tedy pevný bod $x$ kontrakce $A$ včetně konstruktivního návodu,
jak takový bod nalézt včetně odhadu rychlosti konvergence.
Jsou-li dále $x,y\in X$ dva pevné body kontrakce $A$, máme
$\varrho(x,y)=\varrho(Ax,Ay)\leq\alpha\varrho(x,y)$.
Toto je ale možné jen případě, že $\varrho(x,y)$, tj. $x=y$.
Tím jsme dokázali jednoznačnost pevného bodu.

\paragraph{Aplikace věty o pevném bodě při numerickém řešení nelineárních rovnic.}
Řešme rovnici 
\begin{equation}\label{eq:nelin}
   f(x)=0
\end{equation}
kde $f$ je reálná funkce reálné proměnné $x$.
Jednou z metod pro její řešení je metoda pevného bodu.
Přepišme rovnici~(\ref{eq:nelin}) do tvaru
\[
   x = g(x)
\]
($g$ se nazývá iterační funkce).
Zvolme $x_0\in I$ a definujme
\begin{equation}\label{eq:iter}
   x_n = g(x_{n-1}), \; n\in\mathbb{N}.
\end{equation}
Předpokládejme, že existuje uzavřený interval $I$ tak, že $g(I)\subset I$,
iterační funkce je diferencovatelná na $I$ a platí $|g'(x)|\leq 1$ pro $x\in I$
a nějakou konstantu $0\leq K<1$.
Pak má~(\ref{eq:nelin}) řešení $x^*\in I$ a posloupnost $\{x_n\}$ konverguje k $x^*$.

Důkaz: Dvojice $(I,|\cdot|)$ je uzavřený metrický podprostor $(\mathbb{R},|\cdot|)$,
je tedy je úplný. Nechť $x,y\in I$, $x<y$. Pak
\[
   \left|\frac{g(y) - g(x)}{y-x}\right| = |g'(c)| \leq K,
\]
kde $c\in(x,y)$ (z věty o střední hodnotě).
Odtud $|g(y) - g(x)| \leq K|y-x|$ pro každé $x,y\in I$,
a tedy $g$ je kontrakce na $I$.
Z věty o pevném bodě tedy plyne, že existuje $x^*\in I$ takové,
že $x^*=g(x^*)$ a z důkazu této věty plyne $x^n\rightarrow x^*$.

Zpravidla se uvádí slabší formulace, kde předpokládáme, že $g$
je spojitě diferencovatelná na nějakém otevřeném okolí pevného bodu $x^*$
a platí na něm $|g'(x)|\leq K<1$. Zvolíme-li $x_0$ v dostatečně malém okolí
bodu $x^*$, konverguje posloupnost definovaná v~(\ref{eq:iter}) k $x^*$.

\begin{ex}
   Hledejme kořen rovnice $f(x)=x^2-x-2=0$.
   Zvolme $g(x)=\sqrt{x+2}$. 
   Možných voleb funkce $g$ je několik,
   ne všechny vedou ke konvergenci iterací~(\ref{eq:iter}).
   Volbou kladného znaménka odmocniny jsme redukovali množinu pevných bodů
   ze dvou na jeden.
   Zvolme $I=[0,7]$. Pak $g(0)\sqrt{2}\geq 0$ a $g(7)=3\leq 7$,
   je tedy $g(I)\subset I$.
   Derivací $g$ podle $x$ máme $g'(x)=1/(2\sqrt{x+2})$.
   Platí $g'(x)<1$ pro každé $x>-7/4$, tedy i pro $x\in I$.
   Odtud posloupnost iterací~(\ref{eq:iter}) konverguje ke kořenu
   rovnice $f(x)=0$ na $I$, tj. k $x^*=2$.
   Volíme-li např. $x_0=7$, máme
   $x_1=7.0000$, $x_2=3.0000$, $x_3=2.2361$, $x_4=2.0582$,
   $x_5=2.0145$, $x_6=2.0036$, $x_7=2.0009$ atd.
\end{ex}

\paragraph{Aplikace věty o pevném bodě v teorii diferenciálních rovnic.}
Nechť $G$ je otevřená souvislá množina (oblast) v $\mathbb{R}^2$,
$(x_0,y_0)\in G$ a nechť $f$ je spojitá na $G$ 
a splňuje Lipschitzovu podmínku v argumentu $y$
($f$ je lipschitzovská, lipschitzovsky spojitá)
$|f(x,y_1)-f(x,y_2)|\leq M|y_1-y_2|$ pro každé $(x,y_1)\in G$, $(x,y_2)\in G$,
kde $M>0$ je pevná konstanta (nezávislá na $x$, $y_1$, $y_2$,
tzv. Lipschitzova konstanta).
Pak existuje $\delta > 0$ tak, že rovnice $y'(x) = f(x,y)$ má právě jedno řešení
$y(x)=\varphi(x)$ na intervalu $[x_0-\delta,x_0+\delta]$ splňující
počáteční podmínku $\varphi(x_0)=y_0$.

Důkaz: Řešení rovnice $y'(x) = f(x,y)$ s počáteční podmínkou $\varphi(x_0)=y_0$
můžeme zapsat ve tvaru
\begin{equation}\label{eq:int_eq}
   \varphi(x) = y_0 + \int\limits_{x_0}^x f(x,\varphi(t))\,dt.
\end{equation}
Funkce $f$ je na $G$ spojitá (poznamenejme, že na funkci $f$
se můžeme dívat jako na operátor $f\,\mathbb{R}^2\rightarrow\mathbb{R}$),
tedy je omezená, tj. existuje $K>0$ tak, že $|f(x,y)|\leq K$ na nějaké
podoblasti $G'\subset G$ takové, že $(x_0,y_0)\in G'$.
Vyberme $\delta>0$ tak, že oblast $G'$ obsahuje obdélník
$R\equiv [x_0-\delta,x_0+\delta]\times [y_0-K\delta,y_0+K\delta]$
a zároveň $\delta\leq 1/M$.
Označme $C^*\equiv\{\psi\in C[x_0-\delta,x_0+\delta]\,|\,
(\forall x\in[x_0-\delta,x_0+\delta])(|\varphi(x)-y_0|\leq K\delta)\}$
a použijme metriku $\varrho_{\infty}$.
Metrický prostor $(C^*,\varrho_{\infty})$ je uzavřený podprostor prostoru
$(C[x_0-\delta,x_0+\delta],\varrho_{\infty})$, který je úplný, a tedy
$(C^*,\varrho_{\infty})$ je rovněž úplný.
Definujme operátor $\psi=A\phi$, $\phi\in C^*$ předpisem
\[
   \psi(x)=(A\varphi)(x)\equiv y_0+\int\limits_{x_0}^x f(t,\varphi(t))\,dt.
\]
Ukážeme, že je $A:\,C^*\rightarrow C^*$ a že $A$ je kontrakce.
Abychom ověřili, že $A$ je operátor do $C^*$, musíme ověřit podmínku
$(\forall x\in[x_0-\delta,x_0+\delta])(|(A\varphi)(x)-y_0|\leq K\delta)$.
Platí
\[
   |(A\varphi)(x)-y_0|=\left|\int\limits_{x_0}^xf(t,\varphi(t))\,dt\right|
                      \leq\int\limits_{x_0}^x|f(t,\varphi(t))|\,dt
                      \leq K|x-x_0|\leq K\delta.
\]
Abychom ověřili kontraktivnost operátoru, nechť $\varphi,\psi\in C^*$, 
$x\in[x_0-\delta,x_0+\delta]$, pak
\[
  \begin{split}
   |(A\varphi)(x)-(A\psi)(x)|&=
    \left|\int\limits_{x_0}^xf(t,\varphi(t))-f(t,\psi(t)\,dt\right|\\
    &\leq\int\limits_{x_0}^x|f(t,\varphi(t))-f(t,\psi(t)|\,dt\\
    &\leq M\int\limits_{x_0}^x|\varphi(t)-\psi(t)|\,dt
    \leq M\delta\,\varrho_{\infty}(\varphi,\psi),
  \end{split}
\]
odkud $\varrho_{\infty}(A\varphi,A\psi)\leq M\delta\varrho(\varphi,\psi)
=\alpha\varrho(\varphi,\psi)$, kde $\alpha\equiv M\delta<1$.
Z věty o pevném bodě existuje tedy $\varphi\in C^*$ tak, že
$A\varphi=\varphi$, tj. existuje právě jedno řešení integrální 
rovnice~(\ref{eq:int_eq}) v $C^*$.

