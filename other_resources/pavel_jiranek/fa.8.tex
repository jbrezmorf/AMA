\chapter{Vývar z funkcionální analýzy}
%%%%%%%%%%%%%%%%%%%%%%%%%%%%%%%%%%%%%%

\section{Lineární prostory}
%%%%%%%%%%%%%%%%%%%%%%%%%%%

Nechť $\mathbb{R}$ je těleso reálných čísel.
Neprázdnou množinu $V$ nazveme \emph{vektorovým prostorem} (prvky nazýváme vektory)
nad tělesem $\mathbb{R}$, je-li definována operace sčítání 
$+:V\times V\rightarrow V$ a operace násobení skalárem 
$\cdot:\mathbb{R}\times V\rightarrow V$ tak, že:
\begin{enumerate}
   \item $(\forall x,y\in V)(x+y=y+x)$ -- komutativita $+$,
   \item $(\forall x,y,z\in V)((x+y)+z=x+(y+z))$ -- asociativita $+$,
   \item $(\exists 0\in V)(\forall x\in V)(0+x=x)$ -- existence nulového prvku 
           (neutrální prvek vzhledem ke sčítání),
   \item $(\forall x\in V)(\exists (-x)\in V)(x+(-x)=0)$ -- existence opačného prvku,
   \item $(\forall \alpha,\beta\in\mathbb{R})(\forall x\in V)
          (\alpha(\beta x)=(\alpha\beta x))$,
   \item $(\forall x\in V)(1\cdot x = x)$,
   \item $(\forall \alpha,\beta\in\mathbb{R})(\forall x\in V)
         ((\alpha+\beta)x=\alpha x+\beta x)$ -- distributivita I,
   \item $(\forall \alpha\in\mathbb{R})(\forall x,y\in V)
         (\alpha(x+y)=\alpha x+\beta y)$ -- distributivita II.
\end{enumerate}
Pro $x,y\in V$ definujeme operaci odečítání vztahem $x-y=x+(-y)$,
kde $-y$ je prvek opačný k $y$. Symbol $\cdot$ při násobení budeme zpravidla
vynechávat.

Důsledky definice: nulový prvek je jediný, ke každému prvku z $V$ existuje
právě jeden opačný prvek, pro každé $x\in V$ platí $-x=(-1)\cdot x$
a $0\cdot x = 0$ (pozor: nula na levé straně je nula v $\mathbb{R}$, nula na pravé
straně je nulový prvek ve $V$!).

Podmnožina $V_0\subset V$ vektorového prostoru $V$ je \emph{podprostor}
prostoru $V$, pokud je uzavřená vzhledem ke sčítání vektorů a násobení skaláru
a vektoru; značíme $V_0\subset\subset V$.

\begin{ex}
   Příkladem vektorového prostoru je prostor $n$-tic ($n\in\mathbb{N}$)
   reálných čísel $\mathbb{R}^n$ s obvyklou definicí sčítání a násobení skalárem.
   Nulovým prvkem je vektor $(0,\ldots,0)$.
\end{ex}

\begin{ex}
   Množina $C[a,b]$ spojitých funkcí na uzavřeném intervalu je vektorový prostor
   (součet spojitých funkcí je spojitá funkce, násobek skaláru a spojité funkce je
   opět spojitá funkce).
   Nulový prvek je nulová funkce na $[a,b]$.
\end{ex}

\begin{ex}
   Množina $P_n[a,b]$ polynomů stupně nejvýše $n\in\mathbb{N}$, 
   tj. všech funkcí ve tvaru
   \[
      p:\;t\mapsto\sum\limits_{i=0}^{n}\alpha_i t^i, \; t\in[a,b],
   \]
   kde $\{\alpha_i\}_{i=0}^n\subset\mathbb{R}$.
   Protože každý polynom je spojitá funkce a $P_n[a,b]$ je uzavřený vzhledem
   ke sčítání a násobení skalárem, platí $P_n[a,b]\subset\subset C[a,b]$.
\end{ex}

\section{Banachovy a Hilbertovy prostory}
%%%%%%%%%%%%%%%%%%%%%%%%%%%%%%%%%%%%%%%%%

Zobrazení $\|\cdot\|:\;V\rightarrow \mathbb{R}$ definované na lineárním prostoru $V$
nazveme \emph{normou} na $V$, jestliže
\begin{enumerate}
   \item $(\forall x\in V)(\|x\|=0\;\Leftrightarrow x=0)$,
   \item $(\forall x,y\in V)(\|x+y\|\leq\|x\|+\|y\|)$ -- trojúhelníková nerovnost,
   \item $(\forall \alpha\in\mathbb{R})(\forall x\in V)(\|\alpha x\|=|\alpha|\|x\|)$
         -- homogenita.
\end{enumerate}

Důsledky definice:
\begin{itemize}
   \item $(\forall x\in V\setminus\{0\})(\|x\|>0)$:
         je-li $x\in V$, pak z trojúhelníkové rovnosti a homogenity plyne
         $0=\|x-x\|\leq\|x\|+\|-x\|=\|x\|+\|x\|=2\|x\|$, takže $\|x\|\geq 0$;
         ale rovnost nastane jen v případě, že $x=0$;
   \item $(\forall x,y\in V)(|\|x\|-\|y\||\leq \|x-y\|)$;
   \item norma je spojitá funkce prvků z $V$.
\end{itemize}

Prostor $V$ s definovanou normou $\|\cdot\|$ nazýváme 
\emph{normovaný lineární prostor}.

\begin{ex}
   Nechť $x\in\mathbb{R}^n$, $x=(\alpha_1,\ldots,\alpha_n)$.
   Na $\mathbb{R}^n$ můžeme definovat normy:
   \[
         \|x\|_1 \equiv \sum\limits_{i=1}^n |\alpha_i|,\;\;
         \|x\|_2 \equiv \sqrt{\sum\limits_{i=1}^n \alpha_i^2},\;\;
         \|x\|_{\infty} \equiv \max\limits_{i=1,\ldots,n}|\alpha_i|.
   \]
\end{ex}

\begin{ex}
   Nechť $f\in C[a,b]$.
   Analogicky jako v případě $\mathbb{R}^n$ lze na $C[a,b]$ definovat normy:
   \[
      \begin{split}
         \|f\|_1 \equiv \int\limits_a^b|f(t)|\,dt,\;\;
         \|f\|_2 \equiv \sqrt{\int\limits_a^b|f(t)|^2\,dt},\;\;
         \|f\|_{\infty} \equiv \max\limits_{t\in[a,b]}|f(t)|.
      \end{split}
   \]
\end{ex}

Úplné prostory: Normovaný prostor $V$ je \emph{úplný}, jestliže každá 
Cauchyovská posloupnost%
\footnote{Posloupnost $\{x_n\}\subset V$ je Cauchyovská, pokud 
$(\forall \varepsilon<0)(\exists N\in\mathbb{N})(\forall m,n>N)
(\|x_m-x_n\|<\varepsilon)$. Jinými slovy, pokud jsou si prvky této posloupnosti
libovolně blízké počínaje nějakým indexem $N$. Platí, že každá konvergentní
posloupnost je Cauchyovská, naopak to však platit nemusí.}
ve $V$ konverguje k prvku ve $V$.
Úplnému normovanému lineárnímu prostoru říkáme \emph{Banachův prostor}.

\begin{ex}
   Těleso reálných čísel $\mathbb{R}$ chápáno jako vektorový prostor nad $\mathbb{R}$
   s normou $\|x\|=|x|$, $x\in\mathbb{R}$, je Banachův prostor,
   neboť každá Cauchyovská posloupnost reálných čísel je konvergentní v $\mathbb{R}$.
   Stejně tak prostor $\mathbb{R}^n$, $n\in\mathbb{N}$ je Banachův.
\end{ex}

\begin{ex}
   Těleso $\mathbb{Q}$ racionálních čísel chápáno jako vektorový nad $\mathbb{Q}$
   není Banachův prostor, neboť $\{x_n\}\subset\mathbb{Q}$, kde
   $x_n\equiv (1+n^{-1})^n$, $n\in\mathbb{N}$, konverguje v $\mathbb{R}$
   k číslu $e=\exp(1)\not\in\mathbb{Q}$.
\end{ex}

\begin{ex}
   Prostor $C[a,b]$ s normou $\|\cdot\|_{\infty}$
   je Banachův prostor, neboť konvergence spojitých funkcí v normě 
   $\|\cdot\|_{\infty}$ je ekvivalentní stejnoměrné konvergenci spojitých funkcí,
   jejíž limita je spojitá funkce na $[a,b]$.
\end{ex}

\begin{ex}
   Prostor $C[a,b]$ s normou $\|\cdot\|_2$ není Banachův.
   Pro jednoduchost uvažujme $a=-1$ a $b=1$ a pro $n\in\mathbb{N}$ definujme
   \[
      f_n\,:t\mapsto\begin{cases}
          -1&\text{pro }t\in(-1,-\frac{1}{n})\\
           1&\text{pro }t\in(\frac{1}{n},1)\\
           nt&\text{pro }t\in[-\frac{1}{n},\frac{1}{n}].
           \end{cases}
   \]
   Zřejmě $\{f_n\}\subset C[-1,1]$.
   Definujeme-li dále
   \[
      f\,:t\mapsto\begin{cases}
          -1&\text{pro }t\in(-1,0)\\
           1&\text{pro }t\in(0,1)\\
           0&\text{pro }t=0.
           \end{cases}
   \]
   Samozřejmě $f\not\in C[-1,1]$.
   Posloupnost $\{f_n\}$ však konverguje k $f$ v normě $\|\cdot\|_2$, neboť
   \[
     \begin{split}
      \|f-f_n\|_2&=\int\limits_{-1}^1|f(t)-f_n(t)|\,dt
                  =\int\limits_{-\frac{1}{n}}^0 (1+nt)\,dt
                  +\int\limits_0^{\frac{1}{n}}(1-nt)\,dt
                  =\frac{1}{2n}+\frac{1}{2n}=\frac{1}{n}.
     \end{split}
   \]
   Protože $\frac{1}{n}\rightarrow 0$ pro $n\rightarrow\infty$,
   také $\|f-f_n\|_2\rightarrow 0$, takže $f_n\rightarrow f$.
   Našli jsme tedy posloupnost spojitých funkcí, která nekonverguje ke spojité funkci
   v normě $\|\cdot\|_2$.
\end{ex}

Poznámka: Odpověď na otázku, zda je normovaný prostor Banachův, 
může silně záviset na zvolené normě, jak bylo zřejmé z předchozího příkladu.
Na prostorech konečné dimenze (prostory s konečnou bází) je odpověď jasná:
je-li $\{x_1,\ldots,x_n\}\subset V$ báze $V$ a pro $x\in V$
je $x=\sum\limits_{i=1}^n\alpha_ix_i$ ($\{\alpha_i\}_{i=1}^n\subset\mathbb{R}$)
a definujeme-li $\|x\|_{\infty}\equiv\max\limits_{i=1,\ldots,n}|\alpha_i|$,
lze ukázat, že $V$ je pak úplný v této normě, a tedy Banachův.
Protože jsou v prostorech konečné dimenze všechny normy 
ekvivalentní%
\footnote{Dvě normy $\|\cdot\|_a$ a $\|\cdot\|_b$ na $V$ jsou ekvivalentní,
pokud existují kladné konstanty $\alpha$ a $\beta$ takové, že
$(\forall x\in V)(\alpha\|x\|_a\leq\|x\|_b\leq\beta\|x\|_a)$.},
jsou konečně-dimenzionální normované prostory úplné (Banachovy)
v každé normě!

Nechť $V$ je lineární vektorový prostor.
Zobrazení $\langle\cdot,\cdot\rangle:\,V\times V\rightarrow\mathbb{R}$
nazveme skalární součin na $V$, pokud platí:
\begin{enumerate}
   \item $(\forall x\in V)(\langle x,x\rangle\geq 0\;
            \&\;\langle x,x\rangle=0\;\Leftrightarrow x=0)$,
   \item $(\forall x,y\in V)(\langle x,y\rangle=\langle y,x\rangle)$ -- symetrie,
   \item $(\forall \alpha\in V)(\forall x,y\in V)
          (\langle\alpha x,y\rangle=\alpha\langle x,y\rangle)$ -- homogenita,
   \item $(\forall x,y,z\in V)
          (\langle x,y+z\rangle = \langle x,y\rangle+\langle x,z\rangle)$ 
          -- aditivita.
\end{enumerate}
Prostor $V$ s definovaným skalárním součinem nazýváme
\emph{unitární prostor}, popř. Euklidův prostor.

Důsledky definice:
\begin{itemize}
   \item $(\forall \alpha\in V)(\forall x,y\in V)
          (\langle x,\alpha y\rangle=\alpha\langle x,y\rangle)$;
   \item $(\forall x,y,z\in V)
          (\langle x+y,z\rangle = \langle x,z\rangle+\langle y,z\rangle)$;
   \item Schwarzova nerovnost: $(\forall x,y\in V)
         (|\langle x,y\rangle|\leq\sqrt{\langle x,x\rangle}\sqrt{\langle y,y\rangle})$.
\end{itemize}

Definujeme-li pro $x\in V$ $\|x\|=\sqrt{\langle x,x\rangle}$,
dostaneme porovnáním s definicí normy za použití vlastností skalárního součinu
a důsledků této definice, že $\|\cdot\|$ je norma na $V$.
Je-li tedy prostor $V$ unitární, je zároveň normovaný.
Je-li navíc $V$ úplný v této normě, tj. Banachův, nazývá se \emph{Hilbertův prostor}.

\begin{ex}
   Nechť $x,y\in\mathbb{R}^n$, $x=(\alpha_1,\ldots,\alpha_n)$,
   $x=(\beta_1,\ldots,\beta_n)$.
   Na $\mathbb{R}^n$ lze definovat Euklidův skalární součin vztahem
   \[
      \langle x,y\rangle=\sum\limits_{i=1}^n \alpha_i \beta_i.
   \]
   Tento skalární součin definuje normu $\|\cdot\|_2$ na $\mathbb{R}^n$.
\end{ex}

\begin{ex}
   Nechť $f,g\in C[a,b]$.
   Obdobně i zde můžeme definovat skalární součin vztahem
   \[
      \langle x,y\rangle=\int\limits_a^b f(t)g(t)\,dt.
   \]
   Tento skalární součin definuje normu $\|\cdot\|_2$ na $C[a,b]$.
\end{ex}

Řekneme, že dva vektory $x,y\in V$ jsou \emph{ortogonální}, 
pokud $\langle x,y\rangle=0$.

