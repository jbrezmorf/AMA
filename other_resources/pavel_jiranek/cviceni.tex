\documentclass[openany,a4paper,10pt]{book}

%\usepackage{czech}
\usepackage[czech]{babel}
\usepackage[latin2]{inputenc}
\usepackage{amsmath}
\usepackage{amssymb}

\newtheorem{ex}{P��klad}
\newenvironment{sol}{\paragraph{�e�en�}}{}

\renewcommand{\vec}[1]{\mathbf{#1}}
\renewcommand{\Re}{\mathrm{Re}\,}
\renewcommand{\Im}{\mathrm{Im}\,}
\newcommand{\Div}{\mathrm{div}\,}
\newcommand{\Rot}{\mathrm{rot}\,}
\newcommand{\Grad}{\mathrm{grad}\,}

\parindent=0pt

\begin{document}

\chapter{Opakov�n�}
%%%%%%%%%%%%%%%%%%%%%%%%%%%%%%%%%%%%%%%%%%%%%%%%%%%%%%%%%%%%%%%%%%%%%%%%%%%%%%%%

\section{Dvojn� a trojn� integr�ly}
%%%%%%%%%%%%%%%%%%%%%%%%%%%%%%%%%%%%%%%%%%%%%%%%%%%%%%%%%%%%%%%%%%%%%%%%%%%%%%%%

V n�sleduj�c�m textu si zopakujeme v�po�et dvojn�ch a trojn�ch integr�l�
na speci�ln�ch typech oblast� v $E_2$ a $E_3$.
Tyto oblasti budeme jednotn� zna�it symbolem $\Omega$,
integrovanou funkci budeme zna�it symbolem $f$.

Integra�n� obory v $E_2$:
\begin{itemize}
   \item Kart�zsk� sou�adnice:
   \begin{itemize}
      \item Obrazec mezi grafy spojit�ch funkc� $y=l(x)$ a $y=u(x)$
            a p��mkami $x=a$, $x=b$ ($a<b$, $l(x)\leq u(x)$ pro $x\in [a,b]$),
            \[
               \iint\limits_{\Omega} f(x,y)\,dx\,dy = \int\limits_a^b\left(\int\limits_{l(x)}^{u(x)}f(x,y)\,dy\right)\,dx.
            \]
      \item Obrazec mezi grafy spojit�ch funkc� $x=l(y)$ a $x=u(y)$
            a p��mkami $y=a$, $y=b$ ($a<b$, $l(y)\leq u(y)$ pro $y\in [a,b]$).
            \[
               \iint\limits_{\Omega} f(x,y)\,dx\,dy = \int\limits_a^b\left(\int\limits_{l(y)}^{u(y)}f(x,y)\,dx\right)\,dy.
            \]
   \end{itemize}
   \item Pol�rn� sou�adnice:
         (transforma�n� rovnice $x=\varrho\cos\varphi$, $y=\varrho\sin\varphi$):
         Obrazec mezi hrani�n�mi polop��mkami $\varphi=\alpha$, $\varphi=\beta$,
            ($\alpha\leq\beta$, $\beta-\alpha\leq 2\pi$) a grafy spojit�ch
            funkc� $\varrho=l(\varphi)$, $\varrho=u(\varphi)$,
            \[
               \iint\limits_{\Omega} f(x,y)\,dx\,dy = \iint\limits_{\Omega} g(\varrho,\varphi)\varrho\,d\varrho\,d\varphi,
            \]
            tj.
            \[
               \iint\limits_{\Omega} f(x,y)\,dx\,dy = \int\limits_{\alpha}^{\beta}\left(\int\limits_{l(\varphi)}^{u(\varphi)}g(\varrho,\varphi)\varrho\,d\varrho\right)\,d\varphi,
            \]
            kde $g(\varrho,\varphi)=f(\varrho\cos\varphi,\varrho\sin\varphi)$.
\end{itemize}

Integra�n� obory v $E_3$:
\begin{itemize}
   \item Kart�zsk� sou�adnice:
         Oblast ur�en� nerovnostmi $x\in[a,b]$, $y\in [l(x),u(x)]$, 
            $z\in[L(x,y),U(x,y)]$, kde $l,u,L,U$ jsou spojit� funkce,
            \[
               \iiint\limits_{\Omega} f(x,y,z)\,dx\,dy\,dz = \int\limits_a^b\left(\int\limits_{l(x)}^{u(x)}\left(\int\limits_{L(x,y)}^{H(x,y)}f(x,y,z)\,dz\right)\,dy\right)\,dx.
            \]
   \item Cylindrick� sou�adnice 
         (transforma�n� rovnice $x=\varrho\cos\varphi$, $y=\varrho\sin\varphi$, $z=z$):
         Oblast ur�en� nerovnostmi $\varphi\in[\alpha,\beta]$, $\varrho\in [l(\varphi),u(\varphi)]$, 
            $z\in[L(\varrho,\varphi),U(\varrho,\varphi)]$, kde $l,u,L,U$ jsou spojit� funkce,
            \[
               \iiint\limits_{\Omega} f(x,y,z)\,dx\,dy\,dz = \iiint\limits_{\Omega} g(\varrho,\varphi,z)\varrho\,d\varrho\,d\varphi\,d\varphi,
            \]
            tj.
            \[
               \iiint\limits_{\Omega} f(x,y,z)\,dx\,dy\,dz = \int\limits_{\alpha}^{\beta}\left(\int\limits_{l(\varphi)}^{u(\varphi)}\left(\int\limits_{L(\varrho,\varphi)}^{H(\varrho,\varphi)}g(\varrho,\varphi,z)\,dz\right)\,d\varrho\right)\,d\varphi,
            \]
            kde $g(\varrho,\varphi,z)=f(\varrho\cos\varphi,\varrho\sin\varphi,z)$.
   \item Sf�rick� sou�adnice
         (transforma�n� sou�adnice $x=\varrho\cos\vartheta\cos\varphi$,
         $y=\varrho\cos\vartheta\sin\varphi$, $z=\varrho\sin\vartheta$):
         Oblast ur�en� nerovnostmi $\varphi\in[\alpha,\beta]$, $\vartheta\in[\gamma,\delta]$,
            $\varrho\in [L(\varphi,\vartheta),U(\varphi,\vartheta)]$, 
            kde $L,U$ jsou spojit� funkce,
            \[
               \iiint\limits_{\Omega} f(x,y,z)\,dx\,dy\,dz = \iiint\limits_{\Omega} g(\varrho,\varphi,\vartheta)\varrho^2\sin\vartheta\,d\varrho\,d\varphi\,d\varphi,
            \]
            tj.
            \[
               \iiint\limits_{\Omega} f(x,y,z)\,dx\,dy\,dz = \int\limits_{\alpha}^{\beta}\left(\int\limits_{\gamma}^{\delta}\left(\int\limits_{L(\varphi,\vartheta)}^{H(\varphi,\vartheta)}g(\varrho,\varphi,\vartheta)\rho^2\sin\theta\,d\varrho\right)\,d\vartheta\right)\,d\varphi,
            \]
            kde $g(\varrho,\varphi,\vartheta)=f(\varrho\cos\vartheta\cos\varphi,\varrho\cos\vartheta\sin\varphi,\varrho\sin\vartheta)$.
\end{itemize}

\begin{ex}
   Spo�t�te integr�ly
   \begin{enumerate}
      \item
      \[
         I = \iint\limits_{\Omega}(x+y)\,dx\,dy,
      \]
      kde $\Omega=\{[x,y]\in E_2\,|\,y\in[x+1,x^2],\;x\in[0,1]\}$.
      \item
      \[
         I = \iint\limits_{\Omega}e^{(y^2)}\,dx\,dy,
      \]
      kde $\Omega=\{[x,y]\in E_2\,|\,y\in[x,1],\;x\in[0,1]\}$.
      \item
      \[
         I = \iint\limits_{\Omega}x\,dx\,dy,
      \]
      kde $\Omega=\{[x,y]\in E_2\,|\,\;y=[-x,x], \;x\in[0,1]\;\&\;x^2+(y-r)^2\leq r^2\}$.
      \item
      \[
         I = \iint\limits_{\Omega}\,dx\,dy,
      \]
      kde $\Omega=\{[x,y]\in E_2\,|\,\;r_1\leq x^2+y^2\leq r_2, \; r_1<r_2\}$.
      \item
      \[
         I = \iiint\limits_{\Omega}\,dx\,dy\,dz,
      \]
      kde $\Omega=\{[x,y,z]\in E_3\,|\,x\in[0,1],\;y\in[0,\frac{1}{2}(1-x)],\;z\in[0,1-x-2y]\}$.
      \item
      \[
         I = \iiint\limits_{\Omega}z\sqrt{x^2+y^2}\,dx\,dy\,dz,
      \]
      kde $\Omega=\{[x,y,z]\in E_3\,|\,x^2+y^2-2x\leq 0\;\&\;y\geq 0\;\&\;z\in[0,k],\;k>0\}$.
      \item
      \[
         I = \iiint\limits_{\Omega}x^2\,dx\,dy\,dz,
      \]
      kde $\Omega = \{[x,y,z]\in E_3\,|\,x^2+y^2+z^2\leq r^2\}$.
   \end{enumerate}
\end{ex}
\begin{sol}
   \begin{enumerate}
      \item $I=\frac{33}{20}$.
      \item $I = \frac{1}{2}(e-1)$.
      \item $I = 0$.
      \item $I=\pi(r_2^2-r_1^2)$.
      \item $I = \frac{1}{12}$.
      \item $I=\frac{8}{9}k^2$.
      \item $I=\frac{4}{15}\pi r^5$.
   \end{enumerate}
\end{sol}

%%%%%%%%%%%%%%%%%%%%%%%%%%%%%%%%%%%%%%%%%%%%%%%%%%%%%%%%%%%%%%%%%%%%%%%%%%%%%%%%
\chapter{Vektorov� anal�za}
%%%%%%%%%%%%%%%%%%%%%%%%%%%%%%%%%%%%%%%%%%%%%%%%%%%%%%%%%%%%%%%%%%%%%%%%%%%%%%%%

%%%%%%%%%%%%%%%%%%%%%%%%%%%%%%%%%%%%%%%%%%%%%%%%%%%%%%%%%%%%%%%%%%%%%%%%%%%%%%%%
\section{Vektorov� pole. Z�kladn� pojmy.}
%%%%%%%%%%%%%%%%%%%%%%%%%%%%%%%%%%%%%%%%%%%%%%%%%%%%%%%%%%%%%%%%%%%%%%%%%%%%%%%%

Vektorov� funkce (vektorov� pole) re�ln� prom�nn� $t$ v $E_3$: 
$\vec{f}=f_1\vec{e}_1+f_2\vec{e}_1+f_3\vec{e}_3$,
kde $f_1$, $f_2$, $f_3$ jsou (re�ln�) funkce,
$\vec{e}_1=(1,0,0)$, $\vec{e}_2=(0,1,0)$, $\vec{e}_3=(0,0,1)$.

Oper�tor ``nabla'': $\nabla\equiv\partial_x\vec{e}_1+\partial_y\vec{e}_2+\partial_z\vec{e}_3$.

Divergence: $\Div\vec{f} \equiv \nabla\cdot\vec{f} = \partial_x f_1 + \partial_y f_2 + \partial_z f_3$.

Rotace: $\Rot\vec{f} \equiv \nabla\times\vec{f} = (\partial_y f_3 - \partial_z f_2)\vec{e}_1
                               +(\partial_z f_1 - \partial_x f_3)\vec{e}_2
                               +(\partial_x f_2 - \partial_y f_1)\vec{e}_3$.
        Symbolicky
        \[
           \nabla\times\vec{f} = \det\begin{pmatrix}
                                    \vec{e}_1 & \vec{e}_2 & \vec{e}_3 \\
                                    \partial_x & \partial_y & \partial_z \\
                                    f_1 & f_2 & f_3  
                                 \end{pmatrix}.
        \]

Gradient skal�rn� funkce $g$: $\Grad g \equiv \nabla g = \partial_x g\vec{e}_1+\partial_y g\vec{e}_2+\partial_z g\vec{e}_3$.

Potenci�lov� pole: $\vec{f}$ je potenci�lov� na $M\subset E_3$
                   $\Leftrightarrow$ $\exists$ funkce $g$ se spojit�mi 
                   parci�ln�mi derivacemi na $M$ tak,
                   �e $\vec{f} = \Grad g$ na $M$ ($g$ je potenci�lov� funkce)
                   $\Leftrightarrow$ $\Rot\vec{f}=0$.

V $E_2$ se podm�nka $\Rot\vec{f}=0$ redukuje na podm�nku $\partial_x f_2 = \partial_y f_1$.

\begin{ex}
   Zjist�te, zda jsou vektorov� pole 
   \begin{enumerate}
      \item $\vec{f}(x,y) = y^2\vec{e}_1+(2xy+1)\vec{e}_2+3z^2\vec{e}_3$,
      \item $\vec{f}(x,y) = x^2y\vec{e}_1+xy\vec{e}_2$,
      \item $\vec{f}(x,y)=2xy\vec{e}_1+(x^2-y)\vec{e}_2$.
      \item $\vec{f}(x,y)=x^2y\vec{e}_1+xy\vec{e}_2$,
      \item $\vec{f}(x,y)=2xy\vec{e}_1+x^2\vec{e}_2$.
   \end{enumerate}
   potenci�lov� na $E_3$, pop�. na $E_2$ a v kladn�m p��pad� 
   nalezn�te potenci�lov� funkce $g$.
\end{ex}
\begin{sol}
   \begin{enumerate}
      \item Ano, $g(x,y,z)=xy^2+y+z^3+k$, $k\in E_1$.
      \item Ne.
      \item Ano, $g(x,y)=x^2y-\frac{1}{2}y^2+k$, $k\in E_1$.
      \item Ne.
      \item Ano, $g(x,y)=x^2y+k$, $k\in E_1$.
   \end{enumerate}
\end{sol}

%%%%%%%%%%%%%%%%%%%%%%%%%%%%%%%%%%%%%%%%%%%%%%%%%%%%%%%%%%%%%%%%%%%%%%%%%%%%%%%%
\section{K�ivkov� integr�ly}
%%%%%%%%%%%%%%%%%%%%%%%%%%%%%%%%%%%%%%%%%%%%%%%%%%%%%%%%%%%%%%%%%%%%%%%%%%%%%%%%

\subsection{K�ivky}
%%%%%%%%%%%%%%%%%%%%%%%%%%%%%%%%%%%%%%%%%%%%%%%%%%%%%%%%%%%%%%%%%%%%%%%%%%%%%%%%

Zad�n� k�ivky:
\begin{itemize}
   \item parametricky, $\gamma$: $x=x(t)$, $y=y(t)$, $z=z(t)$, $t\in[a,b]$,
   \item ve vektorov�m tvaru, $\gamma$: $\vec{r}(t)=x(t)\vec{e}_1+y(t)\vec{e}_2+z(t)\vec{e}_3$, $t\in[a,b]$.
\end{itemize}

te�n� vektor: 
   $\vec{t}(t) \equiv \dot{x}(t)\vec{e}_1+\dot{y}(t)\vec{e}_2+\dot{z}(t)\vec{e}_3$.

$\gamma$ je hladk� $\Leftrightarrow$ $(\forall t\in[a,b])(\exists$ $\dot{\vec{r}}(t)\neq 0)$
(existuj� te�n� vektory).

$\gamma$ je po ��stech hladk� $\Leftrightarrow$ $[a,b]$ sjednocen�m disjunktn�ch interval�
a na ka�d�m z t�chto interval� je $\gamma$ hladk� (te�n� vektor existuje skoro v�ude).

$\gamma$ je jednoduch� $\Leftrightarrow$ $(\forall t_1,t_2\in[a,b])(t_1\neq t_2\Rightarrow\vec{r}(t_1)\neq\vec{r}(t_2))$.

$\gamma$ je uzav�en� $\Leftrightarrow$ $\vec{r}(a)=\vec{r}(b)$.

Orientace k�ivky je d�na uspo��d�n�m intervalu $[a,b]$, resp. t�m, jak�m ``sm�rem'' proch�z�
parametr $t$ tento interval.

Jednoduch� uzav�en� k�ivka je orientovan� kladn�, resp. z�porn�, pokud je 
p�i ob�h�n� k�ivky ve sm�ru jej� orientace ``vnit�ek'' k�ivky po lev�, resp. po prav� stran�.

\subsection{K�ivkov� integr�l prvn�ho druhu}
%%%%%%%%%%%%%%%%%%%%%%%%%%%%%%%%%%%%%%%%%%%%%%%%%%%%%%%%%%%%%%%%%%%%%%%%%%%%%%%%

Zobecn�n� pojmu ur�it�ho integr�lu (jako integr�lu po �se�ce) na libovolnou k�ivku.

Je-li $\gamma$ hladk�, pak
\[
   \int\limits_{\gamma} f(x,y,z)\,ds = \int\limits_a^b f(x(t),y(t),z(t))\sqrt{\dot{x}^2(t)+\dot{y}^2(t)+\dot{z}^2(t)}\,dt.
\]

\begin{ex}
   Spo�t�te integr�ly
   \begin{enumerate}
      \item
      \[
         I = \int\limits_{\gamma}(x^2-y+3z)\,ds,
      \]
      kde $\gamma$ je �se�ka z bodu $[0,0,0]$ do bodu $[1,2,1]$.
      \item
      \[
         I = \int\limits_{\gamma}(x^2+y^2)\,ds,
      \]
      kde $\gamma$ je kladn� orientovan� hranice troj�heln�ka s vrcholy $[0,0]$, $[1,0]$, $[0,1]$.
      \item
      \[
         I = \left|\int\limits_{\gamma}\,ds\right|,
      \]
      kde $\gamma$ je kru�nice $x^2+y^2=r^2$.
      \item
      \[
         I = \int\limits_{\gamma}(x+2)\,ds,
      \]
      kde $\gamma=\{[x,y,z]\in E_3\,|\,x=t,\;y=\frac{4}{3}t^{\frac{3}{2}},\;z=\frac{1}{2}t^2,\;t\in[0,2]\}$.
   \end{enumerate}
\end{ex}
\begin{sol}
   \begin{enumerate}
      \item $I=\frac{5\sqrt{6}}{6}$.
      \item $I=\frac{2}{3}(1+\sqrt{2})$.
      \item $I=2\pi r$.
      \item $I=\frac{1}{3}(13\sqrt{13}-1)$.
   \end{enumerate}
\end{sol}

\subsection{K�ivkov� integr�l druh�ho druhu}
%%%%%%%%%%%%%%%%%%%%%%%%%%%%%%%%%%%%%%%%%%%%%%%%%%%%%%%%%%%%%%%%%%%%%%%%%%%%%%%%

V�znam pr�ce vektorov�ho pole $\vec{f}(x,y,z)$ po k�ivce $\gamma$.

Je-li $\gamma$ hladk�, pak
\[
   \int\limits_{\gamma}\vec{f}\,d\vec{r} = \int\limits_a^b \vec{f}(x(t),y(t),z(t))\dot{\vec{r}}(t)\,dt 
    = \int\limits_{\gamma} f_1\,dx+f_2\,dy+f_3\,dz.
\]

\begin{ex}
   Spo�t�te integr�ly
   \begin{enumerate}
      \item
      \[
         I = \int\limits_{\gamma}\,dx+x\,dy,
      \]
      kde $\gamma$ je ��st paraboly $y=x^2$ s po��te�n�m bodem $[1,1]$ a koncov�m bodem $[2,4]$.
      \item
      \[
         I = \int\limits_{\gamma}\,y\,dx+x^2\,dy,
      \]
      kde $\gamma$ je ��st paraboly $y=4-x^2$ z $[4,0]$ do $[1,3]$.
      \item
      \[
         I = \int\limits_{\gamma}\,(2x-y)\,dx+(x+3y)\,dy,
      \]
      kde $\gamma$ je lomen� ��ra spojuj�c� body $[0,0]$, $[3,0]$, $[3,3]$.
   \end{enumerate}
\end{ex}
\begin{sol}
   \begin{enumerate}
      \item $I=\frac{17}{3}$.
      \item $I=\frac{69}{2}$.
      \item $I=\frac{63}{2}$.
   \end{enumerate}
\end{sol}

\begin{ex}
   Spo�t�te pr�ci vektorov�ho pole $\vec{f}$ po k�ivce $\gamma$, kde
   \begin{enumerate}
      \item $\vec{f}(x,y)=xy\vec{e}_1+y\vec{e}_2$, $\gamma=\{[x,y]\in E_2\,|\,x=4t,\;y=t,\;t\in[0,1]\}$,
      \item $\vec{f}(x,y)=3x\vec{e}_1+4y\vec{e}_2$, $\gamma=\{[x,y]\in E_2\,|\,x=2\cos t,\;y=2\sin t,\;t\in[0,\frac{\pi}{2}]\}$,
      \item $\vec{f}(x,y)=x^2y\vec{e}_1+(x-z)\vec{e}_2+xyz\vec{e}_3$, $\gamma=\{[x,y,z]\in E_3\,|\,x=t,\;y=t^2,\;z=2,\;t\in[0,1]\}$,
      \item $\vec{f}(x,y)=x\vec{e}_1+y\vec{e}_2-5z\vec{e}_3$, $\gamma=\{[x,y,z]\in E_3\,|\,x=2\cos t,\;y=2\sin t,\;z=t,\;t\in[0,2\pi]\}$.
   \end{enumerate}
\end{ex}
\begin{sol}
   \begin{enumerate}
      \item $\frac{35}{6}$.
      \item $2$.
      \item $-\frac{17}{15}$.
      \item $-10\pi^2$.
   \end{enumerate}
\end{sol}

\subsection{Nez�vislost na integra�n� cest�}
%%%%%%%%%%%%%%%%%%%%%%%%%%%%%%%%%%%%%%%%%%%%%%%%%%%%%%%%%%%%%%%%%%%%%%%%%%%%%%%%

M�-li $\vec{f}$ spojit� slo�ky na $M\subset E_3$, $\vec{f}$ je potenci�lov� pole
na $M$ s potenci�lovou funkc� $g$, $\gamma$ hladk� k�ivka v $M$, pak
\[
   \int\limits_{\gamma}\vec{f}\,d\vec{r} = [g(x,y,z)]_{[x(a),y(a),z(a)]}^{[x(b),y(b),z(b)]}.
\]
Integr�l tedy nez�vis� na integra�n� cest�, tj. na tvaru k�ivky $\gamma$, 
ale jen na hodnot�ch potenci�lov� funkce $g$ v po��te�n�m a koncov�m bodu.
Je-li nav�c $\gamma$ uzav�en�, je
\[
   \oint\limits_{\gamma}\vec{f}\,d\vec{r} = 0.
\]

\begin{ex}
   Spo�t�te integr�ly
   \begin{enumerate}
      \item
      \[
         I = \int\limits_{\gamma}y\,dx+x\,dy,
      \]
      kde $\gamma$ je lomen� ��ra spojuj�c� postupn� body $[0,0]$, $[1,1]$, $[2,4]$, $[4,-1]$.
      \item
      \[
         I = \int\limits_{\gamma}2xy\,dx+(x^2-y)\,dy,
      \]
      kde $\gamma$ je hladk� k�ivka spojuj�c� body $[-1,4]$ a $[1,2]$.
      \item
      \[
         I = \int\limits_{\gamma}(y\vec{e}_1+x\vec{e}_2)\,d\vec{r},
      \]
      kde $\gamma$ je lomen� ��ra spojuj�c� body $[0,0]$, $[1,2]$, $[4,5]$, $[5,8]$.
      \item
      \[
         I = \int\limits_{\gamma}e^x\sin y\,dx+e^x\cos y\,dy,
      \]
      kde $\gamma$ je ��st cykloidy $x=t-\sin t$, $y=1-\cos t$ z $[0,0]$ do $[2\pi,0]$.
      \item
      \[
         I = \int\limits_{\gamma}(y^3+1)\,dx+(3xy^2+1)\,dy,
      \]
      kde $\gamma=\{[x,y]\in E_2\,|\,x^2+y^2=r^2, \; r>0\}$.
   \end{enumerate}
\end{ex}
\begin{sol}
   \begin{enumerate}
      \item $I=-4$.
      \item $I=4$.
      \item $I=24$.
      \item $I=0$.
      \item $I=0$.
   \end{enumerate}
\end{sol}

\subsection{Greenova v�ta}
%%%%%%%%%%%%%%%%%%%%%%%%%%%%%%%%%%%%%%%%%%%%%%%%%%%%%%%%%%%%%%%%%%%%%%%%%%%%%%%%

Nech� $\Omega\subset E_2$ je oblast, jej�� hranice $\gamma$
je jednoduch� uzav�en� po ��stech hladk� k�ivka (orientovan� kladn�)
a $f$, $g$, $\partial_y f$, $\partial_x g$ jsou spojit� na $\Omega$ a $\gamma$.
Pak
\[
   \int\limits_{\gamma} f(x,y)\,dx+g(x,y)\,dy = \iint\limits_{\Omega}(\partial_x g(x,y)-\partial_y f(x,y))\,dx\,dy.
\]

\begin{ex}
   Spo�t�te integr�ly
   \begin{enumerate}
      \item
      \[
         I = \int\limits_{\gamma}y^3\,dx+(x^3+3xy^2)\,dy,
      \]
      kde $\gamma$ je kladn� orientovan� uzav�en� k�ivka sest�vaj�c� se z ��sti kubick� paraboly $y=x^3$
      mezi body $[0,0]$ do $[1,1]$ a �se�ky $y=x$ spojuj�c� tyto body.
      \item
      \[
         I = \int\limits_{\gamma}y^3\,dx+(x^3+3xy^2)\,dy,
      \]
      kde $\gamma$ je kladn� orientovan� kru�nice $x^2+y^2=9$.
      \item
      \[
         I = \int\limits_{\gamma}(y-x)\,dx+(2x-y)\,dy,
      \]
      kde $\gamma$ je kladn� orientovan� hranice mezi grafy funkc� $y=x$, $y=x^2-x$.
      \item
      \[
         I = \int\limits_{\gamma}\sin x\cos x\,dx+(xy+\cos x\sin y)\,dy,
      \]
      kde $\gamma$ je kladn� orientovan� hranice mezi grafy funkc� $y=x$, $y=\sqrt{x}$.
   \end{enumerate}
\end{ex}
\begin{sol}
   \begin{enumerate}
      \item $I=-4$.
      \item $I=\frac{243}{4}\pi$.
      \item $I=\frac{4}{3}$.
      \item $I=\frac{1}{12}$.
   \end{enumerate}
\end{sol}

%%%%%%%%%%%%%%%%%%%%%%%%%%%%%%%%%%%%%%%%%%%%%%%%%%%%%%%%%%%%%%%%%%%%%%%%%%%%%%%%
\section{Plo�n� integr�ly}
%%%%%%%%%%%%%%%%%%%%%%%%%%%%%%%%%%%%%%%%%%%%%%%%%%%%%%%%%%%%%%%%%%%%%%%%%%%%%%%%

\subsection{Plo�n� integr�l prvn�ho druhu}
%%%%%%%%%%%%%%%%%%%%%%%%%%%%%%%%%%%%%%%%%%%%%%%%%%%%%%%%%%%%%%%%%%%%%%%%%%%%%%%%

Je-li plocha $\Gamma$ zad�na funkc� $z=g(x,y)$ se spojit�mi derivacemi $\partial_x g$ a $\partial_y g$, je
\[
   \iint\limits_{\Gamma}f(x,y,z)\,dS = \iint\limits_{G}f(x,y,g(x,y))\,\sqrt{1+(\partial_x g(x,y))^2+(\partial_y g(x,y))^2}\,dx\,dy,
\]
kde $G$ je pr�m�t plochy $\Gamma$ do roviny $z=0$.

\begin{ex}
   Spo�t�te
   \[
      I=\int\limits f_(x,y,z)\,dS,
   \]
   kde
   \begin{enumerate}
      \item $f(x,y,z)=y^2+2yz$ a $\Gamma$ je ��st roviny $2x+y+2z-6=0$ v prvn�m oktantu,
      \item $f(x,y,z)=x+z$ a $\Gamma$ je ��st v�lcov� plochy $x^2+y^2=9$ v prvn�m oktantu o��znut� rovinou $z=4$,
      \item $f(x,y,z)=xy$ a $\Gamma$ je ��st roviny $z=6-x-2y$ v prvn�m oktantu,
      \item $f(x,y,z)=\sqrt{a^2+x^2+y^2}$ a $\Gamma$ je ��st paraboloidu $x^2+y^2=2az$ o��znut� ku�elem $x^2+y^2=z^2$,
      \item $f(x,y,z)=x^2+y^2+z^2$ a $\Gamma=\{[x,y,z]\in E_3\,|\,z=x+2\;\&\;x^2+y^2\leq 1\}$.
   \end{enumerate}
\end{ex}
\begin{sol}
   \begin{enumerate}
      \item $I=\frac{243}{2}$,
      \item $I=12\pi+36$,
      \item $I=\frac{27\sqrt{6}}{2}$,
      \item $I=12\pi a^3$,
      \item $I=\frac{19\sqrt{2}\pi}{4}$,
   \end{enumerate}
\end{sol}

\subsection{Plo�n� integr�l druh�ho druhu}
%%%%%%%%%%%%%%%%%%%%%%%%%%%%%%%%%%%%%%%%%%%%%%%%%%%%%%%%%%%%%%%%%%%%%%%%%%%%%%%%

Orientace plochy:
Nech� $\Gamma$ je omezen� plocha, oboustrann�.
Orientaci lze zadat slovn�, p��p. sm�rem vn�j�� norm�ly.

Je-li plocha zadan� funkc� $z=g(x,y)$ ($g$ m� spojit� prvn� derivace), jsou
\[
 \begin{split}
   \vec{n}_1 &= \frac{-\partial_x g\,\vec{e}_1-\partial_y g\,\vec{e}_2+\vec{e}_3}{\sqrt{1+(\partial_x g(x,y))^2+(\partial_y g(x,y))^2}},\\
   \vec{n}_2 &= -\vec{n}_1=\frac{\partial_x g\,\vec{e}_1+\partial_y g\,\vec{e}_2-\vec{e}_3}{\sqrt{1+(\partial_x g(x,y))^2+(\partial_y g(x,y))^2}}
 \end{split}
\]
dv� mo�n� volby vn�j�� norm�ly.

Jestli�e je $\vec{f}=f_1\,\vec{e}_1+f_2\,\vec{e}_2+f_3\,\vec{e}_3$
a $\Gamma$ je omezen� orientovan� plocha s jednotkovou vn�j�� norm�lou $\vec{n}$, ozna�ujeme v�razem
\[
   \iint\limits_{\Gamma} \vec{f}\vec{n}\,dS, \text{ resp. } \iint\limits_{\Gamma} f_1\,dy\,dz+f_2\,dx\,dz+f_3\,dx\,dy
\]
plo�n� integr�l druh�ho druhu vektorov� funkce $\vec{f}$ na plo�e $\Gamma$.

Integr�l druh�ho druhu m� fyzik�ln� v�znam toku vektorov�ho pole.

Po dosazen�:
\[
  \begin{split}
     \iint\limits_{\Gamma}\vec{f}\vec{n}_1\,dS
       &= \iint\limits_{\Gamma}(f_1\,\vec{e}_1+f_2\,\vec{e}_2+f_3\,\vec{e}_3))\frac{-\partial_x g\,\vec{e}_1-\partial_y g\,\vec{e}_2+\vec{e}_3}{\sqrt{1+(\partial_x g(x,y))^2+(\partial_y g(x,y))^2}}\,dS\\
       &= \iint\limits_{G}(-f_1\partial_x g-f_2\partial_y g+f_3)\,dx\,dy,
  \end{split}
\]
kde $G$ je pr�m�t plochy $\Gamma$ do roviny $z=0$.

\begin{ex}
   Spo�t�te
   \[
      I=\iint\limits \vec{f}\vec{n}\,dS,
   \]
   kde
   \begin{enumerate}
      \item $\vec{f}(x,y,z)=x\,\vec{e}_1+y\,\vec{e}_2+z\,\vec{e}_3$ a $\Gamma$ je ��st paraboloidu $z=4-x^2-y^2$ nad rovinou $z=0$
            s norm�lou $\vec{n}=\vec{e}_3$ v bod� $(0,0,4)$,
      \item $\vec{f}(x,y,z)=3z\,\vec{e}_1-4\,\vec{e}_2+y\,\vec{e}_3$ a $\Gamma$ je ��st roviny $x+y+z=1$ orientovan� vektorem norm�ly
            sm��uj�c�m do poloprostoru obsahuj�c�ho po��tek,
      \item $\vec{f}(x,y,z)=xy\,\vec{e}_1-x^2\,\vec{e}_2+(x+z)\,\vec{e}_3$ a $\Gamma$ je ��st roviny $2x+2y+z=6$ orientovan� vektorem norm�ly
            sm��uj�c�m do poloprostoru neobsahuj�c�ho po��tek.
   \end{enumerate}
\end{ex}
\begin{sol}
   \begin{enumerate}
      \item $I=24\pi$,
      \item $I=413$,
      \item $I=\frac{27}{4}$.
   \end{enumerate}
\end{sol}

\subsection{Stokesova a Gaussova-Ostrogradsk�ho v�ta}
%%%%%%%%%%%%%%%%%%%%%%%%%%%%%%%%%%%%%%%%%%%%%%%%%%%%%%%%%%%%%%%%%%%%%%%%%%%%%%%%

Orientace okraje plochy: Nech� $\Gamma$ je plocha s okrajem (okrajovou k�ivkou) $\gamma$.
$\gamma$ je orientovan� ve kladn�m smyslu orientace $\Gamma$,
jestli�e p�i sledov�n� k�ivky $\gamma$ v kladn�m smyslu orientace z�st�v�
��st plochy $\Gamma$ uzav�en� k�ivkou $\Gamma$ po lev� stran� (pozorov�no
ze strany vn�j�� norm�ly k plo�e).

Stokesova v�ta: $\Gamma$ plocha s okrajem $\gamma$ a vn�j�� norm�lou $\vec{n}$,
$\gamma$ tvo�� uzav�en� jednoduch� po ��stech hladk� k�ivka,
$\gamma$ je orientovan� ve kladn�m smyslu orientace $\Gamma$,
$\vec{f}=f_1\,\vec{e}_1+f_2\,\vec{e}_2+f_3\,\vec{e}_3$ je vektorov� pole,
$f_i$ m� spojit� parci�ln� derivace na otev�en� mno�in� obsahuj�c� $\Gamma$ a $\gamma$ ($i=1,2,3$), pak
\[
   \iint\limits_{\Gamma}\Rot\vec{f}\,\vec{n}\,dS = \int\limits_{\gamma}\vec{f}\,d\vec{r}.
\]

Gaussova-Ostrogradsk�ho v�ta:
$\Omega\in E_3$ je oblast, jej�� hranici tvo�� uzav�en� plocha $\Gamma$ orientovan� vn�j�� norm�lou $\vec{n}$,
$\vec{f}=f_1\,\vec{e}_1+f_2\,\vec{e}_2+f_3\,\vec{e}_3$ je vektorov� pole,
$\Div f$ je spojit� funkce na otev�en� mno�in� obsahuj�c� $\Omega$ a $\Gamma$, pak
\[
   \iint\limits_{\Gamma} \vec{f}\,\vec{n}\,dS = \iiint\limits_{\Omega}\Div f\,dx\,dy\,dz.
\]

\begin{ex}
   \begin{enumerate}
      \item
      Vypo�t�te (pomoc� Stokesovy v�ty)
      \[
         I = \int\limits_{\gamma}(-y^2\,\vec{e}_1+z\,\vec{e}_2+x\,\vec{e}_3)\,d\vec{r},
      \]
      kde $\gamma$ je hranice troj�heln�ka tvo�en�ho body $[3,0,0]$, $[0,3,0]$ a $[0,0,6]$
      orientovan� ve smyslu uveden�ho po�ad� bod�.
   \end{enumerate}
\end{ex}
\begin{sol}
   \begin{enumerate}
      \item $I=-9$,
   \end{enumerate}
\end{sol}

\begin{ex}
   Vypo�t�te (pomoc� Gaussovy-Ostrogradsk�ho v�ty)
   \[
      I = \iint\limits_{\Gamma}\vec{f}\,\vec{n}\,dS,
   \]
   kde
   \begin{enumerate}
      \item $\vec{f}(x,y,z)=(z^2-x)\,\vec{e}_1-xy\,\vec{e}_2+3z\,\vec{e}_3$ a $\Gamma$ je uzav�en� plocha ohrani�en�
            plochami $z=4-y^2$, $x=0$, $x=3$ a $z=0$ orientovan� vn�j�� norm�lou,
      \item $\vec{f}(x,y,z)=x\,\vec{e}_1+y^2\,\vec{e}_2+z\,\vec{e}_3$ a $\Gamma$ je uzav�en� plocha ohrani�en�
            rovinami $2x+2y+z+6=0$, $x=0$, $y=0$ a $z=0$ orientovan� vn�j�� norm�lou,
      \item $\vec{f}(x,y,z)=(x^2+\sin\,z)\,\vec{e}_1+(xy+\cos\,z)\,\vec{e}_2+e^y\,\vec{e}_3$ a $\Gamma$ je uzav�en� plocha ohrani�en�
            plochami $x^2+y^2=4$, $x+z=6$ a $z=0$ orientovan� vn�j�� norm�lou.
   \end{enumerate}
\end{ex}
\begin{sol}
   \begin{enumerate}
      \item $I=17$,
      \item $I=\frac{63}{2}$,
      \item $I=-12\pi$.
   \end{enumerate}
\end{sol}


%%%%%%%%%%%%%%%%%%%%%%%%%%%%%%%%%%%%%%%%%%%%%%%%%%%%%%%%%%%%%%%%%%%%%%%%%%%%%%%%
\chapter{Oby�ejn� diferenci�ln� rovnice}
%%%%%%%%%%%%%%%%%%%%%%%%%%%%%%%%%%%%%%%%%%%%%%%%%%%%%%%%%%%%%%%%%%%%%%%%%%%%%%%%

oby�ejn� diferenci�ln� rovnice -- vztah mezi nezn�mou funkc� jedn� prom�nn� a jej�mi derivacemi;
��d ODR -- ��d nejvy��� derivace v ODR.

ODR $n$-t�ho ��du: $F(x,y,y',\ldots,y^{(n)})=0$;
�e�en� (integr�l) ODR -- funkce, kter� vyhovuje rovnici v dan�m oboru.

ODR 1. ��du: $F(x,y,y')=0$, resp. $y'=f(x,y)$ (roz�e�en� vzhledem k derivaci).

\begin{enumerate}
   \item \emph{Rovnice se separovan�mi prom�nn�mi} ve tvaru $f(x)+g(y)y'=0$,
   kde $f,g$ jsou funkce. 
   Je-li $f$ spojit� na $(a,b)$ a $g$ je spojit� na $(c,d)$,
   potom ka�d� �e�en� na $I\subset(a,b)$ spl�uje na $I$ rovnici
   \[
      \int f(x)\,dx + \int g(y)\,dy = C, \; C\in\mathbb{R}.
   \]
   Funkce ur�en� na intervalu $J\subset(c,d)$ touto rovnic�, je �e�en�m diferenci�ln� rovnice na $J$.
   %
   \item \emph{Line�rn� diferenci�ln� rovnice 1. ��du} ve tvaru $y'+f(x)y=g(x)$,
   kde $f,g$ jsou funkce spojit� na $(a,b)$.
   Je-li $g=0$, naz�v� se rovnice homogenn�, v opa�n�m p��pad� je nehomogenn�.
   Homogenn� rovnici �e��me separac� prom�nn�ch; m�me
   \[
      y'+f(x)y = 0.
   \]
   Odtud integrac� dostaneme
   \[
      y(x) = Ce^{-\int f(x)\,dx}, \; C\in\mathbb{R}.
   \]
   Nehomogenn� rovnici �e��me variac� konstanty. Uva�ujeme ho ve tvaru
   \[
      y(x) = C(x)e^{-\int f(x)\,dx},
   \]
   kde $C$ je funkce parametru $x$.
   Dosazen�m do p�vodn� rovnice m�me
   \[
      C'(x) = g(x)e^{\int f(x)\,dx}.
   \]
   Pak tedy
   \[
      y(x)=\left[\int g(x)e^{\int f(x)\,dx}\,+K\right]\,e^{-\int f(x)\,dx}, \;K\in\mathbb{R}
   \]
   je �e�en�m nehomogenn� rovnice na intervalu $(a,b)$.
   %
   \item \emph{Bernoulliho rovnice} (p��klad p�evodu neline�rn� rovnice na line�rn�) ve tvaru 
   $y'+f(x)y=g(x)y^(\alpha)$, $f,g$ funkce spojit� na $(a,b)$, $\alpha\in\mathbb{R}\setminus\{0,1\}$.
   Pro $\alpha>0$ je jedn�m z �e�en� funkce $y=0$.
   Ostatn� �e�en� z�sk�me n�sledovn�.
   N�soben�me-li rovnici $y^{-\alpha}$, dostaneme
   \[
      y^{-\alpha}y' + f(x)y^{1-\alpha} = g(x).
   \]
   Polo��me-li $z=y^{1-\alpha}$, potom $z'=(1-\alpha)y^{-\alpha}y'$.
   Dosazen�m do p�vodn� rovnice tedy m�me
   \[
      \frac{1}{1-\alpha}z' + f(x)z = g(x),
   \]
   co� je line�rn� diferenci�ln� rovnice 1. ��du.
   %
   \item \emph{Line�rn� diferenci�ln� rovnice $n$-t�ho ��du s konstantn�mi koeficienty}
   je rovnice ve tvaru
   \[
      a_ny^{(n)}+a_{n-1}y^{(n-1)}+\cdots+a_1y'+a_0y = f(x),
   \]
   kde $f$ je spojit� na n�jak�m intervalu $(a,b)$, $a_i\in\mathbb{R}$ ($i=0,1,\ldots,n$) a $a_n\neq 0$.
   Na tomto intervalu existuje �e�en� t�to rovnice;
   toto je ur�eno jednozna�n� po��te�n�mi podm�nkami
   $y(x_0)=y_0, \; y'(x_0)=y_0^{1},\; \ldots, \; y^{(n)}(x_0)=y_0^{n}$
   pro n�jak� $x_0\in(a,b)$. 
   Rovnice
   \[
      a_ny^{(n)}+a_{n-1}y^{(n-1)}+\cdots+a_1y'+a_0y = 0
   \]
   se naz�v� homogenn� (je-li $f(x)\neq 0$ n�kde v $(a,b)$, naz�v� se rovnice nehomogenn�).
   �e�en� homogenn� rovnice tvo�� vektorov� podprostor $C(a,b)$
   a existuje $n$ line�rn� nez�visl�ch funkc� (s nenulov�m Wronsk�ho determinantem), 
   kter� jsou jej�m �e�en�m (fundament�ln� syst�m �e�en�).
   Ka�d� �e�en� lze potom jednozna�n� zapsat jako line�rn� kombinaci funkc� fundament�ln�ho syst�mu.
   Nech� $\lambda\in\mathbb{C}$ je ko�en charakteristick� rovnice
   \[
       a_n\lambda^n+a_{n-1}\lambda^{n-1}+\cdots+a_1\lambda+a_0 = 0
   \]
   s n�sobnost� $r$.
   Je-li $\lambda\in\mathbb{R}$, jsou funkce
   \[
      S(\lambda)=\{x^ie^{\lambda x}\,| \; i=0,1,\ldots,r-1\}
   \]
   �e�en�m homogenn� rovnice.
   Je-li $\lambda\in\mathbb{C}$ ko�en charakteristick� rovnice
   (a tedy i $\bar{\lambda}$ je ko�enem), kde $\lambda=\sigma+\mathrm{i}\omega$ s $\sigma,\omega\in\mathbb{R}$, 
   jsou funkce
   \[
      S(\lambda)=\{x^ie^{\sigma x}\cos\omega x,\;x^ie^{\sigma x}\sin\omega x\,|\;i=0,1,\ldots,r-1\}
   \]
   rovn� �e�en�m homogenn� rovnice (p�esn�ji re�ln� �e�en�).
   Mno�ina
   \[
      S = \bigcup\,\{S(\lambda)\,|\,\lambda\text{ je ko�en charakteristick� rovnice}\}
   \]
   tvo�� fundament�ln� syst�m �e�en� homogenn� rovnice.
   �e�en� nehomogenn� rovnice (tj. rovnice s nenulovou pravou stranu) hled�me
   nap�. jako v p��pad� �e�en� line�rn� rovnice prvn�ho ��du variac� konstant. 
   Sta�� nal�zt alespo� jedno �e�en�, nebo� �e�en� nehomogenn� rovnice lze ps�t
   jako sou�et libovoln�ho �e�en� nehomogenn� rovnice (partikul�rn� �e�en�)
   a funkce, kter� je line�rn� kombinac� fundament�ln�ho syst�mu �e�en� homogenn� rovnice. 
   Uva�ujme pravou stranu ve speci�ln�m tvaru $p(x)e^{\lambda x}$,
   kde $p$ je polynom v $x$. Je-li $\lambda$ $r$-n�sobn� ko�en charakteristick� rovnice ($r\geq 0$)
   a $p$ je stupn� $k$, pak hled�me �e�en� ve tvaru $t^rP(x)e^{\lambda x}$,
   kde $P$ je polynom v $x$ rovn� stupn� $k$.
   M�-li prav� strana tvar $p(x)e^{\sigma x}\cos\omega x+q(x)e^{\sigma x}\sin\omega x$,
   kde $p$ a $q$ jsou polynomy v $x$ stupn� nejv��e $k$,
   a je-li $\sigma+\mathrm{i}\omega$ (a tedy rovn� $\sigma-\mathrm{i}\omega$)
   $r$-n�sobn� ko�en charakteristick� rovnice ($r\geq 0$),
   hled�me �e�en� ve tvaru $t^rP(x)e^{\sigma x}\cos\omega x+t^rQ(x)e^{\sigma x}\sin\omega x$,
   kde $P$ a $Q$ jsou polynomy prom�nn� $x$ nejv��e $k$.
\end{enumerate}

\section*{�e�en� p��klady}

\begin{ex}
   �e�te rovnici $xy'=2y+y'$.
\end{ex}
\begin{sol}
   Rovnici p�evedeme do ekvivalentn�ho tvaru $(x-1)y'=2y$ (jedn� se o rovnici se separovateln�mi prom�nn�mi,
   tj. rovnici, kterou lze p�ev�st na rovnici se separovan�mi prom�nn�mi).
   Nech� $y\neq 0$ (nulov� �e�en� je �e�en�m rovnice) a $x\in(1,\infty)$.
   Pak dostaneme rovnici $y'/y=2/(x-1)$. Funkce $1/y$ je spojit� v $(0,\infty)$ (rovn� v $(-\infty,0)$)
   a $1/(x-1)$ je spojit� na uva�ovan�m intervalu $(1,\infty)$.
   Integrac� m�me
   \[
      \ln|y| = \ln|x-1|+C, \; C\in\mathbb{R}.
   \]
   Odlogaritmov�n�m dostaneme
   \[
      |y| = C(x-1)^2, \; C>0.
   \]
   Proto�e uva�ujeme $y>0$ a prav� strana je kladn�, m�me �e�en�
   \[
      y(x) = C(x-1)^2, \; C>0.
   \] 
   Uva�ujeme-li $y<0$, dostaneme �e�en�
   \[
      y(x) = -C(x-1)^2, \;C>0.
   \]
   Proto�e nulov� funkce je rovn� �e�en�m p�vodn� diferenci�ln� rovnice,
   jej� �e�en� na intervalu $(1,\infty)$ maj� tedy tvar
   \[
      y(x) = C(x-1)^2, \; C\in\mathbb{R}.
   \]
   (Podobn� m��eme dostat �e�en� na intervalu $(-\infty,1)$.)
\end{sol}

\begin{ex}
   �e�te rovnici $y'-y\,\mathrm{cotg}\,x=e^x\,\sin\,x$.
\end{ex}
\begin{sol}
   Jedn� se o nehomogenn� line�rn� rovnici prvn�ho ��du.
   Nejprve nalezneme �e�en� homogenn� rovnice $y'-y\,\mathrm{coth}\,x=0$.
   Funkce $\mathrm{cotg}\,x$ je spojit� na ka�d�m intervalu ve tvaru
   $(k\pi,(k+1)\pi)$, kde $k\in\mathrm{Z}$.
   Pro jednoduchost uva�ujme $x\in(0,\pi)$.
   Integrac� dostaneme
   \[
      \ln |y| = \ln |\sin x|+C, \; C\in\mathrm{R}.
   \]
   Odlogaritmov�n�m je
   \[
      |y| = C\sin x, \; C>0
   \]
   Proto�e prav� strana je kladn� a nulov� �e�en� je rovn� �e�en�m homogenn� rovnice, 
   m�me �e�en� homogenn� rovnice ve tvaru
   \[
      y(x) = C\sin x,\; C\in\mathrm{R}.
   \]
   �e�en� nehomogenn� rovnice nalezneme variac� konstanty, �ili polo�me $C\equiv C(x)$.
   Dosazen�m �e�en� ve tvaru $y(x)=C(x)\sin x$ do p�vodn� rovnice m�me $C'(x) = e^x$,
   odkud $C(x) = e^x+K$, $K\in\mathbb{R}$.
   Obecn� �e�en� nehomogenn� rovnice na intervalu $(0,\pi)$ je tedy
   \[
      y(x) = (e^x+K)\sin x, \; K\in\mathbb{R}.
   \]
\end{sol}

\begin{ex}
   �e�te rovnici $y''-y=x^2e^x$ s po��te�n�mi podm�nkami $y(0)=1$, $y'(0)=0$.
\end{ex}
\begin{sol}
   Ko�eny charakteristick� rovnice $\lambda^2-1=0$ jsou $\lambda_1=1$, $\lambda_2=-1$,
   tak�e fundament�ln� syst�m �e�en� p��slu�n� homogenn� rovnice je tvo�en funkcemi $e^x$ a $e^{-x}$.
   �e�en� nehomogenn� rovnice hled�me ve tvaru $y_p(x)=x(ax^2+bx+c)e^x$, kde $a,b,c\in\mathbb{R}$.
   Derivov�n�m dostaneme
   \[
      (y''_p-y_p)(x) = (6ax^2+(6a+4b)x+2b+2c)e^x
   \]
   a porovn�n�m s pravou stranou m�me podm�nky $6a=1$, $6a+4b=0$ a $2b+2c=0$, odkud
   $a=1/6$, $b=-1/4$ a $c=1/4$, tak�e obecn� �e�en� nehomogenn� rovnice m� tvar
   \[
      y(x) = \left(\frac{1}{6}x^2-\frac{1}{4}x+\frac{1}{4}\right)e^x+c_1e^x+c_2e^{-x}, \; c_1,c_2\in\mathbb{R}.
   \]
   Konstanty $c_1$ a $c_2$ ur��me z po��te�n�ch podm�nek.
   Derivac� je
   \[
      y'(x) = \left(\frac{1}{6}x^2+\frac{1}{12}x\right)e^x+c_1e^x-c_2e^{-x},
   \]
   tak�e m� platit $c_1+c_2=-1/4$ a $c_1-c_2=0$, odkud $c_1=c_2=-1/8$.
   �e�en�m tedy je funkce
   \[
      y(x) = \left(\frac{1}{6}x^2-\frac{1}{4}x+\frac{1}{8}\right)e^x-\frac{1}{8}e^{-x}.
   \]
\end{sol}

\begin{ex}
   �e�te rovnici $y''-2y'+2y=\cos x$.% s po��te�n�mi podm�nkami $y(0)=1$, $y'(0)=0$.
\end{ex}
\begin{sol}
   Ko�eny charakteristick� rovnice $\lambda^2-2\lambda+2=0$ jsou $\lambda_1=1+\mathrm{i}$, $\lambda_2=1-\mathrm{i}$,
   tak�e fundament�ln� syst�m �e�en� p��slu�n� homogenn� rovnice je tvo�en funkcemi 
   $e^x\cos x$ a $e^x\sin x$.
   Proto�e $0+\mathrm{i}$ nen� ko�enem charakteristick� rovnice (resp. je $0$-n�sobn�m ko�enem),
   hled�me �e�en� nehomogenn� rovnice ve tvaru $y_p(x)=a\cos x+b\sin x$, kde $a,b\in\mathbb{R}$.
   Derivov�n�m dostaneme
   \[
      (y''_p-2y'_p+2y_p)(x) = (a-2b)\cos x+(2a+b)\sin x
   \]
   a porovn�n�m s pravou stranou m�me podm�nky $a-2b=1$ a $2a+b=0$, odkud
   $a=1/5$ a $b=-2/5$, tak�e obecn� �e�en� nehomogenn� rovnice m� tvar
   \[
      y(x) = \left(c_1e^x+\frac{1}{5}\right)\cos x+\left(c_2e^x-\frac{2}{5}\right)\sin x, \; c_1,c_2\in\mathbb{R}.
   \]
\end{sol}

\section*{Ne�e�en� p��klady}

\begin{ex}
   �e�te diferenci�ln� rovnice
   \begin{enumerate}
      \item $yy'=e^y$,
      \item $x+yy'=0$,
      \item $(x^2+1)(y^2-1)+xyy'=0$, $y(1)=\sqrt{2}$,
      \item $(x+1)y'+xy=0$, $y(0)=1$,      
      \bigskip
      \item $y'+\frac{y}{x}=x$,
      \item $y'-\frac{2}{x+1}y=(x+1)^3$,
      \bigskip
      \item $xy'-y=x^2y^{-1}$,
      \item $y'+xy=xy^3$,
      \bigskip
      \item $y''+y=(4x+2)\cos x+6\sin x$.
   \end{enumerate}
\end{ex}
\begin{sol}
   \begin{enumerate}
      \item $(1+y)e^{-y}=C-x$, $C\in\mathbb{R}$,
      \item $x^2+y^2=C$, $C\in\mathbb{R}$,
      \item obecn� �e�en� $\ln|y^2-1|+x^2+\ln x^2=C$, $C\in\mathbb{R}$; $C=1$, $y(x)=\sqrt{1+e^{1-x^2}/x^2}$,
      \item $y(x)=(x+1)e^{-x}$,
      %--
      \item $y(x)=\frac{x^2}{3}+\frac{C}{x}$, $x>0$, $C\in\mathbb{R}$,
      \item $y(x)=(\frac{1}{2}x^2+x+C)(x+1)^2$, $C\in\mathbb{R}$,
      %--
      \item $y^2=x^2(\ln\,x^2+C)$, $x>0$, $C\in\mathbb{R}$,
      \item $y(x)=\pm\frac{1}{\sqrt{1+Ce^{x^2}}}$, $C\in\mathbb{R}$,
      %--
      \item $y(x)=(c_1-2x)\cos x + (c_2+x^2+x)\sin x$
   \end{enumerate}
\end{sol}


\chapter{Soustavy oby�ejn�ch diferenci�ln�ch rovnic}
%%%%%%%%%%%%%%%%%%%%%%%%%%%%%%%%%%%%%%%%%%%%%%%%%%%%%%%%%%%%%%%%%%%%%%%%%%%%%%%%%%%%%%%%%%%%%%%%%%%%
%%%%%%%%%%%%%%%%%%%%%%%%%%%%%%%%%%%%%%%%%%%%%%%%%%%%%%%%%%%%%%%%%%%%%%%%%%%%%%%%%%%%%%%%%%%%%%%%%%%%

Soustavou oby�ejn�ch diferenci�ln�ch rovnic prvn�ho ��du rozum�me (vektorovou) rovnici ve tvaru
\[
   \vec{y}' = \vec{f}(x,\vec{y}),
\]
kde $\vec{f}:\mathbb{R}\times\mathbb{R}^n\rightarrow\mathbb{R}^n$, $\vec{y}=(y_1,\ldots,y_n)^T$, $n\in\mathbb{N}$.

Definujeme derivaci vektorov� funkce vztahem $\vec{y}'=(y'_1,\ldots,y'_n)^T$.

\paragraph{Soustavy line�rn�ch diferenci�ln�ch rovnic prvn�ho ��du s konstantn� matic�}
%%%%%%%%%%%%%%%%%%%%%%%%%%%%%%%%%%%%%%%%%%%%%%%%%%%%%%%%%%%%%%%%%%%%%%%%%%%%%%%%%%%%%%%
V t�to kapitole se budeme zab�vat �e�en�m soustavy line�rn�ch diferenci�ln�ch rovnic prvn�ho ��du
s konstantn� matic�, kter� m� tvar
\begin{equation}\label{eq:slodr}
   \vec{y}' = \vec{A}\vec{y} + \vec{b}(x), \; \vec{A}\in\mathbb{R}^{n,n}, \; \vec{b}:\,\mathbb{R}\rightarrow\mathbb{R}^n,
\end{equation}
kde $\vec{b}$ je spojit� vektorov� fukce na $(a,b)$.
Tato soustava m� na $(a,b)$ �e�en�,
kter� je jednozna�n� ur�eno po��te�n� podm�nkou $\vec{y}(x_0)=\vec{y}_0$, kde $x_0\in(a,b)$.

\paragraph{Homogenn� soustavy s konstantn�mi koeficienty}
%%%%%%%%%%%%%%%%%%%%%%%%%%%%%%%%%%%%%%%%%%%%%%%%%%%%%%%%%
Homogenn� soustava p��slu�n� soustav�~(\ref{eq:slodr}) m� tvar
\begin{equation}\label{eq:hslodr}
   \vec{y}'=\vec{A}\vec{y}.
\end{equation}
�e�en�~(\ref{eq:hslodr}) tvo�� vektorov� prostor.

P�edpokl�dejme, �e matice $\vec{A}$ je diagonalizovateln�, tj. m� �pln� syst�m vlastn�ch vektor�.
Je-li $\lambda\in\mathbb{R}$ re�ln� vlastn� ��slo matice $\vec{A}$ a $\vec{u}\in\mathbb{R}^n$ 
je jemu odpov�daj�c� vlastn� vektor, pak funkce
\[
   \vec{v}(x) = \vec{u}e^{\lambda x}
\]
je �e�en�m rovnice~(\ref{eq:hslodr}).
Je-li $\lambda\in\mathbb{C}\setminus\mathbb{R}$ komplexn� vlastn� ��slo matice $\vec{A}$
(a tedy rovn� $\bar{\lambda}$ je vlastn� ��slo $\vec{A}$)
a $\vec{u}\in\mathbb{C}^n$ je jemu odpov�daj�c� vlastn� vektor (vlastn�mu ��slu $\bar{\lambda}$ odpov�d�
vlastn� vektor $\bar{\vec{u}}$), pak funkce
\[
   \vec{v}_1(x) = \vec{u}_1 e^{\sigma x}\cos\omega x - \vec{u}_2 e^{\sigma x}\sin\omega x,\;\;
   \vec{v}_2(x) = \vec{u}_1 e^{\sigma x}\sin\omega x + \vec{u}_2 e^{\sigma x}\cos\omega x
\]
jsou re�ln� �e�en� rovnice~(\ref{eq:hslodr}),
kde $\sigma=\Re\lambda$, $\omega=\Im\lambda$, $\vec{u}_1 = \Re\vec{u}$, $\vec{u}_2=\Im\vec{u}$.

Nech� $\vec{V}(x)$, $x\in\mathbb{R}$ je matice obsahuj�c� ve sloupc�ch funkce definovan� v p�edchoz�m odstavci
pro ka�d� vlastn� ��slo matice $\vec{A}$ (p��p. dvojici komplexn� sdru�en�ch vlastn�ch ��sel)
-- tyto funkce tvo�� fundament�ln� syst�m �e�en� soustavy~(\ref{eq:hslodr}).
Plat� tedy
\[
   \vec{V}'(x) = \vec{A}\vec{V}(x), \; x\in\mathbb{R}.
\]
Je-li $\vec{T}\in\mathbb{R}^{n,n}$ libovoln� regul�rn� matice stupn� $n$, pak 
(definujeme-li $\tilde{\vec{V}}(x)\equiv\vec{V}(x)\vec{T}$, $x\in\mathbb{R}$) rovn� plat�
\[
   \tilde{\vec{V}}'(x) = \vec{A}\tilde{\vec{V}}(x), \; x\in\mathbb{R}.
\]
Matice $\vec{V}(x)$ je regul�rn� matice (m� line�rn� nez�visl� sloupce), a tedy
rovn� matice $\tilde{\vec{V}}(x)$.
V�echny takov� matice naz�v�me fundament�ln�mi maticemi soustavy~(\ref{eq:hslodr})
a ka�d� �e�en� soustavy~(\ref{eq:hslodr}) m��eme napsat jako line�rn� kombinaci sloupc�
$\vec{V}(x)$, tj. 
\[
   \vec{y}(x) = \vec{V}(x)\vec{c}, \; x\in\mathbb{R},
\]
pro n�jak� $\vec{c}\in\mathbb{R}^{n}$.

\begin{ex}\label{priklad:1}
   Najd�te fundament�ln� matici soustavy
   \[
      \vec{y}'\equiv
      \begin{pmatrix}
         y_1\\y_2
      \end{pmatrix}'
      =
      \begin{pmatrix}
         0 & 1 \\ -2 & -3
      \end{pmatrix}
      \begin{pmatrix}
         y_1\\y_2
      \end{pmatrix}
      \equiv \vec{A}\vec{y}
   \]
\end{ex}
\begin{sol}
   Vlastn� ��sla matice $\vec{A}$ jsou $\lambda_1=-1$ a $\lambda_2=-2$
   s odpov�daj�c�mi vlastn�mi vektory $\vec{u}_1=(-1,1)^T$ a $\vec{u}_2=(-1,2)^T$.
   Fundament�ln� syst�m soustavy tedy tvo�� funkce
   \[
      \vec{v}_1(x) = \begin{pmatrix}
         -1 \\ 1
      \end{pmatrix}e^{-x}, \;\;
      \vec{v}_1(x) = \begin{pmatrix}
         -1 \\ 2
      \end{pmatrix}e^{-2x}.
   \]
   Fundament�ln� matice soustavy (resp. jedna z fundament�ln�ch matic) je
   \[
      \vec{V}(x) = \begin{pmatrix}
         -e^{-x} &  -e^{-2x}\\
          e^{-x} &  2e^{-2x}
      \end{pmatrix}.
   \]
\end{sol}

\paragraph{Homogenn� soustavy s konstantn�mi koeficienty -- Cauchyho �loha}
%%%%%%%%%%%%%%%%%%%%%%%%%%%%%%%%%%%%%%%%%%%%%%%%%%%%%%%%%%%%%%%%%%%%%%%%%%%
�e�me Cauchyho �lohu pro soustavu~(\ref{eq:hslodr}), tj. soustavu s podm�nkou $\vec{y}(x_0)=\vec{y}_0$.
Ka�d� �e�en� soustavy~(\ref{eq:hslodr}) lze ps�t ve tvaru $\vec{y}(x)=\vec{V}(x)\vec{c}$,
kde $\vec{c}\in\mathbb{R}^n$.
U�it�m po��te�n� podm�nky m�me $\vec{y}_0=\vec{V}(x_0)\vec{c}$, odkud $\vec{c} = \vec{V}^{-1}(x_0)\vec{y}_0$,
tak�e
\[
   \vec{y}(x) = \vec{U}(x)\vec{y}_0
\]
je �e�en�m po��te�n� �lohy.
Matice $\vec{U}(x) = \vec{V}(x)\vec{V}^{-1}(x_0)$ se naz�v� standardn� fundament�ln� matice
p��slu�n� Cauchyho �loze~(\ref{eq:hslodr}) s po��te�n� podm�nkou $\vec{y}(x_0)=\vec{y}_0$.

\begin{ex}\label{priklad:2}
   Najd�te standardn� fundament�ln� matici soustavy
   \[
      \vec{y}'\equiv
      \begin{pmatrix}
         y_1\\y_2
      \end{pmatrix}'
      =
      \begin{pmatrix}
         0 & 1 \\ -2 & -3
      \end{pmatrix}
      \begin{pmatrix}
         y_1\\y_2
      \end{pmatrix}
      \equiv \vec{A}\vec{y}
   \]
   a najd�te �e�en� soustavy s po��te�n� podm�nkou $\vec{y}_0 = (1,1)^T$.
\end{ex}
\begin{sol}
   Jedna z fundament�ln�ch matic soustavy je podle p��kladu~(\ref{priklad:1})
   \[
      \vec{V}(x) = \begin{pmatrix}
         -e^{-x} &  -e^{-2x}\\
          e^{-x} &  2e^{-2x}
      \end{pmatrix}.
   \]
   D�le je
   \[
      \vec{V}(0) = \begin{pmatrix}
         -1 & -1\\
          1 &  2
      \end{pmatrix},
   \]
   odkud
   \[
      \vec{V}^{-1}(0) = \begin{pmatrix}
         -2 & -1\\
          1 &  1
      \end{pmatrix},
   \]
   tak�e
   \[
      \begin{split}
      \vec{U}(x) = \vec{V}(x)\vec{V}^{-1}(0) &= 
      \begin{pmatrix}
         -e^{-x} &  -e^{-2x}\\
          e^{-x} &  2e^{-2x}
      \end{pmatrix}
      \begin{pmatrix}
         -2 & -1\\
          1 &  1
      \end{pmatrix}\\
      &=
      \begin{pmatrix}
         2e^{-x}-e^{-2x} & e^{-x}-e^{-2x}\\
         -2e^{-x}+2e^{-2x} & -e^{-x}+2e^{-2x}
      \end{pmatrix}.
      \end{split}
   \]
   �e�en�m Cauchyho �lohy je tedy funkce
   \[
      \vec{y}(x) = \vec{U}(x)\vec{y}_0
      =
      \begin{pmatrix}
         3e^{-x}-2e^{-2x}\\
         -3e^{-x}+4e^{-2x}
      \end{pmatrix}.
   \]
\end{sol}

\begin{ex}
   �e�te soustavu
   \[
      \vec{y}'\equiv
      \begin{pmatrix}
         y_1\\y_2\\y_3
      \end{pmatrix}'
      =
      \begin{pmatrix}
         0 & 1 & 0\\ 0 & 0 & 1\\ 0 & -1 & 0
      \end{pmatrix}
      \begin{pmatrix}
         y_1\\y_2\\y_3
      \end{pmatrix}
      \equiv \vec{A}\vec{y}
   \]
   s po��te�n� podm�nkou $\vec{y}_0=(1,1,1)^T$.
\end{ex}
\begin{sol}
   Vlastn� ��sla matice $\vec{A}$ jsou $\lambda_1=0$, $\lambda_2=\mathrm{i}$ a $\lambda_3=-\mathrm{i}$.
   Odpov�daj�c� vlastn� vektory jsou
   \[
      \vec{u}_1 = \begin{pmatrix}1\\0\\0\end{pmatrix}, \;\;
      \vec{u}_2 = \begin{pmatrix}-1\\-i\\1\end{pmatrix}, \;\;
      \vec{u}_3 = \begin{pmatrix}-1\\i\\1\end{pmatrix}.
   \]
   Funkce fundament�ln�ho syst�mu jsou
   \[
      \vec{v}_1(x) = \begin{pmatrix}
         1 \\ 0 \\ 0
      \end{pmatrix},\;
      \vec{v}_2(x) = \begin{pmatrix}
         -\cos x \\ \sin x \\ \cos x
      \end{pmatrix},\;
      \vec{v}_3(x) = \begin{pmatrix}
         -\sin x \\ -\cos x \\ \sin x
      \end{pmatrix}.
   \]
   Fundament�ln� matice je
   \[
      \vec{V}(x) = \begin{pmatrix}
         1 & -\cos x & -\sin x\\
         0 & \sin x  & -\cos x\\
         0 & \cos x  & \sin x
      \end{pmatrix}.
   \]
   Odtud
   \[
      \vec{V}(0) = \begin{pmatrix}
         1 & -1 & 0\\
         0 & 0  &-1\\
         0 & 1  & 0
      \end{pmatrix},
   \]
   tak�e
   \[
      \vec{V}^{-1}(0) = \begin{pmatrix}
         1 & 0 & 1\\
         0 & 0  &1\\
         0 & -1  & 0
      \end{pmatrix},
   \]
   odkud dostaneme standardn� fundament�ln� matici ve tvaru
   \[
      \begin{split}
      \vec{U}(x) = \vec{V}(x)\vec{V}^{-1}(0) &=
      \begin{pmatrix}
         1 & -\cos x & -\sin x\\
         0 & \sin x  & -\cos x\\
         0 & \cos x  & \sin x
      \end{pmatrix}
      \begin{pmatrix}
         1 & 0 & 1\\
         0 & 0  &1\\
         0 & -1  & 0
      \end{pmatrix}\\
      &=
      \begin{pmatrix}
         1 & \sin x  & 1-\cos x\\
         0 & \cos x  & \sin x\\
         0 & -\sin x & \cos x
      \end{pmatrix}.
      \end{split}
   \]
   �e�en� Cauchyho �lohy je tedy
   \[
      \vec{y}(x) = \vec{U}(x)\vec{y}_0
      =
      \begin{pmatrix}
         2 + \sin x  -\cos x\\
         \cos x  + \sin x\\
         \cos x - \sin x
      \end{pmatrix}.
   \]
\end{sol}

\paragraph{Nehomogenn� soustavy s konstantn� matic�}
%%%%%%%%%%%%%%%%%%%%%%%%%%%%%%%%%%%%%%%%%%%%%%%%%%%%
M�jme (nehomogenn�) soustavu line�rn�ch diferenci�ln�ch rovnic prvn�ho ��du s konstantn� matic� ve tvaru
\begin{equation}\label{eq:nhslodr}
   \vec{y}' = \vec{A}\vec{y} + \vec{b}(x), \; \vec{A}\in\mathbb{R}^{n,n}, \; \vec{b}:\,\mathbb{R}\rightarrow\mathbb{R}^n,
\end{equation}
kde $\vec{b}$ je spojit� vektorov� fukce na $(a,b)$.
�e�en� nehomogenn� rovnice pak m��eme vyj�d�it jako sou�et partikul�rn�ho �e�en� 
(tj. libovoln�ho �e�en� rovnice~(\ref{eq:nhslodr})) a �e�en� homogenn� rovnice, tedy plat�
\[
   \vec{y}(x) = \vec{y}_p(x) + \vec{U}(x)\vec{c}, \; x\in(a,b),
\]
kde $\vec{U}(x)$ je standardn� fundament�ln� matice p��slu�n� homogenn� soustavy, $\vec{c}\in\mathbb{R}^n$ a
\[
   \vec{y}_p(x) = \int\limits_{x_0}^x\vec{U}(x-\xi)\vec{b}(\xi)\,d\xi.
\]

\begin{ex}
   Najd�te �e�en� soustavy
   \[
      \vec{y}'\equiv
      \begin{pmatrix}
         y_1\\y_2
      \end{pmatrix}'
      =
      \begin{pmatrix}
         0 & 1 \\ -2 & -3
      \end{pmatrix}
      \begin{pmatrix}
         y_1\\y_2
      \end{pmatrix}
      +
      \begin{pmatrix}
         1 \\ 2
      \end{pmatrix}
      e^{x}
      \equiv \vec{A}\vec{y}
   \]
   a najd�te �e�en� soustavy s po��te�n� podm�nkou $\vec{y}_0 = (1,1)^T$.
\end{ex}
\begin{sol}
   Z p��kladu~(\ref{priklad:2}) m�me standardn� fundament�ln� matici homogenn� soustavy ve tvaru
   \[
      \vec{U}(x) = 
      \begin{pmatrix}
         -1 &  -1\\
         1 &  2
      \end{pmatrix}
      \begin{pmatrix}
         e^{-x} &  0\\
         0 &  e^{-2x}
      \end{pmatrix}
      \begin{pmatrix}
         -2 & -1\\
          1 &  1
      \end{pmatrix}.
   \]
   Odtud
   \[
      \begin{split}
      \vec{U}^{-1}(x) &= 
      \begin{pmatrix}
         -2 & -1\\
          1 &  1
      \end{pmatrix}^{-1}
      \begin{pmatrix}
         e^{-x} &  0\\
         0 &  e^{-2x}
      \end{pmatrix}^{-1}
      \begin{pmatrix}
         -1 &  -1\\
         1 &  2
      \end{pmatrix}^{-1}\\
      &=
      \begin{pmatrix}
         -1 &  -1\\
         1 &  2
      \end{pmatrix}
      \begin{pmatrix}
         e^{x} &  0\\
         0 &  e^{2x}
      \end{pmatrix}
      \begin{pmatrix}
         -2 & -1\\
          1 &  1
      \end{pmatrix}
      \end{split}
   \]
   tak�e
   \[
      \begin{split}
      \int\limits_{0}^x\vec{U}^{-1}(\xi)\vec{b}(\xi)\,d\xi
                   &=\int\limits_{0}^x
                      \begin{pmatrix}
                        2e^{\xi}-e^{2\xi} & e^{\xi}-e^{2\xi} \\ -2e^{\xi}+2e^{2\xi} & -e^{\xi}+2e^{2\xi}
                      \end{pmatrix}
                      \begin{pmatrix}
                         1 \\ 2
                      \end{pmatrix}
                      e^{\xi}
                      \,d\xi\\
                   &=\int\limits_{0}^x
                      \begin{pmatrix}
                         4e^{2\xi}-3e^{3\xi}\\
                         -4e^{2\xi}+6e^{3\xi}
                      \end{pmatrix}
                      \,d\xi
                   =\begin{pmatrix}
                      2e^{2x}-e^{3x}-1\\
                      -2e^{2x}+2e^{3x}
                   \end{pmatrix},
      \end{split}
   \]
   odkud
   \[
      \begin{split}
      \vec{y}_p(x) &= 
      \begin{pmatrix}
         -1 &  -1\\
         1 &  2
      \end{pmatrix}
      \begin{pmatrix}
         e^{-x} &  0\\
         0 &  e^{-2x}
      \end{pmatrix}
      \begin{pmatrix}
         -2 & -1\\
          1 &  1
      \end{pmatrix}
      \begin{pmatrix}
            2e^{2x}-e^{3x}-1\\
            -2e^{2x}+2e^{3x}
      \end{pmatrix}\\
      &=
      \begin{pmatrix}
         e^{x}-2e^{-x}+e^{-2x}\\
         2e^{-x}-2e^{-2x}
      \end{pmatrix}
      \end{split}
   \]
   �e�en� Cauchyho �lohy je tedy
   \[
      \vec{y}(x) = \vec{y}_p(x) + \vec{U}(x)\vec{y}_0
       =\begin{pmatrix}
         e^{x}+e^{-x}-e^{-2x}\\
         -e^{-x}+2e^{-2x}
       \end{pmatrix}
   \]
\end{sol}



\paragraph{Putzerova metoda v�po�tu fundament�ln� matice}
%%%%%%%%%%%%%%%%%%%%%%%%%%%%%%%%%%%%%%%%%%%%%%%%%%%%%%%%% 
Tato metoda je pou�iteln� i v p��pad� nediagonalizovateln� matice $\vec{A}$.
Nech� $\lambda_1,\ldots,\lambda_n\in\mathbb{C}$ jsou vlastn� ��sla matice $\vec{A}$ v�etn� n�sobnosti.
Definujeme matice
\[
   \begin{split}
   \vec{P}_0 = \vec{I},\;
   \vec{P}_1 = (\vec{A}-\lambda_1\vec{I})\vec{P}_0,\;
   \vec{P}_2 = (\vec{A}-\lambda_2\vec{I})\vec{P}_1,\;
   \ldots,\;
   \vec{P}_{n-1} = (A-\lambda_{n-1}I)\vec{P}_{n-2}.
   \end{split}
\]
Nalezn�me funkce $q_i$, $i=1,\ldots,n$ jako �e�en� Cauchyho �loh
\begin{align*}
      q_1' &= \lambda_1q_1,  & q_1(0) &= 1,\\
      q_2' &= \lambda_2q_2+q_1,  & q_2(0) &= 0,\\
      &\cdots\\
      q_n' &= \lambda_nq_n+q_{n-1},  & q_n(0) &= 0.
\end{align*}
Matice
\[
  \vec{U}(x) = q_1(x)\vec{P}_0 + q_2(x)\vec{P}_1 + \cdots + q_n(x)\vec{P}_{n-1}
\]
je standardn� fundament�ln� matic� soustavy~(\ref{eq:hslodr}).

\begin{ex}
   Najd�te �e�en� soustavy
   \[
      \vec{y}'\equiv
      \begin{pmatrix}
         y_1\\y_2
      \end{pmatrix}'
      =
      \begin{pmatrix}
         7 & -18 \\ 3 & -8
      \end{pmatrix}
      \begin{pmatrix}
         y_1\\y_2
      \end{pmatrix}
      +
      \begin{pmatrix}
         12 \\ 5
      \end{pmatrix}
      e^{-x}
      \equiv \vec{A}\vec{y}+\vec{b}(x)
   \]
   s podm�nkou $\vec{y}_0=(2,1)^T$.
\end{ex}
\begin{sol}
   Vlastn� ��sla matice $\vec{A}$ jsou $\lambda_1 = 1$ a $\lambda_2 = -2$.
   Polo�me
   \[
      \vec{P}_0 = \begin{pmatrix}
         1 & 0 \\ 0 & 1
      \end{pmatrix}, \;\;
      \vec{P}_1 = \begin{pmatrix}
         6 & -18 \\ 3 & -9
      \end{pmatrix}.
   \]
   D�le �e�me rovnice $q'_1=q'_1$ s podm�nkou $q_1(0)=1$; odtud $q_1(x)=e^{x}$;
   $q'_2=-2q_2+e^x$ s podm�nkou $q_2(0)=0$; odtud $q_2(x)=\frac{1}{3}(e^x-e^{-2x})$.
   Odtud
   \[
      \vec{U}(x)=q_1(x)\vec{P}_0+q_2(x)\vec{P}_1 = 
      \begin{pmatrix}
         3 & -6 \\ 1 & -2
      \end{pmatrix}
      e^x
      +
      \begin{pmatrix}
         -2 & 6 \\ -1 & 3
      \end{pmatrix}
      e^{-2x},
   \]
   tak�e
   \[
     \begin{split}
        \vec{y}(x) &= \vec{y}_p + \vec{U}(x)\vec{y}_0\\
                   &= \int\limits_0^x\vec{U}(x-\xi)\vec{b}(\xi)+\vec{U}(x)\vec{y}_0\\
                   &= \int\limits_0^x\left(
                         \begin{pmatrix}3 & -6 \\ 1 & -2\end{pmatrix}e^{x-\xi}+\begin{pmatrix}-2 & 6 \\ -1 & 3\end{pmatrix}e^{-2x+2\xi}\right)
                         \begin{pmatrix}12 \\ 5\end{pmatrix}e^{-\xi}\,d\xi\\
                    &+\left(
                           \begin{pmatrix}
                              3 & -6 \\ 1 & -2
                           \end{pmatrix}
                           e^x
                           +
                           \begin{pmatrix}
                              -2 & 6 \\ -1 & 3
                           \end{pmatrix}
                           e^{-2x}
                        \right)
                        \begin{pmatrix}
                        2 \\ 1
                        \end{pmatrix}\\
                   &= \int\limits_0^x\left(
                         \begin{pmatrix}6\\2\end{pmatrix}e^{x-2\xi}+\begin{pmatrix}6\\3\end{pmatrix}e^{-2x+\xi}\right)
                         \,d\xi
                    +
                        \begin{pmatrix}
                        2 \\ 1
                        \end{pmatrix}
                        e^{-2x}\\
                   &= 
                         \begin{pmatrix}-3\\-1\end{pmatrix}e^{x}(e^{-2x}-1)+\begin{pmatrix}6\\3\end{pmatrix}e^{-2x}(e^x-1)
                    +
                        \begin{pmatrix}
                        2 \\ 1
                        \end{pmatrix}
                        e^{-2x}\\
                   &= 
                         \begin{pmatrix}-3\\-1\end{pmatrix}(e^{-x}-e^x)+\begin{pmatrix}6\\3\end{pmatrix}(e^{-x}-e^{-2x})
                    +
                        \begin{pmatrix}
                        2 \\ 1
                        \end{pmatrix}
                        e^{-2x}\\
                   &= 
                        \begin{pmatrix}
                        -4 \\ -2
                        \end{pmatrix}
                        e^{-2x}
                        +
                        \begin{pmatrix}
                        3 \\ 2
                        \end{pmatrix}
                        e^{-x}
                        +
                        \begin{pmatrix}
                        3 \\ 1
                        \end{pmatrix}
                        e^{x}.
     \end{split}
   \]
\end{sol}
      
      


\paragraph{Mocninn� metoda v��tu fundament�ln� matice}
%%%%%%%%%%%%%%%%%%%%%%%%%%%%%%%%%%%%%%%%%%%%%%%%%%%%%%
Definujeme-li exponent matice $\vec{A}$ jako
\[
   e^{\vec{A}x} \equiv \sum\limits_{k=1}^{\infty}\frac{1}{k!}(\vec{A}x)^k,
\]
lze uk�zat, �e $e^{\vec{A}x}$ je standardn� fundament�ln� matice~(\ref{eq:hslodr}).
Proto�e $p({A})=0$, kde $p$ je charakteristick� polynom matice $\vec{A}$,
lze ka�dou mocninu matice $\vec{A}^i$, $i\geq n$ vyj�d�it jako line�rn� kombinaci
matic $\vec{I}, \; \vec{A},\;\ldots,\;\vec{A}^{n-1}$.
Matici $e^{\vec{A}x}$ lze tedy hledat ve tvaru
\[
   e^{\vec{A}x} = b_0(x)\vec{I}+b_1(x)\vec{A}+\cdots+b_{n-1}(x)\vec{A}^{n-1},
\]
kde $b_i$ ($i=0,\ldots,n-1$) jsou funkce $x$.
Je-li $\lambda_j$ $r$-n�sobn� vlastn� ��slo matice $\vec{A}$,
nalezneme tyto funkce jako �e�en� soustav (p�es v�echna vlastn� ��sla matice $\vec{A}$)
\[
   \begin{split}
   b_0(x)+b_1(x)\lambda_j+\cdots+b_{n-1}(x)\lambda_j^{n-1}&=e^{\lambda_j x},\\
   b_1(x)+\cdots+(n-1)b_{n-1}(x)\lambda_j^{n-2}&=xe^{\lambda_j x},\\
     &\cdots\\   
   (r-1)!b_{r-1}+\cdots+\frac{(n-1)!}{(n-r-1)!}b_{n-1}(x)\lambda_j^{n-r-1}&=x^{r-1}e^{\lambda_jx}
   \end{split}
\]

\begin{ex}
   Najd�te �e�en� soustavy
   \[
      \vec{y}'\equiv
      \begin{pmatrix}
         y_1\\y_2
      \end{pmatrix}'
      =
      \begin{pmatrix}
         1 & 1 \\ 0 & 1
      \end{pmatrix}
      \begin{pmatrix}
         y_1\\y_2
      \end{pmatrix}
      +
      \begin{pmatrix}
         1 \\ 2
      \end{pmatrix}
      e^{x}
      \equiv \vec{A}\vec{y}+\vec{b}(x)
   \]
   s podm�nkou $\vec{y}_0=(1,1)^T$.
\end{ex}
\begin{sol}
   Matice $\vec{A}$ m� dvojn�sobn� vlastn� ��slo $\lambda=1$.
   Standardn� fundament�ln� matici hled�me ve tvaru
   \[
      \vec{U}(x) = b_0(x)\vec{I} + b_1(x)\vec{A},
   \]
   kde $b_0$ a $b_1$ jsou �e�en�m soustavy
   \[
      \begin{split}
         b_0(x) + b_1(x) &= e^{x},\\
                  b_1(x) &= xe^{x}.
      \end{split}
   \]
   Odtud
   \[
      b_0(x) = (1-x)e^x, \; b_1(x) = xe^x
   \]
   a tedy
   \[
      \vec{U}(x) = \begin{pmatrix}1&x\\0&1\end{pmatrix}e^x.
   \]
   Nakonec
   \[
     \begin{split}
        \vec{y}(x) &= \vec{y}_p + \vec{U}(x)\vec{y}_0\\
                   &= \int\limits_0^x\vec{U}(x-\xi)\vec{b}(\xi)+\vec{U}(x)\vec{y}_0\\
                   &= \int\limits_0^x
                         \begin{pmatrix}1&x-\xi\\0&1\end{pmatrix}e^{x-\xi}
                         \begin{pmatrix}1 \\ 2\end{pmatrix}e^{\xi}\,d\xi
                    +\begin{pmatrix}1&x\\0&1\end{pmatrix}e^x
                        \begin{pmatrix}
                        1\\1
                        \end{pmatrix}\\
                   &= \int\limits_0^x
                         \begin{pmatrix}1+2x-2\xi\\2\end{pmatrix}e^{x}
                         \,d\xi
                    +\begin{pmatrix}1+x\\1\end{pmatrix}e^x\\
                   &= 
                     \begin{pmatrix}x^2+x\\2x\end{pmatrix}e^{x}
                    +\begin{pmatrix}1+x\\1\end{pmatrix}e^x
                   = 
                     \begin{pmatrix}x^2+2x+1\\2x+1\end{pmatrix}e^{x}.
     \end{split}
   \]
\end{sol}



\section{P��klady}

\begin{ex}
   Hledejte standardn� fundament�ln� matice soustav $\vec{y}'=\vec{A}\vec{y}$, kde
   \begin{enumerate}
      \item
      \[
         \vec{A} = \begin{pmatrix}
            0 & 1\\ -1 & 0
         \end{pmatrix},
      \]
      \item
      \[
         \vec{A} = \begin{pmatrix}
            1 & 1 \\ 0 & 1
         \end{pmatrix},
      \]
      \item
      \[
         \vec{A} = \begin{pmatrix}0 & 1 \\ -2 & -3\end{pmatrix},
      \]
      \item
      \[
         \vec{A} = \begin{pmatrix}2 & 1 \\ 0 & 2\end{pmatrix},
      \]
      \item
      \[
         \vec{A} = \begin{pmatrix}1 & 2 \\ 2 & 1\end{pmatrix}.
      \]
   \end{enumerate}
\end{ex}
\begin{sol}
   \begin{enumerate}
      \item
      \item
      \item
      \[
         \vec{Y}_S(x) = \begin{pmatrix}2e^{-x}-e^{-2x}&e^{-x}-e^{-2x}\\-2e^{-x}+2e^{-2x}&-e^{-x}+2e^{-2x}\end{pmatrix}.
      \]
      \item
      \[
         \vec{Y}_S(x) = \begin{pmatrix}e^{2x}&xe^{2x}\\0&e^{2x}\end{pmatrix},
      \]
      \item
      \[
         \vec{Y}_S(x) = \frac{1}{2}\begin{pmatrix}e^{-x}+e^{3x}&-e^{-x}+e^{3x}\\-e^{-x}+e^{3x}&e^{-x}+e^{3x}\end{pmatrix}.
      \]
   \end{enumerate}
\end{sol}

\paragraph{P�evod line�rn� diferenci�ln� rovnice $n$-t�ho ��du s konstantn�mi koeficienty
           na soustavu diferenci�ln�ch rovnic prvn�ho ��du}
%%%%%%%%%%%%%%%%%%%%%%%%%%%%%%%%%%%%%%%%%%%%%%%%%%%%%%%%%%%%%%%%%%%%%%%%%%%%%%%%%%%%%%%%%

Uva�ujme rovnici
\[
   y^{(n)} + a_{n-1}y^{(n-1)} + \cdots + a_1 y' + a_0 y = g(x), \; x \in (a,b).
\]
Polo�me 
\[
   y_1=y, \; y_2=y', \; y_3=y'', \ldots, y_{n-1} = y^{(n-2)}, \; y_{n} = y^{(n-1)}.
\]
Pak
\[
   \begin{split}
      y'_1 &= y_2,\\
      y'_2 &= y_3,\\
           &\cdots\\
      y'_{n-1} &= y'_n,\\
      y'_n &= g(x) - a_{n-1}y_n - \cdots - a_1 y_2 - a_0 y_1.
   \end{split}
\]
P�vodn� rovnici jsme p�evedli na ekvivalentn� soustavu
\[
   \begin{pmatrix}
      y_1 \\ y_2 \\\vdots \\ y_{n-1} \\ y_n
   \end{pmatrix}'
   =
   \begin{pmatrix}
         0 & 1 & & \\
         0 & & \ddots & & \\
         \vdots & & & \ddots & \\
         0 & \cdots & \cdots & \cdots &        1\\
      -a_0 & -a_1   & \cdots & \cdots & -a_{n-1}
   \end{pmatrix}
   \begin{pmatrix}
      y_1 \\ y_2 \\ \vdots \\ y_{n-1} \\ y_n
   \end{pmatrix}
   +
   \begin{pmatrix}
      0 \\ 0 \\ \vdots \\ 0 \\ g(x)
   \end{pmatrix}.
\]

\begin{ex}
   Rovnici $y''-y=x^2e^x$ p�eve�te na soustavu rovnic prvn�ho ��du.
\end{ex}
\begin{sol}
   Polo��me
   \[
      y_1 = y, \; y_2 = y'.
   \]
   Potom je
   \[
      \begin{split}
         y_1' &= y_2\\
         y_2' &= x^2e^x+y_1.
      \end{split}
   \]
\end{sol}

\chapter{Metrick� prostory}
%%%%%%%%%%%%%%%%%%%%%%%%%%%%%%%%%%%%%%%%%%%%%%%%%%%%%%%%%%%%%%%%%%%%%%%%%%%%%%%%

Metrick� prostory jsou speci�ln� typy mno�in, na nich� lze definovat tzv.
metriku, resp. vzd�lenost, pro ka�dou dvojici prvk� z t�to mno�iny.
Form�ln�: nech� $X\neq\emptyset$, $\varrho:\,X\times X\rightarrow\mathbb{R}$,
zobrazen�� $\varrho$ nech� m� n�sleduj�c�� vlastnosti:
\begin{enumerate}
\renewcommand{\theenumi}{M\arabic{enumi}}
   \item $(\forall x,y\in X)(\varrho(x,y)\geq 0\;\&\;\varrho(x,y)=0\; 
                 \Leftrightarrow\;x=y)$,\label{item:m1}
   \item $(\forall x,y\in X)(\varrho(x,y)=\varrho(y,x))$ (symetrie),
   \item $(\forall x,y,z\in X))(\varrho(x,z)\leq\varrho(x,y)+\varrho(y,z))$  
          (troj�heln��kov� nerovnost).\label{item:m3}
\end{enumerate}
Zobrazen� $\varrho$ potom ��k�me metrika (na $X$), dvojici $(X,\varrho)$ 
naz�v�me metrick�m prostorem.
�asto se metrick� prostor zastupuje pouze symbol $X$, je-li z�ejm�, jakou
metriku na n�m m�me definovanou.
Je-li $Y\subset X$, m��eme na $Y$ pou��t metriku definovanou na $X$.
Dvojici $(Y,\varrho)$ ��k�me metrick� podprostor prostoru $(X,\varrho)$.

Pojem metrick�ho prostoru umo��uje zobecnit pojmy zn�m� z anal�zy
jako spojitost, konvergence, jako zobecn�n� otev�en�ch a uzav�en�ch interval�
m��eme definovat otev�en� a uzav�en� mno�iny.

Symbolem $B_r(x)$ budeme ozna�ovat otev�enou kouli se st�edem v bod� $x$
a polom�rem $r$, tj. $B_r(x)\equiv\{y\in X\,|\,\varrho(x,y)<r\}$
($r$-okol� bodu $x$).
Z�le��-li na pou�it� metrice, ozna�ujeme ji $B_r^{\varrho}(x)$.

\begin{ex}
   P��klady metrick�ch prostor�:
   \begin{enumerate}
      \item ��seln� mno�iny ($\mathbb{R}$ re�ln�ch, $\mathbb{C}$ racion�ln�ch,
            $\mathbb{Q}$ racion�ln�ch ��sel) je metrick� prostor s metrikou 
            definovanou pomoc� absolutn� hodnoty $|\cdot|$.
            V dal��m textu budeme uva�ovat zpravidla metrick� prostor re�ln�ch ��sel.
      \item Na mno�in� re�ln�ch $n$-tic, $n\in\mathbb{N}$
            (aritmetick�ch vektor�) lze definovat
            nap�. metriky:
            \[
               \begin{split}
                  \varrho_1(x,y)&\equiv
                           \sum\limits_{i=1}^n|\beta_i-\alpha_i|,\\
                  \varrho_2(x,y)&\equiv
                           \sqrt{\sum\limits_{i=1}^n(\beta_i-\alpha_i)^2},\\
                  \varrho_{\infty}(x,y)&\equiv
                           \max\limits_{i=1,\ldots,n}|\beta_i-\alpha_i|,
               \end{split}
            \]
            kde $x=(\alpha_1,\ldots,\alpha_n),
            y=(\beta_1,\ldots,\beta_n)\in\mathbb{R}^n$.
       \item Analogicky jako na mno�in� aritmetick�ch vektor�
             lze definovat metriky na mno�in� $C(I)$ spojit�ch funkc� 
             na uzav�en�m intervalu $I\subset\mathbb{R}$
            \[
               \begin{split}
                  \varrho_1(f,g)&\equiv
                           \int\limits_I|g(x)-f(x)|\,dx,\\
                  \varrho_2(f,g)&\equiv
                           \sqrt{\int\limits_I(g(x)-f(x))^2\,dx},\\
                  \varrho_{\infty}(f,g)&\equiv
                           \max\limits_{x\in I}|g(x)-f(x)|
               \end{split}
            \]
            pro $f,g\in C(I)$.
        \item Nech� $X$ je nepr�zdn� mno�ina, definujme
              $\varrho(x,y)=0$ pro $x=y$ a $\varrho(x,y)=1$ pro $x\neq y$.
              Prostor $(X,\varrho)$ je (diskr�tn�) metrick� prostor.
   \end{enumerate}
\end{ex}

\section{Mno�iny v metrick�ch prostorech}

Nech� $M,N\subset X$ jsou podmno�iny metrick�ho prostoru $X$.
Definujeme
\begin{itemize}
   \item sjednocen�: $M\cup N\equiv\{x\in X\,|\,x\in M\;\vee\;x\in N\}$;
   \item pr�nik: $M\cap N\equiv\{x\in X\,|\,x\in M\;\&\;x\in N\}$;
   \item rozd�l: $M\setminus N\equiv\{x\in X\,|\,x\in M\;\&\;x\not\in N\}$;
   \item dopln�k: $M^c\equiv\{x\in X\,|\,x\not\in M\}$;
\end{itemize}
z�ejm� plat� $M\setminus N=M\cap N^c$, $M^c = X\setminus M$.

Nech� $M\subset X$ je podmno�ina metrick�ho prostoru $X$ a $x\in X$.
��k�me, �e $x$ je
\begin{itemize}
   \item vnit�n�m bodem mno�iny $M$ 
         $\Leftrightarrow\;(\exists r>0)(B_r(x)\subset M)$;
   \item hrani�n�m bodem mno�iny $M$
         $\Leftrightarrow\;(\forall r>0)(B_r(x)\cap M\neq\emptyset
           \;\&\;B_r(x)\cap M^c\neq\emptyset)$;
   \item hromadn�m bodem mno�iny $M$
         $\Leftrightarrow\;(\forall r>0)((B_r(x)\setminus\{x\})\cap M\neq\emptyset)$;
   \item izolovan�m bodem mno�iny $M$
         $\Leftrightarrow\;(\exists r>0)(B_r(x)\cap M=\{x\})$;
\end{itemize}
Definujme: vnit�ek mno�iny $M$, $M^0$, jako mno�inu v�ech vnit�n�ch bod�;
           hranici mno�iny $M$, $M'$, jako mno�inu v�ech hrani�n�ch bod�;
           uz�v�r mno�iny $M$, $\overline{M}\equiv M\cup\partial M$.

Z�ejm�: ka�d� vnit�n� bod mno�iny le�� v t�to mno�in�  
        a ka�d� bod mno�iny le�� v jej�m uz�v�ru ($M^0\subset M\subset\overline{M}$);
        hrani�n� body a hromadn� body nemus� nutn� le�et v dan� mno�in�;
        ka�d� hrani�n� bod je hromadn�m bodem mno�iny;
        izolovan� bod je hrani�n� bod mno�iny a z�rove� v n� le��;
        v ka�d�m okol� hromadn�ho bodu mno�iny le�� nekone�n� mnoho bod� t�to mno�iny,
        v p��pad� izolovan�ho bodu plat� opak;

Mno�ina $M\subset X$ je
\begin{itemize}
   \item otev�en� $\Leftrightarrow \; M=M^0$,
   \item uzav�en� $\Leftrightarrow \; M=\overline{M}$.
\end{itemize}
Plat�: $M$ je otev�en� $\Leftrightarrow$ $M^c$ je uzav�en�
       a naopak
       (d�sledky de Morganov�ch z�kon�:
        $M\cup N=(M^c\cap N^c)^c$, $M\cap N=(M^c\cup N^c)^c$).

\section{Konvergence v metrick�m prostoru}

O prvku $x\in X$ �ekneme, �e je limitou posloupnosti $\{x_n\}\subset X$,
pokud $\varrho(x_n,x)\rightarrow 0$ pro $n\rightarrow\infty$,
tj. kovergenci prvk� metrick�ho prostoru p�evedeme na konvergenci re�ln� posloupnosti.
Definici m��eme formalizovat n�sledovn�:
v�rok $\varrho(x_n,x)\rightarrow 0$ pro $n\rightarrow\infty$
je ekvivalentn� v�roku
$(\forall \varepsilon>0)(\exists N\in\mathbb{N})(\forall n\in\mathbb{N})
(n>N\;\Rightarrow\;\varrho(x_n,x)<\varepsilon)$.

Z jednozna�nosti limity re�ln� posloupnosti a vlastnosti~\ref{item:m1}
metriky plyne jednozna�nost limity posloupnosti v metrick�m prostoru.

M��eme-li na mno�in� $X$ definovat v�ce metrik, nap�. $\varrho_1$ a $\varrho_2$,
nemus� konvergence v jedn� metrice b�t v�dy ekvivalentn� konvergenci v druh� metrice.
Toto plat� jen pro tzv. ekvivalentn� metriky, tj. metriky, pro n� existuj�
kladn� konstanty $\alpha$ a $\beta$ tak, �e 
$\alpha\varrho_1(x,y)\leq\varrho_2(x,y)\leq\beta\varrho_1(x,y)$.
Metriky $\varrho_1$ a $\varrho_2$ na $X$ jsou ekvivalentn�, pr�v� kdy�
$(\forall x\in X)(\forall r>0)(\exists r_1<r)(\exists r_2>r)
 (B_{r_1}^{\varrho_1}\subset B_r^{\varrho_2}\subset B_{r_2}^{\varrho_1})$.

Mno�ina $M$ je uzav�en�, pokud ka�d� konvergentn� posloupnost z $M$
m� limitu v $M$.

\section{�pln� metrick� prostory}

$\{x_n\}\subset X$ je cauchyovsk� $\Leftrightarrow$
$(\forall\varepsilon>0)(\exists N\in\mathbb{N})(\forall m,n\in\mathbb{N})
(m,n>N\;\Rightarrow\;\varrho(x_m,x_n)<\varepsilon)$.
Metrick� prostor $(X,\varrho)$ nazveme �pln�, pokud ka�d� cauchyovsk� posloupnost
$\{x_n\}\in X$ konverguje k prvku $x\in X$ v metrice $\varrho$.

Ka�d� konvergentn� posloupnost je nutn� cauchyovsk�.

Uzav�en� metrick� podprostor (uzav�en� podmno�ina metrick�ho prostoru
s metrikou dan�ho prostoru) �pln�ho metrick�ho prostoru je �pln�.

\begin{ex} P��klady �pln�ch a ne�pln�ch metrick�ch prostor�:
   \begin{enumerate}
   \item P��kladem ne�pln�ho metrick�ho prostoru je $(\mathbb{Q},|\cdot|)$.
         V�me, �e posloupnost $x_n=(1+1/n)^n$, $n\in\mathbb{N}$,
         konverguje k ��slu $e=\exp(1)\not\in\mathbb{Q}$.
   \item Metrick� prostor $(\mathbb{R},|\cdot|)$ je �pln�.
   \item Metrick� prostory 
         $(\mathbb{R}^n,\varrho_1)$, $(\mathbb{R}^n,\varrho_2)$, 
         $(\mathbb{R}^n,\varrho_{\infty})$ jsou �pln�.
   \item Metrick� prostor $(C(I),\varrho_{\infty})$ je �pln�,
         prostory $(C(I),\varrho_{1})$ a $(C(I),\varrho_{2})$ nikoliv.
         Obecn� �plnost metrick�ho prostoru z�vis� na pou�it� metrice
         (viz diskuze o ekvivalentn�ch metrik�ch v��e).
   \item Diskr�tn� metrick� prostor $X$ je �pln�
         (cauchyovsk� posloupnosti maj� tvar $(x,x,\ldots,x,\ldots)$
         a konverguj� k $x$, kde $x\in X$).
   \end{enumerate}
\end{ex}

\section{Oper�tory na metrick�ch prostorech}

Nech� $X$, $Y$ jsou nepr�zdn� mno�iny.
Je-li pro ka�d� $x\in X$ p�i�azeno $y\in Y$, zapisujeme $y=Ax$ ($x\in X$, $y\in Y$),
��k�me, �e je definov�n oper�tor z $X$ do $Y$ (zna��me $A:\,X\rightarrow Y$).
N�kdy se uva�uje oper�tor definovan� jen pro jistou podmno�inu mno�iny $X$
(kter� ��k�me defini�n� obor oper�toru $A$). Tomu se zde vyhneme.
Je-li $y=Ax$ ($x\in X$, $y\in Y$), ��k�me prvku $x$ vzor prvku $y$,
prvku $y$ ��k�me obraz prvku $x$.
Mno�ina $R(A)\equiv\{y\in Y\,|\,(\exists x\in X)(y=Ax)\}\subset Y$
se naz�v� obor hodnot oper�toru $A$;
oper�tor $A:\,X\rightarrow Y$ je spojit�, pokud pro ka�dou posloupnost
$\{x_n\}\subset X$ plat� implikace $x_n\rightarrow x\;\Rightarrow\;Ax_n\rightarrow Ax$
(tj. pokud p�ev�d� konvergentn� posloupnosti v konvergentn�).

Ka�d� spojit� oper�tor je omezen�, tj. p�ev�d� omezen� mno�iny na omezen�.
Mno�inu $M$ nazveme omezenou, je-li 
$\mathrm{diam}(M)=\sup_{x,y\in M}\varrho(x,y)<\infty$.
Ekvivalentn�: je-li oper�tor neomezen�, nen� spojit�.

\begin{ex}
   P��klad nespojit�ho (line�rn�ho) oper�toru:
   Uva�ujme prostor $C[0,1]$ spojit�ch funkc� na intervalu $[0,1]$
   a jeho podprostor $C^1[0,1]\equiv\{f\in C[0,1]\,|\,f'\in C[0,1]\}$,
   oba s metrikou $\varrho_{\infty}$.
   Definujme $A\,:f\mapsto f'\,:C^1[0,1]\rightarrow C[0,1]$, $f\in C^1[0,1]$.
   Uva�ujme $f_n:\,x\mapsto x^n$ pro $n\in\mathbb{N}$.
   Zvolme $n\in\mathbb{N}$,
   pak $\varrho_{\infty}(f_n,f_1)=\max_{x\in[0,1]}|x-x^n|=
   \max_{x\in[0,1]}|x(1-x^{n-1})|\leq 1$.
   Mno�ina $\{f_n\}$ je tedy omezen�.
   Na druhou stranu $\varrho_{\infty}(Af_n,Af_1)=\varrho(nx^{n-1},0)=
   \max_{x\in[0,1]}|nx^n-1|=n$ je neomezen�.
   Oper�tor $A$ je neomezen�, a tedy nen� spojit�.
\end{ex}

\begin{ex}
   P��klad spojit�ho oper�toru:
   Uva�ujme op�t prostor $C[0,1]$
   a definujme oper�tor $A:\,f\mapsto\int_0^1f(x)\,dx:
   \,C[0,1]\rightarrow\mathbb{R}$, $f\in C[0,1]$.
   Pou�ijeme-li nap�. metriku $\varrho_{\infty}$,
   snadno se uk�e, �e oper�tor $A$ p�ev�d� konvergentn� posloupnosti
   na konvergentn�.
\end{ex}

\section{V�ta o pevn�m bod� a jej� aplikace}

Nech� $(X,\varrho)$ je metrick� prostor a $A:\,X\rightarrow X$ je oper�tor na $X$.
Prvku $x\in X$ ��k�me pevn� bod oper�toru $A$, pokud $Ax=x$.
Oper�tor $A$ naz�v�me kontrakce (kontraktivn� oper�tor), pokud existuje
$0\leq \alpha <1$ tak, �e $\varrho(Ax,Ay)\leq\varrho(x,y)$ pro ka�d� $x,y\in X$.

Ka�d� kontrakce je spojit�.

Banachova v�ta o pevn�m bod�:
Ka�d� kontrakce definovan� na �pln�m metrick�m prostoru m� pr�v� jeden pevn� bod.

D�kaz: Nech� $x_0\in X$ a definujme rekurzivn� 
$x_n=Ax_{n-1}=A^2x_{n-2}=\cdots=A^nx_0$ pro $n\in\mathbb{N}$.
Nech� $m,n\in\mathbb{N}$, $m\leq n$, pak
\[
   \begin{split}
      \varrho(x_m,x_n)&=\varrho(A^mx_0,A^nx_0)\leq\alpha^m\varrho(x_0,x_{n-m})\\
                      &\leq\alpha^m(\varrho(x_0,x_1)+\varrho(x_1,x_2)+\cdots
                       +\varrho(x_{n-m-1},x_{n-m})\\
                      &=\alpha^m(\varrho(x_0,x_1)+\varrho(Ax_0,Ax_1)+\cdots
                       +\varrho(A^{n-m-1}x_0,A^{n-m-1}x_1)\\
                      &\leq\alpha^m\varrho(x_0,x_1)(1+\alpha+\cdots
                       +\alpha^{n-m-1})\\
                      &\leq\frac{\alpha^m}{1-\alpha}\varrho(x_0,x_1)
   \end{split}
\]
P�itom jsme vyu�ili troj�heln�kov� nerovnosti~\ref{item:m3}
a sou�tu nekone�n� geometrick� �ady s kvocientem $\alpha<1$.
Pro zvolen� $\varepsilon>0$ a dostate�n� velk� $m$ je $\varrho(x_m,x_n)<\varepsilon$,
tak�e posloupnost $\{x_n\}$ je cauchyovsk� a v d�sledku �plnosti $(X,\varrho)$
m� limitu $x=\lim_{n\rightarrow\infty}x_n\in X$.
Proto�e $x_n\rightarrow x$, plyne ze spojitosti $A$, �e
$Ax=A\lim_{n\rightarrow\infty}x_n=\lim_{n\rightarrow\infty}Ax_n
=\lim_{n\rightarrow\infty}x_{n+1}=x$.
Na�li jsme tedy pevn� bod $x$ kontrakce $A$ v�etn� konstruktivn�ho n�vodu,
jak takov� bod nal�zt v�etn� odhadu rychlosti konvergence.
Jsou-li d�le $x,y\in X$ dva pevn� body kontrakce $A$, m�me
$\varrho(x,y)=\varrho(Ax,Ay)\leq\alpha\varrho(x,y)$.
Toto je ale mo�n� jen p��pad�, �e $\varrho(x,y)$, tj. $x=y$.
T�m jsme dok�zali jednozna�nost pevn�ho bodu.

\paragraph{Aplikace v�ty o pevn�m bod� p�i numerick�m �e�en� neline�rn�ch rovnic.}
�e�me rovnici 
\begin{equation}\label{eq:nelin}
   f(x)=0
\end{equation}
kde $f$ je re�ln� funkce re�ln� prom�nn� $x$.
Jednou z metod pro jej� �e�en� je metoda pevn�ho bodu.
P�epi�me rovnici~(\ref{eq:nelin}) do tvaru
\[
   x = g(x)
\]
($g$ se naz�v� itera�n� funkce).
Zvolme $x_0\in I$ a definujme
\begin{equation}\label{eq:iter}
   x_n = g(x_{n-1}), \; n\in\mathbb{N}.
\end{equation}
P�edpokl�dejme, �e existuje uzav�en� interval $I$ tak, �e $g(I)\subset I$,
itera�n� funkce je diferencovateln� na $I$ a plat� $|g'(x)|\leq 1$ pro $x\in I$
a n�jakou konstantu $0\leq K<1$.
Pak m�~(\ref{eq:nelin}) �e�en� $x^*\in I$ a posloupnost $\{x_n\}$ konverguje k $x^*$.

D�kaz: Dvojice $(I,|\cdot|)$ je uzav�en� metrick� podprostor $(\mathbb{R},|\cdot|)$,
je tedy je �pln�. Nech� $x,y\in I$, $x<y$. Pak
\[
   \left|\frac{g(y) - g(x)}{y-x}\right| = |g'(c)| \leq K,
\]
kde $c\in(x,y)$ (z v�ty o st�edn� hodnot�).
Odtud $|g(y) - g(x)| \leq K|y-x|$ pro ka�d� $x,y\in I$,
a tedy $g$ je kontrakce na $I$.
Z v�ty o pevn�m bod� tedy plyne, �e existuje $x^*\in I$ takov�,
�e $x^*=g(x^*)$ a z d�kazu t�to v�ty plyne $x^n\rightarrow x^*$.

Zpravidla se uv�d� slab�� formulace, kde p�edpokl�d�me, �e $g$
je spojit� diferencovateln� na n�jak�m otev�en�m okol� pevn�ho bodu $x^*$
a plat� na n�m $|g'(x)|\leq K<1$. Zvol�me-li $x_0$ v dostate�n� mal�m okol�
bodu $x^*$, konverguje posloupnost definovan� v~(\ref{eq:iter}) k $x^*$.

\begin{ex}
   Hledejme ko�en rovnice $f(x)=x^2-x-2=0$.
   Zvolme $g(x)=\sqrt{x+2}$. 
   Mo�n�ch voleb funkce $g$ je n�kolik,
   ne v�echny vedou ke konvergenci iterac�~(\ref{eq:iter}).
   Volbou kladn�ho znam�nka odmocniny jsme redukovali mno�inu pevn�ch bod�
   ze dvou na jeden.
   Zvolme $I=[0,7]$. Pak $g(0)\sqrt{2}\geq 0$ a $g(7)=3\leq 7$,
   je tedy $g(I)\subset I$.
   Derivac� $g$ podle $x$ m�me $g'(x)=1/(2\sqrt{x+2})$.
   Plat� $g'(x)<1$ pro ka�d� $x>-7/4$, tedy i pro $x\in I$.
   Odtud posloupnost iterac�~(\ref{eq:iter}) konverguje ke ko�enu
   rovnice $f(x)=0$ na $I$, tj. k $x^*=2$.
   Vol�me-li nap�. $x_0=7$, m�me
   $x_1=7.0000$, $x_2=3.0000$, $x_3=2.2361$, $x_4=2.0582$,
   $x_5=2.0145$, $x_6=2.0036$, $x_7=2.0009$ atd.
\end{ex}

\paragraph{Aplikace v�ty o pevn�m bod� v teorii diferenci�ln�ch rovnic.}
Nech� $G$ je otev�en� souvisl� mno�ina (oblast) v $\mathbb{R}^2$,
$(x_0,y_0)\in G$ a nech� $f$ je spojit� na $G$ 
a spl�uje Lipschitzovu podm�nku v argumentu $y$
($f$ je lipschitzovsk�, lipschitzovsky spojit�)
$|f(x,y_1)-f(x,y_2)|\leq M|y_1-y_2|$ pro ka�d� $(x,y_1)\in G$, $(x,y_2)\in G$,
kde $M>0$ je pevn� konstanta (nez�visl� na $x$, $y_1$, $y_2$,
tzv. Lipschitzova konstanta).
Pak existuje $\delta > 0$ tak, �e rovnice $y'(x) = f(x,y)$ m� pr�v� jedno �e�en�
$y(x)=\varphi(x)$ na intervalu $[x_0-\delta,x_0+\delta]$ spl�uj�c�
po��te�n� podm�nku $\varphi(x_0)=y_0$.

D�kaz: �e�en� rovnice $y'(x) = f(x,y)$ s po��te�n� podm�nkou $\varphi(x_0)=y_0$
m��eme zapsat ve tvaru
\begin{equation}\label{eq:int_eq}
   \varphi(x) = y_0 + \int\limits_{x_0}^x f(x,\varphi(t))\,dt.
\end{equation}
Funkce $f$ je na $G$ spojit� (poznamenejme, �e na funkci $f$
se m��eme d�vat jako na oper�tor $f\,\mathbb{R}^2\rightarrow\mathbb{R}$),
tedy je omezen�, tj. existuje $K>0$ tak, �e $|f(x,y)|\leq K$ na n�jak�
podoblasti $G'\subset G$ takov�, �e $(x_0,y_0)\in G'$.
Vyberme $\delta>0$ tak, �e oblast $G'$ obsahuje obd�ln�k
$R\equiv [x_0-\delta,x_0+\delta]\times [y_0-K\delta,y_0+K\delta]$
a z�rove� $\delta\leq 1/M$.
Ozna�me $C^*\equiv\{\psi\in C[x_0-\delta,x_0+\delta]\,|\,
(\forall x\in[x_0-\delta,x_0+\delta])(|\varphi(x)-y_0|\leq K\delta)\}$
a pou�ijme metriku $\varrho_{\infty}$.
Metrick� prostor $(C^*,\varrho_{\infty})$ je uzav�en� podprostor prostoru
$(C[x_0-\delta,x_0+\delta],\varrho_{\infty})$, kter� je �pln�, a tedy
$(C^*,\varrho_{\infty})$ je rovn� �pln�.
Definujme oper�tor $\psi=A\phi$, $\phi\in C^*$ p�edpisem
\[
   \psi(x)=(A\varphi)(x)\equiv y_0+\int\limits_{x_0}^x f(t,\varphi(t))\,dt.
\]
Uk�eme, �e je $A:\,C^*\rightarrow C^*$ a �e $A$ je kontrakce.
Abychom ov��ili, �e $A$ je oper�tor do $C^*$, mus�me ov��it podm�nku
$(\forall x\in[x_0-\delta,x_0+\delta])(|(A\varphi)(x)-y_0|\leq K\delta)$.
Plat�
\[
   |(A\varphi)(x)-y_0|=\left|\int\limits_{x_0}^xf(t,\varphi(t))\,dt\right|
                      \leq\int\limits_{x_0}^x|f(t,\varphi(t))|\,dt
                      \leq K|x-x_0|\leq K\delta.
\]
Abychom ov��ili kontraktivnost oper�toru, nech� $\varphi,\psi\in C^*$, 
$x\in[x_0-\delta,x_0+\delta]$, pak
\[
  \begin{split}
   |(A\varphi)(x)-(A\psi)(x)|&=
    \left|\int\limits_{x_0}^xf(t,\varphi(t))-f(t,\psi(t)\,dt\right|\\
    &\leq\int\limits_{x_0}^x|f(t,\varphi(t))-f(t,\psi(t)|\,dt\\
    &\leq M\int\limits_{x_0}^x|\varphi(t)-\psi(t)|\,dt
    \leq M\delta\,\varrho_{\infty}(\varphi,\psi),
  \end{split}
\]
odkud $\varrho_{\infty}(A\varphi,A\psi)\leq M\delta\varrho(\varphi,\psi)
=\alpha\varrho(\varphi,\psi)$, kde $\alpha\equiv M\delta<1$.
Z v�ty o pevn�m bod� existuje tedy $\varphi\in C^*$ tak, �e
$A\varphi=\varphi$, tj. existuje pr�v� jedno �e�en� integr�ln� 
rovnice~(\ref{eq:int_eq}) v $C^*$.


\chapter{V�var z funkcion�ln� anal�zy}
%%%%%%%%%%%%%%%%%%%%%%%%%%%%%%%%%%%%%%

\section{Line�rn� prostory}
%%%%%%%%%%%%%%%%%%%%%%%%%%%

Nech� $\mathbb{R}$ je t�leso re�ln�ch ��sel.
Nepr�zdnou mno�inu $V$ nazveme \emph{vektorov�m prostorem} (prvky naz�v�me vektory)
nad t�lesem $\mathbb{R}$, je-li definov�na operace s��t�n� 
$+:V\times V\rightarrow V$ a operace n�soben� skal�rem 
$\cdot:\mathbb{R}\times V\rightarrow V$ tak, �e:
\begin{enumerate}
   \item $(\forall x,y\in V)(x+y=y+x)$ -- komutativita $+$,
   \item $(\forall x,y,z\in V)((x+y)+z=x+(y+z))$ -- asociativita $+$,
   \item $(\exists 0\in V)(\forall x\in V)(0+x=x)$ -- existence nulov�ho prvku 
           (neutr�ln� prvek vzhledem ke s��t�n�),
   \item $(\forall x\in V)(\exists (-x)\in V)(x+(-x)=0)$ -- existence opa�n�ho prvku,
   \item $(\forall \alpha,\beta\in\mathbb{R})(\forall x\in V)
          (\alpha(\beta x)=(\alpha\beta x))$,
   \item $(\forall x\in V)(1\cdot x = x)$,
   \item $(\forall \alpha,\beta\in\mathbb{R})(\forall x\in V)
         ((\alpha+\beta)x=\alpha x+\beta x)$ -- distributivita I,
   \item $(\forall \alpha\in\mathbb{R})(\forall x,y\in V)
         (\alpha(x+y)=\alpha x+\beta y)$ -- distributivita II.
\end{enumerate}
Pro $x,y\in V$ definujeme operaci ode��t�n� vztahem $x-y=x+(-y)$,
kde $-y$ je prvek opa�n� k $y$. Symbol $\cdot$ p�i n�soben� budeme zpravidla
vynech�vat.

D�sledky definice: nulov� prvek je jedin�, ke ka�d�mu prvku z $V$ existuje
pr�v� jeden opa�n� prvek, pro ka�d� $x\in V$ plat� $-x=(-1)\cdot x$
a $0\cdot x = 0$ (pozor: nula na lev� stran� je nula v $\mathbb{R}$, nula na prav�
stran� je nulov� prvek ve $V$!).

Podmno�ina $V_0\subset V$ vektorov�ho prostoru $V$ je \emph{podprostor}
prostoru $V$, pokud je uzav�en� vzhledem ke s��t�n� vektor� a n�soben� skal�ru
a vektoru; zna��me $V_0\subset\subset V$.

\begin{ex}
   P��kladem vektorov�ho prostoru je prostor $n$-tic ($n\in\mathbb{N}$)
   re�ln�ch ��sel $\mathbb{R}^n$ s obvyklou definic� s��t�n� a n�soben� skal�rem.
   Nulov�m prvkem je vektor $(0,\ldots,0)$.
\end{ex}

\begin{ex}
   Mno�ina $C[a,b]$ spojit�ch funkc� na uzav�en�m intervalu je vektorov� prostor
   (sou�et spojit�ch funkc� je spojit� funkce, n�sobek skal�ru a spojit� funkce je
   op�t spojit� funkce).
   Nulov� prvek je nulov� funkce na $[a,b]$.
\end{ex}

\begin{ex}
   Mno�ina $P_n[a,b]$ polynom� stupn� nejv��e $n\in\mathbb{N}$, 
   tj. v�ech funkc� ve tvaru
   \[
      p:\;t\mapsto\sum\limits_{i=0}^{n}\alpha_i t^i, \; t\in[a,b],
   \]
   kde $\{\alpha_i\}_{i=0}^n\subset\mathbb{R}$.
   Proto�e ka�d� polynom je spojit� funkce a $P_n[a,b]$ je uzav�en� vzhledem
   ke s��t�n� a n�soben� skal�rem, plat� $P_n[a,b]\subset\subset C[a,b]$.
\end{ex}

\section{Banachovy a Hilbertovy prostory}
%%%%%%%%%%%%%%%%%%%%%%%%%%%%%%%%%%%%%%%%%

Zobrazen� $\|\cdot\|:\;V\rightarrow \mathbb{R}$ definovan� na line�rn�m prostoru $V$
nazveme \emph{normou} na $V$, jestli�e
\begin{enumerate}
   \item $(\forall x\in V)(\|x\|=0\;\Leftrightarrow x=0)$,
   \item $(\forall x,y\in V)(\|x+y\|\leq\|x\|+\|y\|)$ -- troj�heln�kov� nerovnost,
   \item $(\forall \alpha\in\mathbb{R})(\forall x\in V)(\|\alpha x\|=|\alpha|\|x\|)$
         -- homogenita.
\end{enumerate}

D�sledky definice:
\begin{itemize}
   \item $(\forall x\in V\setminus\{0\})(\|x\|>0)$:
         je-li $x\in V$, pak z troj�heln�kov� rovnosti a homogenity plyne
         $0=\|x-x\|\leq\|x\|+\|-x\|=\|x\|+\|x\|=2\|x\|$, tak�e $\|x\|\geq 0$;
         ale rovnost nastane jen v p��pad�, �e $x=0$;
   \item $(\forall x,y\in V)(|\|x\|-\|y\||\leq \|x-y\|)$;
   \item norma je spojit� funkce prvk� z $V$.
\end{itemize}

Prostor $V$ s definovanou normou $\|\cdot\|$ naz�v�me 
\emph{normovan� line�rn� prostor}.

\begin{ex}
   Nech� $x\in\mathbb{R}^n$, $x=(\alpha_1,\ldots,\alpha_n)$.
   Na $\mathbb{R}^n$ m��eme definovat normy:
   \[
         \|x\|_1 \equiv \sum\limits_{i=1}^n |\alpha_i|,\;\;
         \|x\|_2 \equiv \sqrt{\sum\limits_{i=1}^n \alpha_i^2},\;\;
         \|x\|_{\infty} \equiv \max\limits_{i=1,\ldots,n}|\alpha_i|.
   \]
\end{ex}

\begin{ex}
   Nech� $f\in C[a,b]$.
   Analogicky jako v p��pad� $\mathbb{R}^n$ lze na $C[a,b]$ definovat normy:
   \[
      \begin{split}
         \|f\|_1 \equiv \int\limits_a^b|f(t)|\,dt,\;\;
         \|f\|_2 \equiv \sqrt{\int\limits_a^b|f(t)|^2\,dt},\;\;
         \|f\|_{\infty} \equiv \max\limits_{t\in[a,b]}|f(t)|.
      \end{split}
   \]
\end{ex}

�pln� prostory: Normovan� prostor $V$ je \emph{�pln�}, jestli�e ka�d� 
Cauchyovsk� posloupnost%
\footnote{Posloupnost $\{x_n\}\subset V$ je Cauchyovsk�, pokud 
$(\forall \varepsilon<0)(\exists N\in\mathbb{N})(\forall m,n>N)
(\|x_m-x_n\|<\varepsilon)$. Jin�mi slovy, pokud jsou si prvky t�to posloupnosti
libovoln� bl�zk� po��naje n�jak�m indexem $N$. Plat�, �e ka�d� konvergentn�
posloupnost je Cauchyovsk�, naopak to v�ak platit nemus�.}
ve $V$ konverguje k prvku ve $V$.
�pln�mu normovan�mu line�rn�mu prostoru ��k�me \emph{Banach�v prostor}.

\begin{ex}
   T�leso re�ln�ch ��sel $\mathbb{R}$ ch�p�no jako vektorov� prostor nad $\mathbb{R}$
   s normou $\|x\|=|x|$, $x\in\mathbb{R}$, je Banach�v prostor,
   nebo� ka�d� Cauchyovsk� posloupnost re�ln�ch ��sel je konvergentn� v $\mathbb{R}$.
   Stejn� tak prostor $\mathbb{R}^n$, $n\in\mathbb{N}$ je Banach�v.
\end{ex}

\begin{ex}
   T�leso $\mathbb{Q}$ racion�ln�ch ��sel ch�p�no jako vektorov� nad $\mathbb{Q}$
   nen� Banach�v prostor, nebo� $\{x_n\}\subset\mathbb{Q}$, kde
   $x_n\equiv (1+n^{-1})^n$, $n\in\mathbb{N}$, konverguje v $\mathbb{R}$
   k ��slu $e=\exp(1)\not\in\mathbb{Q}$.
\end{ex}

\begin{ex}
   Prostor $C[a,b]$ s normou $\|\cdot\|_{\infty}$
   je Banach�v prostor, nebo� konvergence spojit�ch funkc� v norm� 
   $\|\cdot\|_{\infty}$ je ekvivalentn� stejnom�rn� konvergenci spojit�ch funkc�,
   jej�� limita je spojit� funkce na $[a,b]$.
\end{ex}

\begin{ex}
   Prostor $C[a,b]$ s normou $\|\cdot\|_2$ nen� Banach�v.
   Pro jednoduchost uva�ujme $a=-1$ a $b=1$ a pro $n\in\mathbb{N}$ definujme
   \[
      f_n\,:t\mapsto\begin{cases}
          -1&\text{pro }t\in(-1,-\frac{1}{n})\\
           1&\text{pro }t\in(\frac{1}{n},1)\\
           nt&\text{pro }t\in[-\frac{1}{n},\frac{1}{n}].
           \end{cases}
   \]
   Z�ejm� $\{f_n\}\subset C[-1,1]$.
   Definujeme-li d�le
   \[
      f\,:t\mapsto\begin{cases}
          -1&\text{pro }t\in(-1,0)\\
           1&\text{pro }t\in(0,1)\\
           0&\text{pro }t=0.
           \end{cases}
   \]
   Samoz�ejm� $f\not\in C[-1,1]$.
   Posloupnost $\{f_n\}$ v�ak konverguje k $f$ v norm� $\|\cdot\|_2$, nebo�
   \[
     \begin{split}
      \|f-f_n\|_2&=\int\limits_{-1}^1|f(t)-f_n(t)|\,dt
                  =\int\limits_{-\frac{1}{n}}^0 (1+nt)\,dt
                  +\int\limits_0^{\frac{1}{n}}(1-nt)\,dt
                  =\frac{1}{2n}+\frac{1}{2n}=\frac{1}{n}.
     \end{split}
   \]
   Proto�e $\frac{1}{n}\rightarrow 0$ pro $n\rightarrow\infty$,
   tak� $\|f-f_n\|_2\rightarrow 0$, tak�e $f_n\rightarrow f$.
   Na�li jsme tedy posloupnost spojit�ch funkc�, kter� nekonverguje ke spojit� funkci
   v norm� $\|\cdot\|_2$.
\end{ex}

Pozn�mka: Odpov�� na ot�zku, zda je normovan� prostor Banach�v, 
m��e siln� z�viset na zvolen� norm�, jak bylo z�ejm� z p�edchoz�ho p��kladu.
Na prostorech kone�n� dimenze (prostory s kone�nou b�z�) je odpov�� jasn�:
je-li $\{x_1,\ldots,x_n\}\subset V$ b�ze $V$ a pro $x\in V$
je $x=\sum\limits_{i=1}^n\alpha_ix_i$ ($\{\alpha_i\}_{i=1}^n\subset\mathbb{R}$)
a definujeme-li $\|x\|_{\infty}\equiv\max\limits_{i=1,\ldots,n}|\alpha_i|$,
lze uk�zat, �e $V$ je pak �pln� v t�to norm�, a tedy Banach�v.
Proto�e jsou v prostorech kone�n� dimenze v�echny normy 
ekvivalentn�%
\footnote{Dv� normy $\|\cdot\|_a$ a $\|\cdot\|_b$ na $V$ jsou ekvivalentn�,
pokud existuj� kladn� konstanty $\alpha$ a $\beta$ takov�, �e
$(\forall x\in V)(\alpha\|x\|_a\leq\|x\|_b\leq\beta\|x\|_a)$.},
jsou kone�n�-dimenzion�ln� normovan� prostory �pln� (Banachovy)
v ka�d� norm�!

Nech� $V$ je line�rn� vektorov� prostor.
Zobrazen� $\langle\cdot,\cdot\rangle:\,V\times V\rightarrow\mathbb{R}$
nazveme skal�rn� sou�in na $V$, pokud plat�:
\begin{enumerate}
   \item $(\forall x\in V)(\langle x,x\rangle\geq 0\;
            \&\;\langle x,x\rangle=0\;\Leftrightarrow x=0)$,
   \item $(\forall x,y\in V)(\langle x,y\rangle=\langle y,x\rangle)$ -- symetrie,
   \item $(\forall \alpha\in V)(\forall x,y\in V)
          (\langle\alpha x,y\rangle=\alpha\langle x,y\rangle)$ -- homogenita,
   \item $(\forall x,y,z\in V)
          (\langle x,y+z\rangle = \langle x,y\rangle+\langle x,z\rangle)$ 
          -- aditivita.
\end{enumerate}
Prostor $V$ s definovan�m skal�rn�m sou�inem naz�v�me
\emph{unit�rn� prostor}, pop�. Euklid�v prostor.

D�sledky definice:
\begin{itemize}
   \item $(\forall \alpha\in V)(\forall x,y\in V)
          (\langle x,\alpha y\rangle=\alpha\langle x,y\rangle)$;
   \item $(\forall x,y,z\in V)
          (\langle x+y,z\rangle = \langle x,z\rangle+\langle y,z\rangle)$;
   \item Schwarzova nerovnost: $(\forall x,y\in V)
         (|\langle x,y\rangle|\leq\sqrt{\langle x,x\rangle}\sqrt{\langle y,y\rangle})$.
\end{itemize}

Definujeme-li pro $x\in V$ $\|x\|=\sqrt{\langle x,x\rangle}$,
dostaneme porovn�n�m s definic� normy za pou�it� vlastnost� skal�rn�ho sou�inu
a d�sledk� t�to definice, �e $\|\cdot\|$ je norma na $V$.
Je-li tedy prostor $V$ unit�rn�, je z�rove� normovan�.
Je-li nav�c $V$ �pln� v t�to norm�, tj. Banach�v, naz�v� se \emph{Hilbert�v prostor}.

\begin{ex}
   Nech� $x,y\in\mathbb{R}^n$, $x=(\alpha_1,\ldots,\alpha_n)$,
   $x=(\beta_1,\ldots,\beta_n)$.
   Na $\mathbb{R}^n$ lze definovat Euklid�v skal�rn� sou�in vztahem
   \[
      \langle x,y\rangle=\sum\limits_{i=1}^n \alpha_i \beta_i.
   \]
   Tento skal�rn� sou�in definuje normu $\|\cdot\|_2$ na $\mathbb{R}^n$.
\end{ex}

\begin{ex}
   Nech� $f,g\in C[a,b]$.
   Obdobn� i zde m��eme definovat skal�rn� sou�in vztahem
   \[
      \langle x,y\rangle=\int\limits_a^b f(t)g(t)\,dt.
   \]
   Tento skal�rn� sou�in definuje normu $\|\cdot\|_2$ na $C[a,b]$.
\end{ex}

�ekneme, �e dva vektory $x,y\in V$ jsou \emph{ortogon�ln�}, 
pokud $\langle x,y\rangle=0$.



\begin{thebibliography}{10}
   \bibitem{meznik}
   Ivan Mezn�k, Ji�� Kar�sek, Josef Mikl��ek.
   \emph{Matematika I pro strojn� fakulty.}
   SNTL, 1992.
   \bibitem{rektorys}
   Karel Rektorys a spol.
   \emph{P�ehled u�it� matematiky I.}
   Prometheus, 1996.
   \bibitem{rektorys2}
   Karel Rektorys a spol.
   \emph{P�ehled u�it� matematiky II.}
   Prometheus, 1996.
   \bibitem{kofron}
   Josef Kofro�.
   \emph{Oby�ejn� diferenci�ln� rovnice v re�ln�m oboru.}
   Karolinum, 2004.
   \bibitem{tenenbaum}
   Morris Tennebaum, Harry Pollard.
   \emph{Ordinary Differential Equations.}
   Dover Publications, 1985.
   \bibitem{kolmogorov}
   A. N. Kolmogorov, S. V. Fomin.
   \emph{Introductory Real Analysis.}
   Dover Publications, 1975.
   \bibitem{nagy1}
   Jozef Nagy,
   \emph{Soustavy oby�ejn�ch diferenci�ln�ch rovnic.}
   SNTL, 1983.
\end{thebibliography}

\end{document}
